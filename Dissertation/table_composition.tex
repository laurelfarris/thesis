% Table - Spectral lines in atmosphere
\begin{deluxetable}{c c c c}
    \tablewidth{\textwidth} % default = \pagewidth, natural width = 0pt
    \tablecaption{
        Characteristic spectral lines in the solar atmosphere.
        \label{tab:comp}}
    \tablehead{
        \colhead{Species} &
        \colhead{Wavelength [\AA{}]} &
        \colhead{Layer} &
        \colhead{Temperature [K]}
    }
    \startdata
        \myion{Cl}{I} & \nodata & Low to middle Chrom. & \nodata \\
        \myion{O}{I} & \nodata & Low to middle Chrom. & \nodata \\
        \myion{C}{I} & \nodata & Low to middle Chrom. & \nodata \\

        \myion{Mg}{II} k line& 2796 & Chromosphere & \nodata \\
        \myion{Mg}{II} wing & 2830 & \nodata & \nodata \\
        \myion{Ca}{II} H & 3969 & Chromosphere & \nodata \\
        \myion{Ca}{II} K & 3934 & Chromosphere & \nodata \\

        \myion{C}{II} & 1334 & Upper Chrom, low TR & \nodata \\
        \myion{C}{II} TR line & 1335 & Upper Chrom, low TR & \nodata \\
        \myion{Si}{IV} & 1394/1403 & Transition Region & 65000 \\
        \myion{Si}{IV} TR line & 1400 & Transition Region & 65000 \\

        \myion{O}{IV} & $\sim$1401 & \nodata & 150000 \\
        \myion{Fe}{XIII} & 10747 & Corona & \nodata \\
    \enddata
\end{deluxetable}
