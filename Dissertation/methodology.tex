\section{Methodology}

\mynote[Thursday, October 18, 2018]{
    Rough outline for now;
    appropriate content from first research article will be copied here eventually.}

Outline:
\begin{itemize}
    \item Observations (flare SOL-Feb-15-2011)
    \item Instruments/Data (SDO, IRIS)
    \item Data reduction (alignment, saturation, \texttt{aia\_prep.pro})
    \item Analysis (FT, WA)
\end{itemize}

\subsection{Observations}

% {---

\subsubsection{Flare(s)}

\mynote{Flare on 2011 February 15 took place ``close to disk center''.
How close is ``close''? What are the difficulties in analyzing target close
to the solar limb? Opacity issues?}

\subsubsection{Data}

\paragraph{The Solar Dynamics Observatory (SDO)}

\subparagraph{The Atmospheric Imaging Assembly (AIA)}

\mynote{From propopsal:}
The Atmospheric Imaging Assembly (AIA) \citep{Lemen2012, Boerner2012}
on board \textit{SDO} primarily collects EUV emission
that originates in the corona.
In addition it obtains thermal UV continuum data in
two bandpasses centered on 1600\AA{} and 1700\AA{}. AIA 1700\AA{} samples the
lower atmosphere around $T = 10^{3.7}$ K, close to $T_{min}$. AIA 1600\AA{}
samples the transition region and upper chromosphere, and contains emission
from the \ion{C}{4} 1548\AA{} line, as well as UV continuum. Both data sets
are obtained at a cadence of 24 seconds, with a spatial resolution of $\sim$0.6
arcseconds per pixel.

\mynote{09 October 2018 - From current state of article (Phase 1):}
The data used to analyze this flare is provided by
AIA \citep{Lemen2012},
one of the instruments on board
SDO \citep{Pesnell2012}.
AIA obtains full disk images throughout the solar atmosphere using narrow band
filters centered on 10 different wavelengths, two of which provide measurements
of thermal UV emission from the chromosphere.
The 1700\AA{} channel mostly contains
continuum emission from the temperature minimum, and the
1600\AA{} channel covers both continuum emission and the
\ion{C}{4} spectral line in the upper photosphere and transition region.
Both channels have a cadence of 24 seconds and
spatial size scale of 0.6 arcseconds per pixel.

\subparagraph{The Helioseismic and Magnetic Imager (HMI)}

The Helioseismic and Magnetic Imager (HMI)
is one of three instruments on board
the \textit{Solar Dynamics Observatory (SDO)}. It obtains four types of
filtergrams around the \ion{Fe}{1} line at 6173\AA{}. These filtergrams are
in the form of dopplergrams, vector magnetograms, line-of-sight magnetograms,
and continuum intensity images \citep{hmi}.

For this project, wavelet analysis and
Fourier techniques will be applied to the
continuum intensity data from HMI, starting with the same flare analyzed by
Milligan et al. 2017 (draft),
since it has already been confirmed to show three-minute oscillations. The
analysis will be expanded to include times before and after the flare, in addition
to the time during the actual event, in order to see if this phenomena exists
independently of rapid energy release events.

\mynote{
    {\small\color{magenta}07 June 2017}\\
    According to the preliminary results from Milligan et al. 2017, the
    flare from 12 February 2011 took place at 01:44 UT (or, from the plotted
    emission, started around 1:20 and ended around 3:10). For the purpose of
    looking at this area of the sun before, during, and after the flare, I
    downloaded data from 00:00 to 05:00, padding the time during the flare by a
    little over an hour, though a bigger time difference may be required. Depending
    on just how large this time difference ends up being, it may be necessary to
    account for the rotation of the sun; a full rotation takes about 26 days at the
    equator and 30 days at the poles. So, for HMI images, a spatial resolution of
    0.5 arcseconds per pixel calculates to 12.4 pixels per hour, or 62 pixels over
    5 hours, at the equator. In fact, a single sunspot shifts by 85 pixels over the
    course of 5 hours. It is at higher latitudes, where the rotation speed is
    actually slower, but curvature could also be playing a role. This isn't even
    where the flare is. But higher curvature would mean that each pixel covers more
    physical space, which would mean it should take even longer to rotate through a
    single pixel\ldots what is happening??
}

\mynote{
    09 June 2017\\
The five-hour span of data should have produced 400 images, each 45 seconds
apart. However, only 382 were downloaded, so a slight detour was taken during
which I wrote a subroutine to convert the time information from the header to
seconds since midnight on February 15, 2011, then subtracted each time from the
time after it to see how much time was skipped, and where. Turns out, 15
consecutive images are missing between image 21 and 22. The other three are
missing from random locations throughout the rest of the time series (and are
not consecutive). The latter can be supplemented with a linear interpolation
between the two images around them, but the initial chunk is too large for
interpolation. The first 20 images will probably just be excluded from the
study, unless that time period is absoultely necessary, in which case I would
have to go back and see exactly why I'm missing those images. It may just be an
error on my end, and I may be able to retrieve them after all. But again, I
still don't know how much ``before'' time I need for any given flare.
}

\subsubsection{The Interface Region Imaging Spectrograph (IRIS)}

\begin{itemize}
    \item Lockheed Martin Solar and Astrophysical Laboratory (LMSAL)
    \item Launched June 2013; sun-synchronous, low-Earth orbit.
    \item Goals: Understand how chromosphere is energized
    \item Revealed complexity, density/temperature contrasts in the interface
        region.
    \item 19 cm Cassegrain telescope
        \begin{itemize}
            \item dual-range UV spectrograph (imaging) with 1 second cadence,
                0.3'' spatial resolution, $<$ 1\AA{} spectral resolution.
            \item slit-jaw imager (SJI) with four passbands:
                \begin{enumerate}
                    \item CII 1335\AA{} (transition region line)
                    \item SiIV 1403\AA{} (transition region line)
                    \item Mg II k 2796\AA{} (chromospheric line)
                    \item 2830\AA{} (photospheric passband)
                \end{enumerate}
        \end{itemize}
    \item ``NASA Small Explorer developed and operated by LMSAL with mission operations
    executed at NASA Ames Research center and major contributions to downlink
    communications funded by ESA and the Norwegian Space Center.''
    (Bryans et al. 2016)
\end{itemize}

To expand on the results from HMI, the same flares will be analyzed using data
from IRIS, which provides spectroscopic data in addition to intensity images,
which will greatly constrain the behavior of individual lines.

IRIS - intensity and velocity; cadence and size

Initial data: compare with SDO, x-rays.
Analysis - Gaussian fit, Fourier transform --> 3 min. power

Spectroscopy - rastering and doppler shifts.

Rastering requires a balance between the spatial coverage and temporal coverage on
any given area.

Time vs. locations\ldots ?

There is also a balance between number of areas observed and the total amount of
time observed for each one (see meeting notes).
% ---}

\subsection{Beginning of project - HMI continuum (pre-proposal)}
The first thing I did was look at HMI continuum (intensity) images from the
same flare analyzed by Milligan et al. (2017, draft). Since they had already
confirmed the presence of QPPs in this flare using EVE and RHESSI,
this was a good flare to use to test my potential methods.

\subsection{Quiet sun}

\begin{itemize}
    \item First, test methods on quiet sun to see if 3mOs appear where
        they have previously been observed relative to sunspots.
    \item Initial approach: BDA with dz = 1 hour to
        reproduce power spectrum from Milligan2017 (Figure 5), plus
        before and after.
        \begin{itemize}
            \item input signal composed of integrated emission from AR 11158
                between 00:30 and 03:30, separated into BDA time segments
                one hour each.
            \item show power spectrum for periods between 50 and 400 seconds
                (frequencies between 20 and 0.2 mHz), rather
                than using a filter to cut off periods higher than 400 seconds.
        \end{itemize}
        \mynote{Did they divide integrated emission by number of pixels?
        If so, did they do this before or after calculating the Fourier
        transform?}
    \item Show 3-minute power map for same three time segments (BDA).
        \mynote{Does this support the conclusions from Milligan2017?}
\end{itemize}


\clearpage
\begin{figure}\centering
    %\includegraphics[width=0.5\textwidth]{aia1600quiet.pdf}
    %\epsscale{1.2}
    \plottwo{aia1600quiet.pdf}{aia1700quiet.pdf}
    \caption{Spatial distribution of 3-minute power (top row)
    and 5-minute power (bottom row) for
    AIA 1600\AA{} (left column) and AIA 1700\AA{} (right column)
    from 00:00 to 01:44 on 15 February 2011.
    The black contours outline the approximate location of the umbra
    and penumbra using HMI continuum data in the center of the time
    series used to generate the power maps.
    \label{quiet}}
\end{figure}

\paragraph{05 October 2018 - single page summary to send to Professor James}
This was a test of my methods on non-flaring data to see if the
oscillatory power at 3 and 5 minutes is distributed the way I would expect.
The small flare that occurred around 00:40 took place on the far left
side of AR11158, so the region on the far right side was used here.
The input time segment was between 00:00 and 01:44, for a total of
about 1.75 hours (260 frames), which gave a frequency resolution of
0.16 mHz.
The contours outlining the umbra and penumbra were generated by defining
the boundary between umbra and penumbra as 0.6 multiplied by quiet sun,
and the boundary between penumbra nd quiet sun as 0.9 multiplied by quiet sun.
The contours and data were adjusted by eye for alignment, since both data
sets were aligned independently to correct for rotation.
I do not yet know how to best define the contour levels for magnetic field
strength, so that is still in progress.

The suppression of the 5-minute period above the sunspot umbra is expected,
but I'm not sure what to make of the power maps for the 3-minute period.

To do:
\begin{itemize}
    \item Show center image of AR in each channel above the power maps for
        context.
\end{itemize}

\clearpage

Temporal resolution: reduce dz to 64 frames (25 minutes)

\subsection{Data Analysis}

\subsubsection{``Discrete wavelet analysis''}
Using dz=64, show WA for integrated emission, see if the portion between
01:30 and 02:30 looks similar to Milligan2017 WA.

Use data from WA plot to
show 3-minute power as function of time, averaged over frequencies within
$\Delta\nu$.
Milligan2017 did sort of show this,
if the wavelet analysis plot was collapsed down to time vs power, only for a
single frequency (or range around that frequency).
\mynote{What time/frequency resolution was obtained by Milligan2017 for the
wavelet analysis?}



