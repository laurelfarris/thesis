%\section{Literature Review}
% 25 February 2019 - This should be mixed in with background, which
% will eventually be part of the Introduction for final dissertation,
% if I follow what seems to be a common organization scheme.

%-
%- 17 October 2018 (early morning of 18th)
%- Moved <~100 lines of text from bottoma of background.tex
%-   to separate literature review from background.
%-
%
%
%



Literature review ``section''


% Observations of QPPs in NT emission (if time..)

% Observations of QPPs in thermal emission during flares
Currently, there are few studies of QPPs in the lower atmosphere during flares.
\citep{Brosius2016} studied emission from the \ion{C}{1} line, and found
a period of $\sim$170 seconds. They
attributed this periodicity to the rate of injection of
non-thermal electrons.
Further investigation of QPPs in thermal emission was carried out by
\cite{Milligan2017}, who found enhancements of the 3-minute oscillations
in Lyman continuum and Lyman-$\alpha$ line emission.
However, no signature of a 3-minute oscillation was found in
the X-ray data generated by the flare, indicating
that the chromospheric response was not a
reflection of the rate of energy input.
These findings supported the idea that the chromosphere naturally
responds at its cutoff frequency, regardless of the periodicity of the
energy injection rate.
This result did not support the conclusion from
\cite{Sych2009}, who
proposed that the 3-minute waves leaked from sunspots into the upper
atmosphere by propagating along magnetic field lines, and triggered
the magnetic reconnection and subsequent energy release and particle
acceleration.

\mynote{Applications to my own research:
How/when/why does the 3-minute power change when \emph{other}
frequencies \emph{don't}?
Aka, when does its behavior deviate from the others?
Why do the other frequencies changes at all?
\textbf{What is physically happening??}}

To investigate the transfer of energy during flares, \cite{Monsue2016}
conducted a pilot study of the spatial and temporal flare response of acoustic
oscillations in H$\alpha$ emission.
They were able to preserve both the two-dimensional spatial information
and the temporal information in the original data
with the power and frequency of the output of the Fourier transforms
by incorporating a technique called ``frequency-filtered amplitude movies''
from \cite{Jackiewicz2013}.
They reported a suppression of frequencies
between 1 and 8 mHz during the main phase, and an enhancement of frequencies
between 1 and 2 mHz before and after the flare. They suggested that the
suppression of lower frequencies could be evidence of the conversion of energy
from acoustic to thermal, and that the pre-flare enhancement may have been
indicative of an instability in the chromosphere.
In the conclusion of this study, they encouraged
the further investigation of
earlier time frames before the flare precursor, along with the
inclusion of additional cases to increase the statistical significance of the
findings.

% Temporal dependence of QPPs (flare phases). Science question - Time!!!
% Move this to methodology?
\cite{Hayes2016} examined oscillations during
both the impulsive and decay phase of an X-class flare, and considered
a combination of possible mechanisms for producing QPPs, where the
impulsive phase involved rapid injections from non-thermal particle
acceleration and subsequent heating of the surrounding plasma, while
thermal signatures persisted throughout the decay phase.
Propagating slow magnetoacoustic waves, such as the chromospheric
3-minute oscillations, show characteristic quasi-periodic patterns.
Several recent papers discuss the relationship between QPPs and
chromospheric oscillations.

% Move this to methodology?
The observations were acquired using
Lyman continuum data from \textit{SDO}/EVE,
Lyman-$\alpha$ line emission from \textit{GOES}/EUVS,
1600\AA{} and 1700\AA{} continuum from \textit{SDO}/AIA, and
HXR emission from \textit{RHESSI}.
EVE observes the sun as a star, with disk-integrated data.

\begin{framed}

    \mynote{From current paper (09 October 2018), on what appears to be a
    contradiction in results between Monsue and Milligan, though as I noticed
    later, Monsue's results were for subregions, whereas Milligan did not
    have spatially resolved data:}

    There are a few possible explanations of this
    contradiction in behavior of thermal emission during the flare.
    It may reflect a physical difference in the configuration or dynamics
    between the individual flares themselves.
    It may be a consequence of
    one of the general challenges involved in studying the solar
    chromosphere: the ambiguity in determining the formation height of various
    types of emission.
    H$\alpha$ is particularly difficult to diagnose, since it is
    emitted all throughout the chromosphere, in both
    $\beta > 1$ (thermally dominated) and
    $\beta < 1$ (magnetically dominated) regions.
    Wave behavior changes drastically from one environment to the other, so
    knowledge of the observed location can be crucial.

\end{framed}
