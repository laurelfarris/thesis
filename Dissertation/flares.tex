%% Flares {---

\section{The flaring chromosphere}

\mynote[Mon Feb 25 03:58:44 MST 2019]{%
What was originally observed by people
(which takes place in history
long before we observed the different wavelength regimes and
started to understand the physical mechanisms giving rise to them).
}

\mynote{Sun Nov 18 18:17:21 MST 2018}{
    Do charged particles travel
    along magnetic field lines via the Lorentz force?
    Is particle beam composed of both electrons and ions, or just electrons?
    Are ions only visible in gamma rays? Do they hit the chromosphere?
}

\paragraph{Rocket analogy}
The episodic, rapid release of magnetic energy in the solar atmosphere
associated with flares, CMEs, and SEP events has the potential to wipe us out.
Thus the field of space weather emerged and is devoted to understanding and
preparing for these events. Much like a rocket booster, this outward burst of
energy needs something to push against, and the same acceleration occurs in the
opposite direction: down toward the lower layers of the atmosphere. Most
research has focused on the rocket: the energy and material released outward
into space. However, it is impossible to understand how the rocket works
without studying what it left behind, how it got there, and how the energy may
have been converted through the various stages to its final form.



\subsection{Flare emission and evolution}
\mynote[Mon Feb 25 03:59:14 MST 2019]{%
Physical processes.
}

Need to know general evolution of a flare, particularly the role of the
chromosphere as we currently understand it.
This is necessary in order
to interpret the results from before, during, and after a flare;
provides some context for what is physically happening.

An understanding of the physical processes that take place
during solar flares
\mynote[18 October 2018]{
    (when they happen, where they happen, the various structures involved,
    static physical conditions such as temperature, density, and composition,
    how much and why those values then change, etc.)}
is crucial if one desires to characterize the role played by the
relatively cool, thin atmospheric layer known as the chromosphere.
These stages, as described by the standard model,
provide vital context for the questions proposed, and then
investigated for this dissertation,
particularly the temporal behavior
of the chromospheric 3-minute oscillations before, during, and after flares.

%Figure~\ref{fig:flare_emission} shows the typical pattern of several bandpasses throughout the duration of a flare.
The development and evolution
of the various dynamical phenomena that ultimately
%lead to the burst of emission known as a solar flare
result in a solar flare
takes place throughout the solar atmosphere
in three primary stages:
precursor, impulsive, and gradual.
Each of these are characterized by specific observations that are interpreted
in the context of energy storage, release, and dissipation.
The official time at which the event transitions from one stage to the next
is determined by the SXR emission from the \textit{GOES} satellite.
GOES has two photometers that observe wavelength ranges
1-8\AA{} (1.5-10 keV) and 0.5-4\AA{}.

\mynote[18 October 2018]{
    Energy range for 0.5-4\AA{}? Need source!}

%(similar to the X-ray curve in Figure~\ref{fig:flare_emission})
These points in time are
characterized by the initial increase, peak, and final return
to background levels of the observed SXR flux.
The times at which these features are observed in the
lightcurves of other wavelength regimes in the EM spectrum
tend to deviate from the \textit{GOES} times
since different processes in the course of flare development
produce different types of emission during different stages.

Both the SXR and H$\alpha$ curves are observed to peak twice.
a sharp increase at the onset (the impulsive phase),
followed by a slower, smoother climb and subsequent decay (the gradual phase).
Each phase is characterized by specific observations that are interpreted in
the context of energy storage, release, conversion, and dissipation. These
stages are important in understanding the role of the chromosphere, and provide
the basis for the proposed investigation into the temporal localization of the
chromospheric 3-minute oscillations.

Our current understanding of flares involves their development and evolution
throughout the entire solar atmosphere. This is broken down into three main
phases: the precursor, impulsive, and decay phases. Each of these are
characterized by specific observations that are interpreted in the context of
energy storage, release, and dissipation. These stages are important in
understanding the role of the chromosphere, and will provide the basis for the
proposed investigation into the temporal of the excitation of the chromospheric
3-minute oscillations.

\textbf{Pre-flare: the zero-th stage}\quad
According to the standard flare model \mynote{(source!)},
the phenomenon we observe as a flare is the result of the rapid
release of magnetic energy after
days of slowing building to critical levels in the corona.

\cite{Woods2017} investigated the pre-flare stages of
the March 29 flare in 2014
with IRIS and \textit{Hinode}/EIS.
\mynote[18 October 2018]{\ldots And?}

\mynote[18 October 2018]{
    The following cute story was copied from Evernote, though
    probably came from proposal originally:}

Our journey begins during the ``pre-flare'' stage, during which magnetic energy
builds to critical levels.
This takes place over timescales from several days up to a few weeks.
\cite{Woods2017}
conducted a study to constrain the trigger mechanism
of flares, for which there are several possibilities. They found plasma flows
with velocities on the order of several hundred km s$^{-1}$.
This is a key time period for space weather prediction.


\mynote[Thu Oct 18 14:58:21 EDT 2018]{
    Testing \texttt{:r! date} to insert current date. Works!
    Although I'd prefer a different format\ldots}


\textbf{Precursor: the first stage}\quad
There are several possible mechanisms that could be responsible for triggering
the release of the stored magnetic energy (see Woods et al. (2017) for an
example).

The precursor phase, sometimes known as the pre-eruptive stage,
begins when the release of magnetic energy is triggered.
This stage typically lasts $\sim$3 minutes.
The eruption starts slowly, as
the energy leaks slowly and the
highly sheared magnetic field loses
stability and begins to rearrange itself.
The magnetic
structure surrounding prominence evolves through series of force-free
equilibria until it loses equilibrium and goes unstable.
The plasma around the magnetic reconnection site heats up,
and is observed as a
slow increase in soft X-ray (SXR) and UV emission from active regions.

A sudden increases in speed occurs,
most likely because the onset of magnetic
reconnection releases field lines that were holding the prominence down,
and cut it loose, hence the
``activation of the prominence'' (T-H and Emslie book).


This onset is important (e.g. SXR and UV before ``cataclysmic onset of impulsive
phase''). The magnetic field becomes unstable, starts to readjust. The emergence
of new flux is now a thing. Prominence activation (Martin and Ramsey 1972) and
heating of plasma (Poland et al. 1982; Hernandez et al. 1986; Klein et al.
1987; Machado et al. 1988a\ldots from T-H \& Emslie chapter 6?).

\textbf{Impulsive: the second stage}\quad

Once the magnetic configuration loses stability, an eruption occurs and stage
two, the impulsive stage, begins. This stage only lasts for a minute or less.
Magnetic reconnection begins, whose rapid onset is thought to be the cause of
the sudden drop in evolutionary timescales from the precursor to the impulsive
stage. There is a sudden release of energy that is then converted into various
forms, including kinetic energy of accelerating particles, thermal energy that
heats the plasma, bulk acceleration of fluid, and enhanced radiation fields.
Intense radio, HXR, and gamma rays are emitted in the form of intense, rapidly
fluctuating bursts. The jump in emission after it increased slowly during the
precursor phase is thought to be caused by the rapid onset of MR. This process
releases the prominence previously held down by the magnetic field. The
prominence is the matter that is ejected outward in the form of a CME.
\cite{Priest2017}
distinguish two phases of magnetic reconnection. The first
takes place during the impulsive phase, along the loop arcade, and is called
``zipper reconnection''.

Once the magnetic configuration loses stability, an eruption occurs
and stage two, the \textbf{impulsive} stage, begins. The onset of
magnetic reconnection releases field lines previously holding the
prominence down \citep{Priest2017}, and energy starts to be released
at a much faster rate.
Some of this energy is converted to kinetic energy of
non-thermal particles, which are then accelerated down until they hit the
chromosphere and inject the plasma with an impulsive burst of high energy
(E $\gtrsim$ 10 keV).
Energy is then lost to radiation in the form of
intense, rapidly fluctuating bursts of radio, hard X-ray (HXR),
and gamma rays.
The HXR Bremsstrahlung radiation at the loop footpoints is indicative
of energization of non-thermal particles, and this is
where the initial heating of the chromospheric plasma takes place
\citep{Hoyng1981, Fletcher2013b}.

This entire process takes less than a minute.
Many of the processes that take place during the impulsive phase
are inferred from the response of the surrounding plasma,
but cannot be observed directly due to the extremely short timescales
over which they occur.

\myheading{NT particle acceleration}
Magnetic energy released via magnetic reconnection is converted to
kinetic energy of charged particles that are accelerated downward
with energies exceeding 1 MeV.

After being injected into the denser plasma of the chromosphere,
the interaction between NT electrons and the ambient particles causes
the emission of Bremsstrahlung HXR radiation.
This emission appears as ribbon-like features along the footpoints
of the reconnected loops as they contract down into the lower atmosphere.

Chromospheric evaporation also occurs during this time.
After the plasma is heated, some of the excess energy is shed by
radiation or thermal conduction, but if these mechanisms are not enough
to return the plasma to thermal equilibrium, the high gas pressure causes
the plasma to rise up into the post-flare loops.
Explosive evaporation occurs when the upward force is accompanied by
an equal and opposite force in the opposite direction.
This is revealed by the presence of redshifts in emission lines in
addition to blueshifts. The downward movement of plasma is then
referred to as chromospheric ``condensation''.
Velocities of plasma movement via explosive evaporation can be
$\gtrsim$100 km s$^{-1}$, and $F_{\ell} > 10^{10}$ erg s$^{-1}$ cm$^{-2}$.

Gentle evaporation refers to
when the continuous, lower energy input from thermal conduction
pushes plasma up the loops (evaporation), but not down (no condensation),
so only blueshifts are observed.
This process continues as long as a temperature gradient exists.
Velocities of plasma movement via gentle evaporation are
$\gtrsim$10 km s$^{-1}$, and $F_{\ell} < 10^{10}$ erg s$^{-1}$ cm$^{-2}$.

Explosive can turn into gentle if conductive flux out of explosively heated
plasma becomes comparable to energy flux in the electron beam:
\[
    F_{cond} \approx F_{NTe^{-}s}; \quad T \gtrsim 1 - \gtrsim 10s
\]



\textbf{Gradual: the third and final stage}\quad
Once all the magnetic energy has been released and dissipated,
non-thermal particles are no longer being accelerated downward,
to drive chromospheric evaporation, and
the gradual stage has begun.
There remains a strong temperature gradient between the hot flare loops and
the surrounding, cooler chromospheric plasma.
Some of the energy that
was not lost to radiation continues to heat the plasma via thermal
conduction, which continues to drive evaporation of plasma upward
through post-flare loops \citep{Battaglia2015}, though more slowly
than during the impulsive phase. This final phase is characterized by
a gradual buildup and decay of SXR emission from evaporation of hot
post-flare loops. The loops continue to evolve by cooling and
draining, and eventually retreat back down and become visible in
H$\alpha$ \citep{Hudson2007}. This emission reveals the morphology of
the so-called ``flare ribbons''. These are the footpoints of newly
connected field lines, and are one of the primary observable features
of flares. This stage typically lasts $\sim$30 minutes.

\subsection{Flare classes}

Flare are classified according to their peak
soft x-ray flux (1-8\AA{}), as
measured at Earth from the \textit{GOES} satellite.
The classes are as follows:
\begin{description}
    \item [A] 10$^{−4}$ < F < 10$^{−3}$
    \item [B] 10$^{−3}$ < F < 10$^{−2}$
    \item [C] 10$^{−2}$ < F < 10$^{−1}$
    \item [M] 10$^{−1}$ < F < 1
    \item [X] 1 < F < 10
\end{description}
where F = flux (erg s$^{-1}$ cm$^{−2}$).
These are further subdivided from 1 through 9.
This is a linear relationship so, for example, an X2 flare is twice as powerful
as an X1 flare.

Typical flare temperatures are 10$^{7}$ K.
Emission lines at these temperatures are blueshifted, while those at
chromospheric or TR temperatures are redshifted
\citep{Brosius2016}.
Typical energies are around 10$^{27}−10^{32}$ erg.
The energy for the largest flares (i.e. of class > X10) has
been capped at a few $\times 10^{32}$ erg.



%\subsection{Quasi-periodic pulsations (QPPs)}


\begin{comment}

\mynote[18 October 2018]{
    So I've been reluctant to discuss the following issue because I don't
    actually know much about it, just that a few authors have quoted these words:
    ``There have been few studies of QPPs in thermal emission.''
    However, just found a paper from {} that explains this, which is helpful
    not only because it provides me with an supplementary description,
    but it's another source, so I'm not just quoting Milligan's paper over
    and over.
}

\mynote[18 October 2018]{
    Why do we care about QPPs? This needs to be clear to the
    reader/listener right away.
}

Thermal emission from the lower atmosphere during flares
provides a potentially useful
way to probe the chromospheric dynamics and extract
information about the transportation and conversion of energy in this
region.

\mynote{Brief description of what QPPs are}

Embedded within the global shape of flare emission
are temporal fluctuations known as quasi-periodic pulsations (QPPs),
with periods ranging from $\sim$1 second to several minutes.
They have been observed during all three flare phases
in all wavelength bands.
(See \cite{VanDoorsselaere2016} for a recent review.)


Since they are clearly connected to the flare process,
QPPs are considered to be an intrinsic
property of flares, thereby providing a direct probe into the
reconnection site and the affected regions beyond.
QPPs are ``directly linked to the properties of the flare reocnnection
region and flare acceleration sites'' \citep{Inglis2015},
so a lot can be learned about flares by studying these.


\mynote[18 February 2019]{Copied from E-note. I think this was in my proposal,
but probably took it out.}
There is no strict definition for this kind of pattern, but it is generally
described, as quoted from \cite{Inglis2015}, as
``variations in the flux from a flare\ldots as a function of time,
which appear to include periodic components with characteristic timescales
ranging from one second up to several minutes.''


Constraining the physical mechanism(s) responsible for the
generation of QPPs is an ongoing investigation.
There are two leading theories that describe them as signatures of one of two
things:
\begin{enumerate}
    \item the energy release process (MR),
        i.e. the rate at which magnetic field lines are reconnected
        and accelerate non-thermal particles.
        This is often referred to as a ``load/unload'' mechanism.
    \item MHD oscillations induced in the ambient plasma by
        \begin{itemize}
            \item the magnetic reconnection itself, which would give the resulting
                QPPs the same ``built-in'' periodicity as the reconnection rate.
            \item the same trigger responsible for inducing magnetic reconnection
                in the first place \citep{Nakariakov2009}.
        \end{itemize}
        Theoretically, MHD oscillations would be induced in post-flare loops,
        due to the nature of these types of oscillations in the vicinity of a
        magnetic field.
\end{enumerate}

Each of the above mechanisms can produce similar observational signatures,
adding to the difficulty in identifying which is the cause, if not both
\citep{Brosius2016}.


Their small periodicities can also make them difficult to extract from the main
lightcurve \citep{VanDoorsselaere2016}.
\mynote[Mon Feb 18 03:15-ish MST 2019]{%
Modulation depth? I think that has more to do with the amplitude than
the period. Maybe it's because flare timescales are close to QPP periods,
so everything gets mushed together and it's more difficult to extract any one
individual signal.}
Sometimes these periodicies can be on the order of a few seconds, making
them difficult to probe with current available instrumental cadence.
\mynote[Mon Feb 18 03:37:31 MST 2019]{Oh.}


% Observations of QPPs in thermal emission during flares
Currently, there are few studies of QPPs in thermal emission.
Non-thermal emission in HXR and microwave bands has a higher modulation depth,
particularly during the impulsive phase of flares, when the global emission level
increases substantially
\citep{Hayes2016}.


\citep{Brosius2016} studied emission from the \ion{C}{1} line, and found
a period of $\sim$170 seconds. They
attributed this periodicity to the rate of injection of
non-thermal electrons.
Further investigation of QPPs in thermal emission was carried out by
\cite{Milligan2017}, who found enhancements of the 3-minute oscillations
in Lyman continuum and Lyman-$\alpha$ line emission.
However, no signature of a 3-minute oscillation was found in
the X-ray data generated by the flare, indicating
that the chromospheric response was not a
reflection of the rate of energy input.
These findings supported the idea that the chromosphere naturally
responds at its cutoff frequency, regardless of the periodicity of the
energy injection rate.
This result did not support the conclusion from
\cite{Sych2009}, who
proposed that the 3-minute waves leaked from sunspots into the upper
atmosphere by propagating along magnetic field lines, and triggered
the magnetic reconnection and subsequent energy release and particle
acceleration.


To investigate the transfer of energy during flares, \cite{Monsue2016}
conducted a pilot study of the spatial and temporal flare response of acoustic
oscillations in H$\alpha$ emission.
They were able to preserve both the two-dimensional spatial information
and the temporal information in the original data
with the power and frequency of the output of the Fourier transforms
by incorporating a technique called ``frequency-filtered amplitude movies''
from \cite{Jackiewicz2013}.
They reported a suppression of frequencies
between 1 and 8 mHz during the main phase, and an enhancement of frequencies
between 1 and 2 mHz before and after the flare. They suggested that the
suppression of lower frequencies could be evidence of the conversion of energy
from acoustic to thermal, and that the pre-flare enhancement may have been
indicative of an instability in the chromosphere.
In the conclusion of this study, they encouraged
the further investigation of
earlier time frames before the flare precursor, along with the
inclusion of additional cases to increase the statistical significance of the
findings.

% Temporal dependence of QPPs (flare phases). Science question - Time!!!
% Move this to methodology?
\cite{Hayes2016} examined oscillations during
both the impulsive and decay phase of an X-class flare, and considered
a combination of possible mechanisms for producing QPPs, where the
impulsive phase involved rapid injections from non-thermal particle
acceleration and subsequent heating of the surrounding plasma, while
thermal signatures persisted throughout the decay phase.
Propagating slow magnetoacoustic waves, such as the chromospheric
3-minute oscillations, show characteristic quasi-periodic patterns.
Several recent papers discuss the relationship between QPPs and
chromospheric oscillations.



% Move this to methodology?
The observations were acquired using
Lyman continuum data from \textit{SDO}/EVE,
Lyman-$\alpha$ line emission from \textit{GOES}/EUVS,
1600\AA{} and 1700\AA{} continuum from \textit{SDO}/AIA, and
HXR emission from \textit{RHESSI}.
EVE observes the sun as a star, with disk-integrated data.
%Figure~\ref{fig:Milligan2017} shows their results from the analysis on AIA data.

\end{comment}

% ---}


% ---}
