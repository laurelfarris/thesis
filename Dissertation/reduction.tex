
\clearpage


\section{Data reduction}

\subsection{aia\_prep.pro}

The standard IDL routine \textit{aia\_prep.pro} was applied to all data.
This routine applies corrections such as
alignment of the solar axes with the horizontal/vertical image axes,
(which removes what is referred to as the ``roll angle''),
and co-alignment between all AIA channels.
When applied (rather counter-intuitively)
to HMI data, this routine rescales it to the same spatial resolution as AIA.
It also rotates HMI data 180 degrees, since the initial orientation
of the images is upside down.




\subsection{Alignment}

Due to the global rotation of the sun, as well as smaller shifts caused
by instrumental effects, there are shifts in x and y between
images in a time series data cube.

To correct this,
the subset of data was extracted and aligned by cross correlation
\citep{McAteer2003, McAteer2004}.
The associated routines provided sub-pixel accuracies.
This allowed a Fourier transform to be applied to a single pixel
at $(x_{i}, y_{i})$, which was how we obtained power maps for analysis.



\subsection{Saturation}


AIA data saturated during the X-class flare.
