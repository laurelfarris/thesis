%\section{%
%Enhanced chromospheric 3-minute oscillatory power associated with the 2011-February-15 X2.2 flare}
%% (title of Article1)



%\section{Enhanced chromospheric
3-minute oscillatory power associated with
the 2011-February-15 X2.2 flare
}
% Input command doesn't appear to work inside \section...


{\fontsize{16}{14}\selectfont\centering\bfseries\scshape
    Enhanced chromospheric
3-minute oscillatory power associated with
the 2011-February-15 X2.2 flare
}


\section{Introduction}
%&<tex>
Most of the radiative energy associated with solar flares
is emitted from the chromosphere
in the form of optical and UV emission, but the mechanism of energy
transport from the magnetic reconnection site to the chromosphere
and subsequent conversion to other forms remains unclear.
The chromosphere has been observed
to oscillate in response to an injection of energy,
suggesting that the nature of such oscillations may reveal
something about the nature of energy deposition and conversion
associated with flares.
In this paper, we aim to characterize the oscillatory response of
the chromosphere before, during, and after an X-class flare
with the goal of
further investigating the ``flaring chromosphere'' and
helping to constrain the origin of the persistent 3-minute oscillations
in the chromosphere.

% QPPs
Embedded within the large-scale variations
in a typical flare lightcurve
are temporal fluctuations known as quasi-periodic pulsations (QPPs).
Typical oscillation periods of QPPs
range from $\sim$1 second to several minutes,
and have been observed throughout the duration of solar flares
in all wavelength bands.
QPPs are considered to be an intrinsic
property of flares, thereby providing an observable
probe into the reconnection site and surrounding plasma
\citep{Inglis2015}.
While the specific mechanism that generates QPPs remains uncertain,
there are two prevailing theories for the mechanism
that generates QPPs.
The first theory posits that magnetic field lines reconnect periodically and
QPPs reflect the rate of energy deposition via
non-thermal particles accelerated each time MR occurs.
The second theory explains QPPs as a more indirect signature of
magnetic reconnection,
wherein MHD waves are induced in the plasma
in the immediate vicinity of the reconnection site,
either by the same mechanism that
triggered the initial onset of magnetic reconnection,
or by the reconnection process itself after it had begun
\citep{Nakariakov2009}.
The QPPs would therefore be observational signatures of these MHD waves.
The difficulty in narrowing down the source of QPPs
lies in the similarity in observational signatures
between the two outcomes.
It is also possible that both ocurr simultaneously, or during
different flare phases \citep{Brosius2016}.
The small periodicities can be difficult to extract from the global
lightcurve \citep{VanDoorsselaere2016}.
Thermal emission from the lower atmosphere during flares provides a potentially
useful way to probe the chromospheric dynamics and extract information about
the transportation and conversion of energy in this region.


% 3-minute oscillations in (non-flaring) chromosphere
Observations of non-flaring active regions
in both intensity and velocity
have revealed oscillations in all regions of the chromosphere
with a dominant period around 3 minutes.
They are particularly strong above the umbra of sunspots,
as first discovered by \cite{Beckers1969}, as well as
internetwork regions in the quiet sun \citep{Orrall1966}.
\cite{Reznikova2012} found a concentration of 3-minute power
above sunspot umbra in AIA UV emission.
These observations are interpreted as the upward propagation of
slow magnetoacoustic waves generated below the chromosphere
\citep{Brynildsen2004}.
One prevailing theory for the dominant power at 3 minutes is that
the acoustic cutoff frequency at the base of the chromosphere,
$\nu_{0} \approx 5.6$ mHz
(which corresponds to a period around 3 minutes)
effectively creates a barrier across which waves can travel
only if they propagate with frequency higher than the cutoff.
Another theory attributes these oscillations to
the chromospheric plasma responding to disturbances
at its own natural frequency.
This was predicted and shown numerically by
a series of papers by \cite{Sutmann1995a,Sutmann1995b,Sutmann1998},
and other studies by \cite{Chae2015}
with similar results for both impulsive and continuous stimulation.
\cite{Sych2009} suggested that the leakage of umbral 3-minute oscillations
into the upper atmosphere was the cause of flaring QPPs, supported by
observations of a similar periodicity in the flare emission.


During the past decade, several studies have revealed enhanced oscillations
in the chromosphere associated with an injection of energy.
Using Dopplergrams from MDI on SOHO,
covering several X-class flares,
\cite{Kumar2006}
found enhancements of the 3-minute oscillations in velocity
that preceded the \textit{GOES} peak time of the flares.
These enhancements were locally concentrated around regions that
produced hard X-ray emission, indicating
that the enhancement was caused by energetic non-thermal particles.
\cite{Brosius2015} studied UV stare spectra of an M-class flare
in \ion{Si}{4}, \ion{C}{1}, and \ion{O}{4} lines,
and reported four complete intensity fluctuations with periods
around 171 seconds.
Their results showing periodic brightenings supported the
model of non-thermal particle beams injecting the chromosphere
with energy.
\cite{Kwak2016} observed
the response of the chromosphere to a downflow event
using high-resolution spectra from the
\textit{Interface Region Imaging Spectrograph}
(\textit{IRIS}; \cite{DePontieu2014}).

\cite{Monsue2016} observed H$\alpha$ emission from the GONG network
and preserved both temporal and spatial information
using a technique devised by \cite{Jackiewicz2013}.
They initially observed an enhancement of all frequencies between
1 and 8 mHz during the flare from the entire AR,
but upon further investigation of subregions within the active regions,
they revealed a suppression of oscillatory power between 1 and 8 mHz
during the main phases of an M- and X-class flare,
and enhancement was only observed
at lower frequencies (1-2 mHz) before and after the flare.
They interpreted some of the changes as conversion to
thermal energy in the chromosphere.
They suggested that the enhancement at low frequencies
prior to the precursor pahse
may be attributed to the presence of an instability in the chromosphere
that could potentially precede strong flares.

\cite{Milligan2017}
observed an enhancement of oscillations during the main phase of an
X-class flare at frequencies between 2 and 20 mHz
(500 and 50 seconds, respectively) in thermal emission.
The greatest increase in power for UV emission
occurred around the 120 second period
during the rise phase of the flare, which coincided with the
timescale of the peak power in the RHESSI X-ray spectrum.
The power at 180 seconds, while not as high, started to increase earlier
and the enhancement lasted until around 02:00, several minutes after
the flare peak. The X-ray emission did not show enhancement at this
period at all.
This supports the prediction that the chromosphere naturally responds
to an impulsive disturbance at the acoustic cutoff frequency.

Though the results from \cite{Milligan2017} revealed an
enhancement in oscillatory power from thermal emission,
they did not reveal where this enhancement occurred relative to the
active region where the flare took place.
One of the goals of the present study is to expand previous results to
include spatially resolved distribution of the 3-minute power.
The initial location of the 3-minute power enhancement
may help probe the nature of the energy deposition, which can be
either injection by non-thermal particles beams, or
thermal conduction in some cases.
For example,
\cite{Awasthi2018} found two distinct pre-flare phases,
beginning with non-thermal particles and evolving into a
thermal conduction front.
\cite{Fletcher2013b} studied both the thermal and non-thermal response
of the chromosphere during the early stages of an M-class flare,
and found the main flux to originate from a different location from
the initial brightenings.

The motivation behind including pre-flare data in this project
is twofold.
First, it will provide a comparison between the flaring and non-flaring
chromosphere.
This is necessary to interpret whether
the enhanced power reflects the natural response of the chromosphere
at the acoustic cutoff frequency,
or if it merely reflects the
exterior properties of the energy source.
\mynote{Need to discuss theories of response at natural frequency
vs. rate of energy injection.}
Second, if the chromosphere exhibits pre-flare signals,
this would contribute to the field of space weather prediction.

In the absence of flares,
the 3-minute oscillations cease to dominate below the chromosphere.
However, if the chromospheric layers are greatly disturbed, they may
push into the photospheric layers below, producing oscillations that are
not normally present.
See \cite{Simoes2018} and \cite{Tripathy2018}.

% Hypothesis
The rise in thermal emission from the chromosphere associated with flares
is caused by the rapid energy injection and subsequent heating of the plasma,
which then radiates some of this energy in the form of thermal UV emission.
If the plasma were to oscillate in response to this energy at the site
of energy injection, it is expected that the locations would be the same
at first.

Here we present the spatial and temporal evolution of 3-minute power in the
chromosphere during the
\textit{GOES} X-class flare that occurred on 15 February 2011.
The Atmospheric Imaging Assembly (AIA; \cite{Lemen2012}) on board the
\textit{Solar Dynamics Observatory} (SDO; \cite{Pesnell2012})
provides images
with a spatial size scale of 0.6" per pixel and 24-second cadence
in thermal UV emssion from
two channels that sample the lower atmosphere.
These data allow the computation of spatially resolved power maps centered
on the frequency of interest.
The inclusion of data before and after the flare allows the comparison
of the flaring and non-flaring chromosphere
to distinguish whether the
plasma is oscillating at the natural frequency of the chromosphere
or responding to an impulsive injection of energy.
The flare, data, and methodology are described in \S\ref{data}.
Results are presented and interpreted in \S\ref{results}.
Discussion of results, including potential relationship between
oscillations and magnetic field is in \S\ref{discussion}.
Conclusions and proposed future work are discussed in
\S\ref{conclusions}.



\clearpage
\section{Observations and data reduction}
%\subsection{The SOL2011-02-15T01:56 flare}
The 2011 February 15 X2.2 flare
occurred in NOAA active region (AR) 11158
close to disk center
during solar cycle 24 (SOL2011-02-15T01:56).
The AR was composed of a quadrupole:
two sunspot pairs (four sunspots total).
The X-flare occurred in a delta-spot composed of
the leading spot of the southern pair and
the trailing spot of the northern pair.
It started at 01:44UT, peaked at 01:56UT, and ended at 02:06UT,
as determined by the soft X-ray flux from the
\textit{Geostationary Operational Environmental Satellite}
(\textit{GOES}-15; \cite{Viereck2007}).
The impulsive phase lasted about 10 minutes.
Data covering 5 hours centered on this flare were used for the analysis.
This data includes a C-class flare that occurred between 00:30 and 00:45 UT
on 15 February 2011.


\textit{SDO}/AIA
obtains full disk images \correction{throughout}
the solar atmosphere, using narrow band filters centered on
10 different wavelengths, two of which provide measurements of
thermal UV emission from the \correction{chromosphere}.
The 1700\AA{} channel \correction{mostly contains} continuum emission from the
temperature minimum, and
the 1600\AA{} channel \correction{covers}
both continuum emission and the \ion{C}{4} spectral line in the upper
photosphere and transition region.
Both channels have a cadence of
24 seconds and \correction{spatial size scale} of 0.6 arcseconds per pixel.

Data from
the Helioseismic and Magnetic Imager (HMI; \cite{Scherrer2012}),
also on board \textit{SDO}, is used to \correction{study} potential
correlations between magnetic field strength and oscillatory
behavior in the chromosphere.
HMI obtains full disk data in the form of
line-of-sight magnetograms, vector magnetograms,
Doppler velocity, and continuum intensity,
measured at the \ion{Fe}{1} absorption line at 6173\AA{}
with a passband width of 0.076\AA{}.
Each \correction{data product} has a cadence of 45 seconds (with the exception
of the vector magnetograms, at 135 seconds),
and \correction{spatial size scale} of 0.5 arcseconds per pixel
\citep{Schou2012}.

The standard data reduction routine
\textit{aia\_prep.pro} from solarsoft was
\correction{applied to all data}.

\begin{figure*}[htb!]\centering
    \includegraphics[draft=false,width=1.0\textwidth]{lc.pdf}\\
    \includegraphics[draft=false,width=1.0\textwidth]{lc_goes_20190124.pdf}
    \caption{%
        Top: Light curves of the
        UV continuum emission from AIA 1600\AA{} (blue curve) and
        AIA 1700\AA{} (red curve),
        integrated over the flare region in AR 11158.
        %Bottom: Light curves from top panel normalized between 0.0 and 1.0,
        %overlaid with flux from the \textit{GOES} 1-8\AA{} channel.
        Bottom: Light curves from \textit{GOES-15}
        channels 1-8\AA{} (black curve) and 0.5-4\AA{} (pink curve),
        scaled as log(flux) to enable visibility of the increases
        during smaller events before and
        after the main X-flare.
        \label{lc}}
\end{figure*}

%\begin{figure*}[htb!]\centering
%    \includegraphics[draft=false,width=1.0\textwidth]{lc_log_20181128.pdf}\\
%    \includegraphics[draft=false,width=1.0\textwidth]{lc_goes.pdf}
%    \caption{%
%        Top:
%        Same as top panel of Figure~\ref{lc}, but with AIA emission in
%        log space to obtain a better comparison to the SXR emission from
%        \textit{GOES}.
%        \label{lc_log}}
%\end{figure*}

Figure~\ref{lc} shows light curves for the full 5-hour time series
from 00:00 to 04:59 on 2011-February-15.
The top panel shows both AIA channels.
The bottom panel shows both SXR channels from \textit{GOES-15} at
1-8\AA{} (black curve) and 0.5-4\AA{} (pink curve).
%overlaid and similarly scaled for comparison.
A small C-flare occurred before the X-flare between 00:30 and 00:45 UT, and
two small events occurred after the X-flare,
between 03:00 and 03:15, and between 04:25 and 04:45.

\begin{figure*}[htb!]\centering
    \includegraphics[draft=false,width=1.0\textwidth]{images.pdf}
    \caption{
        Images of active region 11158 in AIA 1600\AA{} (left panels),
        AIA 1700\AA{} (middle panels), and HMI LOS magnetogram (right panels),
        scaled to $\pm300$ Gauss.
        The top panels show the full disk,
        and the bottom panels show the region used for analysis in this study.
        \label{images}}
\end{figure*}


\begin{figure}[htb!]\centering
    \includegraphics[draft=false,width=0.5\textwidth]{hmi_label_sunspots_20190124.pdf}
    \caption{
        HMI LOS magnetogram.
        White and black contours outline positive (+300 Gauss)
        and negative (-300 Gauss) polarities, respectively.
        The two sunspots in the northern pair are labeled AR\_1a (leading sunspot)
        and AR\_1b (trailing sunspot).
        The two sunspots in the southern pair are labeled AR\_2a (leading sunspot)
        and AR\_2b (trailing sunspot).
        \label{label}}
\end{figure}

Pre-flare images of the full disk are shown in Figure~\ref{images},
along with a 300x198 arcsecond subset of the data centered on AR 11158.
\correction{This subset was extracted and aligned by cross correlation}
\citep{McAteer2003,McAteer2004}.
Images were scaled to improve contrast using the
\textit{aia\_intscale.pro} routine from \textit{sswidl}.
The magnetic configuration of the quadrupole is clear in the HMI magnetograms.
The northern pair will be designated as AR\_1
and the southern pair will be designated as AR\_2.
Sunspots in the northern pair will be designated as AR\_1a (positive polarity)
and AR\_1b (negative polarity).
Sunspots in the southern pair will be designated as AR\_2a (positive polarity)
and AR\_2b (negative polarity).

Both AIA channels saturated ($\geq\!15000$ counts) in the center
during the peak of the X-class flare, and a few pixels also saturated
during the smaller events before and after.
Affected pixels were \correction{all} contained within the
300x198 arcsecond \correction{subset of data}
throughout the duration of the time series.
Four images from the 1700\AA{} channel on AIA were missing, between
the images with start times at
00:59:53.12,
01:59:29.12,
02:59:05.12, and
03:58:41.12, and the following images, each with start times
48 seconds after the previous image.
Since the gaps in data were separated by an hour,
it was reasonable to approximate \correction{missing images}
by averaging the two adjacent images.


%\mynote{Flare on 2011 February 15 took place ``close to disk center''.
%How close is ``close''? What are the difficulties in analyzing target close
%to the solar limb? Opacity issues?}



\begin{figure}[htb!]\centering
    \includegraphics[draft=false,width=0.5\textwidth]{hmi_label_sunspots_20190124.pdf}
    \caption{
        HMI LOS magnetogram.
        White and black contours outline positive (+300 Gauss)
        and negative (-300 Gauss) polarities, respectively.
        The two sunspots in the northern pair are labeled AR\_1a (leading sunspot)
        and AR\_1b (trailing sunspot).
        The two sunspots in the southern pair are labeled AR\_2a (leading sunspot)
        and AR\_2b (trailing sunspot).
        \label{hmi}}
\end{figure}


\clearpage
\begin{figure*}[htb!]\centering
    \includegraphics[draft=false,width=1.0\textwidth]{lc.pdf}\\
    \includegraphics[draft=false,width=1.0\textwidth]{lc_log_20181128.pdf}\\
    \includegraphics[draft=false,width=1.0\textwidth]{lc_goes_20190124.pdf}
    \caption{%
        Top: Light curves of the
        UV continuum emission from AIA 1600\AA{} (blue curve) and
        AIA 1700\AA{} (red curve),
        integrated over the flare region in AR 11158.
        Middle: Same as top, but scaled as log(flux).
        Bottom: Light curves from \textit{GOES-15}
        channels 1-8\AA{} (black curve) and 0.5-4\AA{} (pink curve),
        scaled as log(flux) to enable visibility of the increases
        during smaller events before and
        after the main X-flare.
        \label{lc}}
\end{figure*}


\clearpage
\begin{figure}[htb!]\centering
    \includegraphics[draft=false, width=\textwidth]{detrendedLC_20181202.pdf}
    \caption{%
        Light curves of AIA 1600\AA{} and AIA 1700\AA{}, overlaid with
        same light curves after applying a FFT filter with a period
        cutoff of 400 seconds.
    \label{detrended}}
\end{figure}

\mynote[Sun Dec  2 12:37:06 MST 2018]{
    Plot of detrended data goes here.
    Threshold = lower limit on frequency
    to get rid of global variations (e.g. the flare) and keep
    high-frequency variations, like QPPs.}

\begin{figure}[htb!]\centering
    \includegraphics[draft=false, width=\textwidth]{detrended_20181201.pdf}
    \caption{%
        Power spectrum from
        AIA 1600\AA{} flux between 1:30 and 2:30 UT.
        The orange curve is the power from the raw data, and the blue
        curve is the power from flux after applying an FFT high-pass filter
        to remove variations with periods longer than 400 seconds,
        such as the global flare curve, which dominates the spectrum at
        low frequencies.
    \label{detrended2}}
\end{figure}




Table~\ref{tab:bda} shows the time ranges covered by each time period
of interest, and the processes that ocurred during that time.

% Table - BDA
\begin{deluxetable}{c c c c}
    \tablewidth{\textwidth} % default = \pagewidth, natural width = 0pt
    \tablecaption{
        Pre-flare, X-flare, and post-flare things that happened
        \label{tab:bda}}
    \tablehead{
        \colhead{Phase} &
        \colhead{Indices} &
        \colhead{Time range (UT)} &
        \colhead{What's happening}
    }
    \startdata
        Before  & 0:63 & 00:00--00:25 & blip at very beginning\\
        \nodata & 16:79 & 00:07--00:32 & Quiet\\
        \nodata & 27:90 & 00:11--00:36 & pre-flare through C-flare peak\\
        \nodata & 80:143 & 00:32--00:57 & Centered on C-flare\\
        \nodata & 90:153 & 00:36--01:01 & C-flare peak (close to previous t-seg; start to peak $\sim$4 min!\\
        \nodata & 147:200 & 00:59--01:24 & Through where em. back to pre-Cflare levels\\
        \nodata & \nodata & \nodata & also centered on brief flux increase at $\sim$1:01--1:05 (2b)\\
        \nodata & 175:238 & 01:10--01:35 & Quiet, cleared of obvious emission increase\\
        \nodata & 197:260 & 01:19--01:44 & BP in (1a), been there since 00:30,\\
        \nodata & \nodata & \nodata & though no noticable increase in flux from this SS\\
        During  & \nodata & 01:45--02:30 & \nodata\\
        After   & \nodata & 02:30--04:59 & \nodata\\
    \enddata
\end{deluxetable}






\clearpage
\section{Analysis}
The technique used to calculate power maps as functions of
space and time is similar to that \correction{employed} by
\cite{Jackiewicz2013} and \correction{further employed} by \cite{Monsue2016}.
The general method is as follows:
For a data set of $N$ images, each power map
$P(x,y,t_{i})$ is generated by
applying a Fourier transform to every pixel at
$(x,y)$ in the temporal direction,
from $t_{i}$ to $t_{i}+T$, where $T$ is the length of the time segment.
The power is averaged over a frequency band $\Delta{\nu}$
of user-defined width, centered on the frequency
of interest. This process is repeated at every timestep, for starting
times from $t_{0}$ to $t_{N-N_{T}}$.

The data set for AR 11158 consisted of $N$ = 749 images (5 hours)
of AIA observations in each channel.
Each time
segment $T$ was set to 64 images ($\sim$25.6 minutes).
The value of $T$ was chosen based on
a balance between sufficient length to obtain frequencies close to
that of the 3-minute period and not so long as to lose information on
timescales over which the 3-minute power was previously observed to change.
Each Fourier transform was applied without detrending the data since
the frequency of interest was well outside the global flare signal.
(As a check on this, a Fourier filter was applied with a cutoff period
above 400 seconds. The power spectra for the periods of interest did not change.)
If a saturated pixel was encountered in any segment $t_{i}$ to $t_{i}+T$,
it was excluded from the power map for that time segment,
and the location ($x, y$) of that pixel was set to zero.

The frequency bandwidth $\Delta\nu$ was set to 1 mHz
\correction{centered on $\nu \sim 5.6$ mHz}.
This is consistent with similar techniques applied in previous studies.
For instance, \cite{Stangalini2011} used a 1-mHz frequency bandpass between 4.8 mHz
(208.3 seconds) and 5.8 mHz (172.4 seconds) when calculating power maps
\correction{around 5.6 mHz} for the chromosphere and photosphere.
\cite{Tripathy2018} also used a band of 1 mHz
over 0.1-mHz steps from 1 to 10.5 mHz.
\cite{Reznikova2012} used a bandpass of only 0.4 mHz\ldots

With these input parameters,
a frequency resolution $\partial\nu$ of $\sim$0.65 mHz
was obtained.
Two frequencies were obtained within $\Delta\nu$ at
5.21 mHz (192.00 seconds) and
5.86 mHz (170.67 seconds).

The average power over $\Delta\nu$ for each unsaturated pixel in
time segment $T$ from
$t_{i}$ to $t_{i}+T$
was taken to be the 3-minute power in each power map.
Since only two frequencies were obtained within $\Delta\nu$,
and were centered around the frequency of interest,
the average was computed without the application of a filter.

Power maps representing the 3-minute power over NOAA AR 11158
in space and time
were obtained at every starting point in the time series (up to $N-T$)
by applying a Fourier transform to the signal from each pixel,
and averaging the power within the
1-mHz frequency bandwidth $\Delta\nu$ centered around 5.6 mHz (3 minutes).



\clearpage
\section{Results and discussion}
%This section first described the general results from the power
maps that remained consistant throughout the time series.
Then \S\ref{time} describes the temporal variation in power:
before, during, and after the flare.

\subsection{Spatial distribution of 3-minute power}

The dark regions in many of the power maps, particularly
in the center of maps that cover time during the X-class flare,
represent locations where saturation occurred, and were set to
zero in order to improve contrast between the remaining pixels.
Bleeding that occurred at the edges of saturated areas could not
be excluded since these values started to approach those held
by pixels that were not saturated.

While the location of power enhancement appears to be correlated with
intensity, not all locations of high intensity are accompanied by
enhanced power.
The 3-minute power appears to be spatially correlated with a few
areas of high intensity in and
directly around the active region, but does not branch out into the quiescent
network/internetwork regions beyond.
This trend is consistent throughout the duration of the flare and post-flare
phases.
A particularly prominent spot emerges in the lower edge of AR\_1a
and persists until the decay phase of the X-class flare.
Power enhancement occurs in relatively \emph{small} regions.
A few isolated regions of enhancement covered an area of
$\sim 6" \times 6"$.
The 3-minute power appears to be suppressed in most of the areas
directly over the umbra.


\subsubsection{Correlation with $B_{LOS}$}

\begin{figure*}[htb!]\centering
    \includegraphics[draft=false,width=0.48\textwidth]{aia1600big_image.pdf}
    \includegraphics[draft=false,width=0.48\textwidth]{aia1700big_image.pdf}\\
    \includegraphics[draft=false,width=0.48\textwidth]{aia1600big_map.pdf}
    \includegraphics[draft=false,width=0.48\textwidth]{aia1700big_map.pdf}
    \caption{%
        Top row:
        Post-flare images, shown as a composite product over
        the time range shown, overlaid with contours showing the approximate
        location of the $B_{LOS}$ at $\pm$300 Gauss.
        Bottom rows: Spatial distribution of 3-minute power, obtained
        by applying a pixel-by-pixel Fourier transform over the images
        included in the top row.
        The left column shows results from AIA 1600\AA{}, and
        the right column shows results from AIA 1700\AA{}.
        White and black contours represent positive and negative polarity,
        respectively.
    \label{contours}}
\end{figure*}

\myfig Figure~\ref{contours} shows post-flare power maps for
AIA 1600\AA{} and AIA 1700\AA{}, overlaid with
HMI $B_{LOS}$ contours at $\pm$300 Gauss
from the center of the time segment used to produce the power map
(around 1:31 UT).
Negative polarities are outlined in black and
positive polarities are outlined in white.
This figure serves as an example of general spatial distribution
of oscillatory power, and
will be used as a reference in this section.

In most power maps, the power enhancement
is located along the boundaries of magnetic field
strength = $\pm$300 Gauss,
which correlates with the outer boundary of the penumbra.
(It should be noted that the alignment procedures may have resulted in a slight
offset between the channels, in addition to any existing LOS affects.)

In addition to being localized to small areas,
the enhanced areas appear to be located over
umbral regions at opposite polarities on the HMI magnetogram in
\myfig Figure~\ref{images}.

\subsection{Temporal evolution of oscillatory power}\label{time}

\subsubsection{Discrete wavelet analysis}
\begin{figure*}[htb!]\centering
    \includegraphics[draft=false,width=0.9\textwidth]{wa.pdf}
    \caption{
        Time-frequency power plots from AIA 1600\AA{} (top panel) and AIA
        1700\AA{} (bottom panel), obtained by applying a Fourier transform to
        integrated emission from NOAA AR 11158 in discrete time increments of
        64 frames ($\sim$25.6 minutes) each. The dashed horizontal line marks the
        central frequency $\nu_{c}$ at $\sim$5.6 mHz, corresponding to a period
        of 3 minutes. The dotted horizontal lines on either side of $\nu_{c}$
        mark the edges of the frequency bandpass $\Delta\nu$ = 1 mHz. The
        vertical lines mark the flare, start, peak, and end times as determined
        by \textit{GOES}. The power is scaled logarithmically and over the same
        range in both channels.
        \label{wa}}
\end{figure*}

The technique described in \S\ref{analysis} was applied
to the integrated flux from AR 11158 at
discrete intervals of $T$ = 64 images with no overlap
(i.e. for start time $t_{0}$, then $t_{1} = t_{0}+T$, etc.).
This provided a ``quick and dirty'' way
to compare the spectral power at a range of frequencies.
This method produces similar results to those
obtained with wavelet analysis, though
at lower resulting frequency and time resolution.
The results are are shown in \myfig Figure~\ref{wa}
for frequencies between 2.5 and 20.0 mHz (400 and 50 seconds, respectively).
The central frequency $\nu_{c} = 5.56$ mHz
and the frequency bandpass $\Delta\nu$ at 5 and 6 mHz
are marked by the horizontal dashed lines.
The power at all frequencies
appears to be enhanced during the X-flare compared to their
non-flaring power before and after.
%This implies that the chromospheric plasma oscillates at a range
%of frequencies in response to energy injection.
During the small events before and after the flare,
the power at lower frequencies is enhanced,
but the power at higher frequencies is suppressed relative to
the same frequencies for adjacent time segments.

At all points in time when power enhancement occurs for any frequency,
there appears to be a correlation with flux increase.

\subsubsection{3-minute power vs. time}

\begin{figure*}[htb!]\centering
    \includegraphics[draft=false,width=1.00\textwidth]{time-3minpower_flux.pdf}
    \includegraphics[draft=false,width=1.00\textwidth]{time-3minpower_maps.pdf}
    \caption{%
        Temporal evolution of the 3-minute power $P(t)$ in
        AIA 1600\AA{} (blue curve) and AIA 1700\AA{} (red curve).
        Top: $P(t)$ obtained by applying a Fourier transform to the
        integrated flux from AR 11158.
        Bottom: $P(t)$ per unsaturated pixel, obtained by summing over power maps.
        Each point in time
        %represents the sum (in x and y) of each power map and
        is plotted as a function of the center
        of the time segment over which the Fourier transform was applied to
        obtain the power map over which the point was summed.
        %The axis labeled $T = 64$ is scaled to show the
        %length of each time segment relative to the full time series.
        The vertical dashed lines mark the \textit{GOES} start, peak, and end
        times of the flare at 01:44, 01:56, and 02:06 UT, respectively.
        \label{power_vs_time}}
\end{figure*}

The evolution of the 3-minute power with time
was calculated from the power maps
by summing over $x$ and $y$ in each map $P(x,y,t_{i})$, and taking the total
to be the 3-minute power of the active region during each time segment.
This is shown in
\myfig Figure~\ref{power_vs_time}.
Each point is plotted as a function of the center of the time segment
over which the Fourier transform was applied to obtain that point.
The axis labelled $T=64$ shows the scale for this length of time.

Power as a function of time obtained from total flux and power maps
is shown to check for possible contradictions between the two.
Integrating flux over the AR
before applying the Fourier transform
has the potential effect of reducing or canceling signal from
pixels whose intensity variations are out of phase.

The 3-minute power for both AIA channels
appears to have its own periodicity with time.
The calculations were repeated for several other frequencies, including
5-minute period and 2-minute period.
The same trends were produced, with the additional result
that the length of ``period'' scaled with the length of the period
the plots were supposed to represent.
It is reasonable to attribute this to computational effects.

\subsubsection{Pre-flare}

\begin{figure*}[htb!]\centering
    \includegraphics[draft=false,width=0.9\textwidth]{aia1600before_20181204.pdf}
    \includegraphics[draft=false,width=0.9\textwidth]{aia1700before_20181204.pdf}
    \caption{%
        Spatial distribution of 3-minute power before the X-class flare,
        in log scale to bridge large contrasts.
        Locations whose time segment included saturated pixels were set to zero.
        \label{before}}
\end{figure*}

\myfig Figure~\ref{before} shows the spatial distribution of 3-minute power for
various time segments of interest from 00:00 UT up to 01:44 UT.
This period includes a C-class flare that occurred an hour before the
X-class flare in AR\_2b.

\subsubsection{During the flare}
\begin{figure*}[htb!]\centering
    \includegraphics[draft=false,width=0.9\textwidth]{aia1600during_20181204.pdf}
    \includegraphics[draft=false,width=0.9\textwidth]{aia1700during_20181204.pdf}
    \caption{%
        Spatial distribution of 3-minute power during the X-class flare.
        \label{during}}
\end{figure*}

\myfig Figure~\ref{during}
shows the spatial distribution of 3-minute power for
various time segments of interest during the X-class flare.
Maps (a)-(c) include the first few frames when emission began
to increase in the impulsive phase before the channels
saturated.
These three maps include pre-flare and precursor emission.
They are as close in time as the instrumental cadence allowed,
and there is a noticeable increase in enhancement in the flare site
over AR\_1b and AR\_2a.

The remainder of the power maps include saturated pixels in the data.
Maps (d)-(f) include emission from the impulsive phase,
and maps (g)-(i) include emission from AIA peak emission and
onward through the decay phase.

\subsubsection{Post-flare}
\begin{figure*}[htb!]\centering
    \includegraphics[draft=false,width=0.9\textwidth]{aia1600after_20181204.pdf}
    \includegraphics[draft=false,width=0.9\textwidth]{aia1700after_20181204.pdf}
    \caption{%
        Spatial distribution of 3-minute power after the X-class flare.
        \label{after}}
\end{figure*}
\myfig Figure~\ref{after}
shows the spatial distribution of 3-minute power for
various time segments of interest during the X-class flare.
There were two small events after the X-class flare.
Emission for the first one came first from

\subsection{Discussion}

Almost all events before and after the main X-class flare
occurred in AR\_2b, including briefly at the beginning of the time series.

\subsubsection{Lifetime and navigation of enhanced regions}
The location of these enhancements does not move across the AR.
Rather, it remains in one place until it fades away.
One of the most prominent locations of enhanced power occurs before
the flare at the bottom of the leading sunspot in the northern pair.
The change in location with time of both flare intensity and 3-minute power
implies that the source of the beam of non-thermal particles changes as well.

% Damping behavior of 3-minute oscillations (maybe)
3-minute oscillations are interpreted as slow, propagating magnetoacoustic waves,
which have a characteristic excitation mechanism and damping rate,
depending on the local plasma conditions where they originate.
The timescales over which the oscillatory power is expected to change
depends on the nature of the oscillations themselves,
or maybe the cooling rate of the plasma.
The expected timescales would depend on the cooling rate of the plasma
(images) and the damping time/mechanisms of the 3-minute oscillations.

\subsubsection{AIA formation heights/temperatures}

The amplitude of the small-scale variations in 3-minute
power is higher for AIA 1700\AA{} almost everywhere with
the exception of the main phase of the X-class flare.

%Since each point in the temporal plots represents
%the behavior of a much longer time segment than the plot seems to imply,
%a running average over the length of time segment $T$
%was computed from
%the data obtained from summing the power maps,
%and is shown in \myfig Figure~\ref{power_vs_time}.
%This greatly smoothed the periodicity in the power,
%but with further loss of temporal resolution.

When both plots are normalized between 0.0 and 1.0, the variation in
power is higher from AIA 1700 almost everywhere except the main phase of the
X-class flare, when the power from 1600 is slightly higher.

The power from AIA 1700 is higher than AIA 1600 by
1000 counts at all points throughout the time series.
Compared to 1700\AA{},
the 1600\AA{} 3-minute power appears to increase more
(relative to its own minimum) and at a faster rate.
The standard deviation for $P(t)$ from integrated flux
for 1600\AA{} is $4.9 \times 10^{4}$, and
for 1700\AA{} is $2.7 \times 10^{4}$.

If the emission from AIA 1600\AA{}
originates from a higher location in the atmosphere
than the 1700\AA{} emission,
a possible explanation for the higher, sharper increase
is that the energy from the non-thermal particle beam
dissipates as it travels through deeper layers of the chromosphere.
Because AIA 1700 originates from a deeper layer, it is probably more dense
as well, which would cause the plasma to cool at a slower rate.
If the 3-minute power originated from the transition region rather than the
upper photosphere, this may coincide with the levitation of hot chromospheric
plasma upward into post-flare loops.
\mynote{Source?}
Although the emission from AIA 1700\AA{} is generally thought to originate
in lower formation heights than emission from 1600\AA{},
the latter spans a broader temperature range, and contains
emission from \ion{C}{4} line.
Determination of the AIA 1600\AA{} formation height is more complicated
during flares because the \ion{C}{4} is more likely to be contributing
to the signal, and both channels may be sampling at deeper layers than
they are thought to during non-flaring times. \mynote{Source.}

The persistence of the 3-minute power toward the end of the gradual phase
in AIA 1700\AA{} is consistent with the results of the wavelet analysis
carried out by \cite{Milligan2017}.

\subsection{Subregions}

\begin{figure*}[htb!]\centering
    \includegraphics[draft=false,width=0.80\textwidth]{preflare.pdf}\\
    \includegraphics[draft=false,width=0.80\textwidth]{postflare.pdf}
    \caption{%
        Power maps from AIA 1600\AA{} before and after the flare
        overlaid with contours from HMI B$_{LOS}$ at $\pm$300 Gauss.
        Boxes outline subregions of interest.
        \label{subregions}}
\end{figure*}


% --> can't use this one because of input files nested in ../Article1/results.tex



\begin{figure*}[htb!]\centering
    \includegraphics[draft=false,width=1.0\textwidth]{aia1600maps_before.pdf}
    \includegraphics[draft=false,width=1.0\textwidth]{aia1700maps_before.pdf}
    \includegraphics[draft=false,width=1.0\textwidth]{lc_before.pdf}
    \caption{%
        Spatial distribution of 3-minute power (arbitrary instrumental units)
        for AIA 1600\AA{} (top) and AIA 1700\AA{} (bottom)
        with the central frequency at 5.6 mHz ($\pm$ 0.5 mHz).
        The $x$ and $y$ dimensions are the same as the images in
        Figure~\ref{images}.
    \label{powermaps}}
\end{figure*}

\begin{figure*}[htb!]\centering
    \includegraphics[draft=false,width=1.0\textwidth]{aia1600maps_during.pdf}
    \includegraphics[draft=false,width=1.0\textwidth]{aia1700maps_during.pdf}
    \includegraphics[draft=false,width=1.0\textwidth]{lc_during.pdf}
    \caption{%
        Mid-flare power maps.
    \label{powermaps_during}}
\end{figure*}

\begin{figure*}[htb!]\centering
    \includegraphics[draft=false,width=1.0\textwidth]{aia1600maps_after1.pdf}
    \includegraphics[draft=false,width=1.0\textwidth]{aia1700maps_after1.pdf}
    \includegraphics[draft=false,width=1.0\textwidth]{lc_after1.pdf}
    \caption{%
        Post-flare power maps.
    \label{powermaps_after1}}
\end{figure*}

\begin{figure*}[htb!]\centering
    \includegraphics[draft=false,width=1.0\textwidth]{aia1600maps_after2.pdf}
    \includegraphics[draft=false,width=1.0\textwidth]{aia1700maps_after2.pdf}
    \includegraphics[draft=false,width=1.0\textwidth]{lc_after2.pdf}
    \caption{%
        Post-flare power maps.
    \label{powermaps_after2}}
\end{figure*}

\begin{figure*}[htb!]\centering
    \includegraphics[draft=false,width=0.48\textwidth]{aia1600big_image.pdf}
    \includegraphics[draft=false,width=0.48\textwidth]{aia1700big_image.pdf}\\
    \includegraphics[draft=false,width=0.48\textwidth]{aia1600big_map.pdf}
    \includegraphics[draft=false,width=0.48\textwidth]{aia1700big_map.pdf}
    \caption{%
        Post-flare power maps overlaid with contours showing the approximate
        location of the $B_{LOS}$ at $\pm$300 Gauss.
        White and black contours represent positive and negative polarity,
        respectively.
    \label{contours}}
\end{figure*}

\begin{figure*}[htb!]\centering
    \includegraphics[draft=false,width=0.9\textwidth]{wa.pdf}
    \caption{
        Time-frequency power plots from AIA 1600\AA{} (top panel) and AIA
        1700\AA{} (bottom panel), obtained by applying a Fourier transform to
        integrated emission from NOAA AR 11158 in discrete time increments of
        64 frames ($\sim$25.6 minutes) each. The dashed horizontal line marks the
        central frequency $\nu_{c}$ at $\sim$5.6 mHz, corresponding to a period
        of 3 minutes. The dotted horizontal lines on either side of $\nu_{c}$
        mark the edges of the frequency bandpass $\Delta\nu$ = 1 mHz. The
        vertical lines mark the flare, start, peak, and end times as determined
        by \textit{GOES}. The power is scaled logarithmically and over the same
        range in both channels.
        Note that the x-axis labels do not necessarily line up with the
        boundaries of the data columns.
        \label{wa}}
\end{figure*}

\begin{figure*}[htb!]\centering
    \includegraphics[draft=false,width=1.00\textwidth]{time-3minpower_flux.pdf}
    \includegraphics[draft=false,width=1.00\textwidth]{time-3minpower_maps.pdf}
    \caption{%
        Temporal evolution of the 3-minute power $P_{3min}(t)$ in
        AIA 1600\AA{} (blue curve) and AIA 1700\AA{} (red curve).
        Top: $P(t)$ obtained by applying a Fourier transform to the
        integrated flux from AR 11158.
        Bottom: $P(t)$ per unsaturated pixel, obtained by summing over power maps.
        Each point in time
        %represents the sum (in x and y) of each power map and
        is plotted as a function of the center
        of the time segment over which the Fourier transform was applied to
        obtain the power map over which the point was summed.
        %The axis labeled $T = 64$ is scaled to show the
        %length of each time segment relative to the full time series.
        The vertical dashed lines mark the \textit{GOES} start, peak, and end
        times of the flare at 01:44, 01:56, and 02:06 UT, respectively.
        \label{power_vs_time}}
\end{figure*}

\clearpage
\begin{figure*}[htb!]\centering
    \includegraphics[draft=false,width=0.80\textwidth]{preflare.pdf}\\
    \includegraphics[draft=false,width=0.80\textwidth]{postflare.pdf}
\end{figure*}




\clearpage
\section{Conclusions}
In this work, we have used spatially resolved data from
SDO/AIA and SDO/HMI
to constrain the location and time that enhancement in the power
of the 3-minute oscillations occurred before, during, and after
an X-class flare.
Results support the theory of energy
injection by acceleration of non-thermal particles, and the response
of the chromosphere to this injection in thermal wavelengths.

The preliminary results of this work support the following conclusions:
\begin{enumerate}
%    \item The increase in 3-minute power just before the flare onset,
%        specifically before the increase in SXR emission, suggests that the
%        chromosphere is responding to a disturbance at its natural frequency.
    \item The enhancement of 3-minute
        power is concentrated in small areas that coincide with locations
        of enhanced flare emission. This supports the theory of
        energy injection by a beam of accelerated non-thermal particles.
        It also shows that the chromospheric plasma does not oscillate
        globally as one body across the active region.
    \item The 3-minute power changes more for
        AIA 1600\AA{} than for 1700\AA{}, which suggests that the 1600\AA{}
        emission originates from higher layers.
%    \item The temporal variation in 3-minute power appears to correlate with
%        the input flux, except after the flare peak where there is a second,
%        smaller increase in power as the flare flux is still steadily decreasing.
%        This may be attributable to two different sources of energy,
%        occurring at different times.
%    \item The 3-minute power appears to have a quasi-periodic variability
%        of its own. This is likely due to the computational effects of
%        shifting the input time series forward 24 seconds for each calculation.
\end{enumerate}

%The overall magnetic configuration of this active region is not clear in these
%data. Combining magnetograms from HMI with these results will show whether the
%location of enhancement is correlated with magnetic field strength. By the
%Lorentz force, the force with which charged particles can flow along magnetic
%field lines is proportional to the magnetic field strength, so if this was in
%fact governing the amount of energy that these particles injected into the
%chromosphere, it is expected that the areas of high magnetic field strength
%would be co-spatial with the areas of enhancement. Where are the main AR
%features (umbra, penumbra) relative to the enhanced areas in the power maps? As
%time goes on (especially before and after the flare), are the 3-minute
%oscillations enhanced in different places, or always in the same spot?

%The overall magnetic configuration of this active region is not clear in these
%data. Combining magnetograms from HMI with these results will show whether the
%location of enhancement is correlated with magnetic field strength. By the
%Lorentz force, the force with which charged parti- cles can flow along magnetic
%field lines is proportional to the magnetic field strength, so if this was in
%fact governing the amount of energy that these particles injected into the
%chromosphere, it is expected that the areas of high mag- netic field strength
%would be co-spatial with the areas of enhancement. Where are the main AR
%features (umbra, penumbra) relative to the enhanced areas in the power maps? As
%time goes on (especially before and after the flare), are the 3-minute
%oscillations enhanced in different places, or always in the same spot?


%It is possible that important spatial information is contained
%in location of the flare core,
%but cannot be extracted due to the saturation in the pixels.
%Saturation does not occur as often for flares
%of less powerful classes.
%\mynote{
%    remember proposal\ldots make sure you understand the difference
%    between ``strong'' and ``weak'' flares
%    vs. ``large'' and ``small'' flares.}

%The methods used in the current work were reasonably
%successful in determining the spatial location and distribution
%of concentrated power within a timespan of about 15 minutes.

%Future work will address the question of whether the chromospheric response
%reflects the rate and magnitude of energy input, or if it is more
%characteristic of the ambient plasma than of the nature of the disturbance.

%The pre-flare data shown in this work may not be the best representation
%since another, smaller flare took place in the middle of it.
%It may be worthwhile to obtain more data at earlier times

There are several possibilities for the continuation of this work.

Here we focused on the oscillations centered around the 3-minute period,
but the inclusion of other periods in the typical range of QPPs periods
will be helpful to see how the behavior of the 3-minute oscillations differs
from others, to set it apart from the range of frequencies excited due
to energy injection.

Since several of the data images saturated during the main phase of the flare,
spatial information cannot be obtained at the core location of the flare.
It may be worthwhile to apply these methods to a less powerful flare.

The temporal behavior of oscillations during the main flare remains
inconclusive due to the necessary balance between temporal and frequency
resolution.
Techniques to improve temporal resolution,
such as the standard wavelet analysis
presented by \cite{Torrence1998},
will allow study of chromospheric behavior on timescales comparable to those
over which flare dyanmics are known to occur.

Indeed, the timescales over which the oscillatory power changed in the study by
\cite{Milligan2017}
were much shorter than the sample time length used here.
The choice of $T$ was necessary to obtain sufficient frequency resolution
with the techniques utilized here, at the expense of temporal resolution.

