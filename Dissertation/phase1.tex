\section{Phase 1}


\mynote[Sun Dec  2 05:56:22 MST 2018]
{This is what \textit{mynote} looks like.}

%% Observations {---
% ---}

\begin{figure*}[htb!]\centering
    \includegraphics[draft=false,width=1.0\textwidth]{images.pdf}
    \caption{
        Images of active region 11158 in AIA 1600\AA{} (left panels),
        AIA 1700\AA{} (middle panels), and HMI LOS magnetogram (right panels),
        scaled to $\pm300$ Gauss.
        The top panels show the full disk as imaged by the instruments,
        and the bottom panels show the region used for analysis in this study.
        \label{images}}
\end{figure*}

\clearpage
\begin{figure*}[htb!]\centering
    \includegraphics[draft=false,width=1.0\textwidth]{lc.pdf}\\
    \includegraphics[draft=false,width=1.0\textwidth]{lc_log_20181128.pdf}\\
    \includegraphics[draft=false,width=1.0\textwidth]{lc_goes.pdf}
    \caption{%
        Top: Light curves of the
        UV continuum emission from AIA 1600\AA{} (blue curve) and
        AIA 1700\AA{} (red curve),
        integrated over the flare region in AR 11158.
        Middle: Same as top, but scaled as log(flux).
        %Bottom: Light curves from top panel normalized between 0.0 and 1.0,
        %overlaid with flux from the \textit{GOES} 1-8\AA{} channel.
        Bottom: Light curves from \textit{GOES-15}
        channels 1-8\AA{} (black curve) and 0.5-4\AA{} (pink curve),
        scaled as log(flux) to enable visibility of the increases
        during smaller events before and
        after the main X-flare.
        \label{lc}}
\end{figure*}


\clearpage
\begin{figure}[htb!]\centering
    \includegraphics[draft=false, width=\textwidth]{detrendedLC_20181202.pdf}
    \caption{%
        Light curves of AIA 1600\AA{} and AIA 1700\AA{}, overlaid with
        same light curves after applying a FFT filter with a period
        cutoff of 400 seconds.
    \label{detrended}}
\end{figure}

\mynote[Sun Dec  2 12:37:06 MST 2018]{
    Plot of detrended data goes here.
    Threshold = lower limit on frequency
    to get rid of global variations (e.g. the flare) and keep
    high-frequency variations, like QPPs.}

\begin{figure}[htb!]\centering
    \includegraphics[draft=false, width=\textwidth]{detrended_20181201.pdf}
    \caption{%
        Power spectrum from
        AIA 1600\AA{} flux between 1:30 and 2:30 UT.
        The orange curve is the power from the raw data, and the blue
        curve is the power from flux after applying an FFT high-pass filter
        to remove variations with periods longer than 400 seconds,
        such as the global flare curve, which dominates the spectrum at
        low frequencies.
    \label{detrended}}
\end{figure}


%% Data Prep/reduction {---
% ---}

%% Data Analysis {---
% ---}


\begin{figure*}[htb!]\centering
    \includegraphics[draft=false,width=1.0\textwidth]{aia1600maps_before.pdf}
    \includegraphics[draft=false,width=1.0\textwidth]{aia1700maps_before.pdf}
    \includegraphics[draft=false,width=1.0\textwidth]{lc_before.pdf}
    \caption{%
        Spatial distribution of 3-minute power (arbitrary instrumental units)
        for AIA 1600\AA{} (top) and AIA 1700\AA{} (bottom)
        with the central frequency at 5.6 mHz ($\pm$ 0.5 mHz).
        The $x$ and $y$ dimensions are the same as the images in
        Figure~\ref{images}.
    \label{powermaps}}
\end{figure*}

\begin{figure*}[htb!]\centering
    \includegraphics[draft=false,width=1.0\textwidth]{aia1600maps_during.pdf}
    \includegraphics[draft=false,width=1.0\textwidth]{aia1700maps_during.pdf}
    \includegraphics[draft=false,width=1.0\textwidth]{lc_during.pdf}
    \caption{%
        Mid-flare power maps.
    \label{powermaps_during}}
\end{figure*}

\begin{figure*}[htb!]\centering
    \includegraphics[draft=false,width=1.0\textwidth]{aia1600maps_after1.pdf}
    \includegraphics[draft=false,width=1.0\textwidth]{aia1700maps_after1.pdf}
    \includegraphics[draft=false,width=1.0\textwidth]{lc_after1.pdf}
    \caption{%
        Post-flare power maps.
    \label{powermaps_after1}}
\end{figure*}

\begin{figure*}[htb!]\centering
    \includegraphics[draft=false,width=1.0\textwidth]{aia1600maps_after2.pdf}
    \includegraphics[draft=false,width=1.0\textwidth]{aia1700maps_after2.pdf}
    \includegraphics[draft=false,width=1.0\textwidth]{lc_after2.pdf}
    \caption{%
        Post-flare power maps.
    \label{powermaps_after2}}
\end{figure*}

\begin{figure*}[htb!]\centering
    \includegraphics[draft=false,width=0.48\textwidth]{aia1600big_image.pdf}
    \includegraphics[draft=false,width=0.48\textwidth]{aia1700big_image.pdf}\\
    \includegraphics[draft=false,width=0.48\textwidth]{aia1600big_map.pdf}
    \includegraphics[draft=false,width=0.48\textwidth]{aia1700big_map.pdf}
    \caption{%
        Post-flare power maps overlaid with contours showing the approximate
        location of the $B_{LOS}$ at $\pm$300 Gauss.
        White and black contours represent positive and negative polarity,
        respectively.
    \label{contours}}
\end{figure*}

\begin{figure*}[htb!]\centering
    \includegraphics[draft=false,width=0.9\textwidth]{wa.pdf}
    \caption{
        Time-frequency power plots from AIA 1600\AA{} (top panel) and AIA
        1700\AA{} (bottom panel), obtained by applying a Fourier transform to
        integrated emission from NOAA AR 11158 in discrete time increments of
        64 frames ($\sim$25.6 minutes) each. The dashed horizontal line marks the
        central frequency $\nu_{c}$ at $\sim$5.6 mHz, corresponding to a period
        of 3 minutes. The dotted horizontal lines on either side of $\nu_{c}$
        mark the edges of the frequency bandpass $\Delta\nu$ = 1 mHz. The
        vertical lines mark the flare, start, peak, and end times as determined
        by \textit{GOES}. The power is scaled logarithmically and over the same
        range in both channels.
        Note that the x-axis labels do not necessarily line up with the
        boundaries of the data columns.
        \label{wa}}
\end{figure*}

\begin{figure*}[htb!]\centering
    \includegraphics[draft=false,width=1.00\textwidth]{time-3minpower_flux.pdf}
    \includegraphics[draft=false,width=1.00\textwidth]{time-3minpower_maps.pdf}
    \caption{%
        Temporal evolution of the 3-minute power $P_{3min}(t)$ in
        AIA 1600\AA{} (blue curve) and AIA 1700\AA{} (red curve).
        Top: $P(t)$ obtained by applying a Fourier transform to the
        integrated flux from AR 11158.
        Bottom: $P(t)$ per unsaturated pixel, obtained by summing over power maps.
        Each point in time
        %represents the sum (in x and y) of each power map and
        is plotted as a function of the center
        of the time segment over which the Fourier transform was applied to
        obtain the power map over which the point was summed.
        %The axis labeled $T = 64$ is scaled to show the
        %length of each time segment relative to the full time series.
        The vertical dashed lines mark the \textit{GOES} start, peak, and end
        times of the flare at 01:44, 01:56, and 02:06 UT, respectively.
        \label{power_vs_time}}
\end{figure*}

\clearpage
\begin{figure*}[htb!]\centering
    \includegraphics[draft=false,width=0.80\textwidth]{preflare.pdf}\\
    \includegraphics[draft=false,width=0.80\textwidth]{postflare.pdf}
\end{figure*}
