\section{%
Enhanced chromospheric 3-minute oscillatory power associated with
the 2011-February-15 X2.2 flare}

%% (aka, title of Article1)

\subsection{Introduction}

%% Observations {---

\subsection{Observations and data reduction}

\mynote{Flare on 2011 February 15 took place ``close to disk center''.
How close is ``close''? What are the difficulties in analyzing target close
to the solar limb? Opacity issues?}


The 2011 February 15 X2.2 flare
occurred in NOAA active region (AR) 11158
close to disk center
during solar cycle 24 (SOL2011-02-15T01:56).
The AR was composed of a quadrupole:
two sunspot pairs (four sunspots total).
The X-flare occurred in a delta-spot composed of
the leading spot of the southern pair and
the trailing spot of the northern pair.
It started at 01:44UT, peaked at 01:56UT, and ended at 02:06UT,
as determined by the soft X-ray flux from the
\textit{Geostationary Operational Environmental Satellite}
(\textit{GOES}-15; \cite{Viereck2007}).
The impulsive phase lasted about 10 minutes.
Data covering 5 hours centered on this flare were used for the analysis.
This data includes a C-class flare that occurred between 00:30 and 00:45 UT
on 15 February 2011.



\begin{figure*}[htb!]\centering
    \includegraphics[draft=false,width=1.0\textwidth]{images.pdf}
    \caption{
        Images of active region 11158 in AIA 1600\AA{} (left panels),
        AIA 1700\AA{} (middle panels), and HMI LOS magnetogram (right panels),
        scaled to $\pm300$ Gauss.
        The top panels show the full disk
        (a nice example of the capabilities of AIA),
        and the bottom panels show the region used for analysis in this study.
        \label{images}}
\end{figure*}

\begin{figure*}[htb!]\centering
    \includegraphics[draft=false,width=1.0\textwidth]{images.pdf}
    \caption{
        Images of active region 11158 in AIA 1600\AA{} (left panels),
        AIA 1700\AA{} (middle panels), and HMI LOS magnetogram (right panels),
        scaled to $\pm300$ Gauss.
        The top panels show the full disk,
        and the bottom panels show the region used for analysis in this study.
        \label{images2}}
\end{figure*}


\begin{figure}[htb!]\centering
    \includegraphics[draft=false,width=0.5\textwidth]{hmi_label_sunspots_20190124.pdf}
    \caption{
        HMI LOS magnetogram.
        White and black contours outline positive (+300 Gauss)
        and negative (-300 Gauss) polarities, respectively.
        The two sunspots in the northern pair are labeled AR\_1a (leading sunspot)
        and AR\_1b (trailing sunspot).
        The two sunspots in the southern pair are labeled AR\_2a (leading sunspot)
        and AR\_2b (trailing sunspot).
        \label{label}}
\end{figure}


\clearpage
\begin{figure*}[htb!]\centering
    \includegraphics[draft=false,width=1.0\textwidth]{lc.pdf}\\
    \includegraphics[draft=false,width=1.0\textwidth]{lc_log_20181128.pdf}\\
    \includegraphics[draft=false,width=1.0\textwidth]{lc_goes_20190124.pdf}
    \caption{%
        Top: Light curves of the
        UV continuum emission from AIA 1600\AA{} (blue curve) and
        AIA 1700\AA{} (red curve),
        integrated over the flare region in AR 11158.
        Middle: Same as top, but scaled as log(flux).
        Bottom: Light curves from \textit{GOES-15}
        channels 1-8\AA{} (black curve) and 0.5-4\AA{} (pink curve),
        scaled as log(flux) to enable visibility of the increases
        during smaller events before and
        after the main X-flare.
        \label{lc}}
\end{figure*}


\clearpage
\begin{figure}[htb!]\centering
    \includegraphics[draft=false, width=\textwidth]{detrendedLC_20181202.pdf}
    \caption{%
        Light curves of AIA 1600\AA{} and AIA 1700\AA{}, overlaid with
        same light curves after applying a FFT filter with a period
        cutoff of 400 seconds.
    \label{detrended}}
\end{figure}

\mynote[Sun Dec  2 12:37:06 MST 2018]{
    Plot of detrended data goes here.
    Threshold = lower limit on frequency
    to get rid of global variations (e.g. the flare) and keep
    high-frequency variations, like QPPs.}

\begin{figure}[htb!]\centering
    \includegraphics[draft=false, width=\textwidth]{detrended_20181201.pdf}
    \caption{%
        Power spectrum from
        AIA 1600\AA{} flux between 1:30 and 2:30 UT.
        The orange curve is the power from the raw data, and the blue
        curve is the power from flux after applying an FFT high-pass filter
        to remove variations with periods longer than 400 seconds,
        such as the global flare curve, which dominates the spectrum at
        low frequencies.
    \label{detrended2}}
\end{figure}


Figure~\ref{lc} shows light curves for the full 5-hour time series
from 00:00 to 04:59 on 2011-February-15.
The top panel shows both AIA channels.
The bottom panel shows both SXR channels from \textit{GOES-15} at
1-8\AA{} (black curve) and 0.5-4\AA{} (pink curve).
%overlaid and similarly scaled for comparison.
A small C-flare occurred before the X-flare between 00:30 and 00:45 UT, and
two small events occurred after the X-flare,
between 03:00 and 03:15, and between 04:25 and 04:45.


Pre-flare images of the full disk are shown in Figure~\ref{images},
along with a 300x198 arcsecond subset of the data centered on AR 11158.
{This subset was extracted and aligned by cross correlation}
\citep{McAteer2003,McAteer2004}.
Images were scaled to improve contrast using the
\textit{aia\_intscale.pro} routine from \textit{sswidl}.
The magnetic configuration of the quadrupole is clear in the HMI magnetograms.
The northern pair will be designated as AR\_1
and the southern pair will be designated as AR\_2.
Sunspots in the northern pair will be designated as AR\_1a (positive polarity)
and AR\_1b (negative polarity).
Sunspots in the southern pair will be designated as AR\_2a (positive polarity)
and AR\_2b (negative polarity).

Both AIA channels saturated ($\geq\!15000$ counts) in the center
during the peak of the X-class flare, and a few pixels also saturated
during the smaller events before and after.
Affected pixels were {all} contained within the
300x198 arcsecond {subset of data}
throughout the duration of the time series.
Four images from the 1700\AA{} channel on AIA were missing, between
the images with start times at
00:59:53.12,
01:59:29.12,
02:59:05.12, and
03:58:41.12, and the following images, each with start times
48 seconds after the previous image.
Since the gaps in data were separated by an hour,
it was reasonable to approximate {missing images}
by averaging the two adjacent images.

% end of text copied from Article1 (28 January 2019)
%
%
%

% ---}





\subsection{Analysis}
%% Data Analysis {---
% ---}


\subsection{Results and discussion}

\begin{figure*}[htb!]\centering
    \includegraphics[draft=false,width=1.0\textwidth]{aia1600maps_before.pdf}
    \includegraphics[draft=false,width=1.0\textwidth]{aia1700maps_before.pdf}
    \includegraphics[draft=false,width=1.0\textwidth]{lc_before.pdf}
    \caption{%
        Spatial distribution of 3-minute power (arbitrary instrumental units)
        for AIA 1600\AA{} (top) and AIA 1700\AA{} (bottom)
        with the central frequency at 5.6 mHz ($\pm$ 0.5 mHz).
        The $x$ and $y$ dimensions are the same as the images in
        Figure~\ref{images}.
    \label{powermaps}}
\end{figure*}

\begin{figure*}[htb!]\centering
    \includegraphics[draft=false,width=1.0\textwidth]{aia1600maps_during.pdf}
    \includegraphics[draft=false,width=1.0\textwidth]{aia1700maps_during.pdf}
    \includegraphics[draft=false,width=1.0\textwidth]{lc_during.pdf}
    \caption{%
        Mid-flare power maps.
    \label{powermaps_during}}
\end{figure*}

\begin{figure*}[htb!]\centering
    \includegraphics[draft=false,width=1.0\textwidth]{aia1600maps_after1.pdf}
    \includegraphics[draft=false,width=1.0\textwidth]{aia1700maps_after1.pdf}
    \includegraphics[draft=false,width=1.0\textwidth]{lc_after1.pdf}
    \caption{%
        Post-flare power maps.
    \label{powermaps_after1}}
\end{figure*}

\begin{figure*}[htb!]\centering
    \includegraphics[draft=false,width=1.0\textwidth]{aia1600maps_after2.pdf}
    \includegraphics[draft=false,width=1.0\textwidth]{aia1700maps_after2.pdf}
    \includegraphics[draft=false,width=1.0\textwidth]{lc_after2.pdf}
    \caption{%
        Post-flare power maps.
    \label{powermaps_after2}}
\end{figure*}

\begin{figure*}[htb!]\centering
    \includegraphics[draft=false,width=0.48\textwidth]{aia1600big_image.pdf}
    \includegraphics[draft=false,width=0.48\textwidth]{aia1700big_image.pdf}\\
    \includegraphics[draft=false,width=0.48\textwidth]{aia1600big_map.pdf}
    \includegraphics[draft=false,width=0.48\textwidth]{aia1700big_map.pdf}
    \caption{%
        Post-flare power maps overlaid with contours showing the approximate
        location of the $B_{LOS}$ at $\pm$300 Gauss.
        White and black contours represent positive and negative polarity,
        respectively.
    \label{contours}}
\end{figure*}

\begin{figure*}[htb!]\centering
    \includegraphics[draft=false,width=0.9\textwidth]{wa.pdf}
    \caption{
        Time-frequency power plots from AIA 1600\AA{} (top panel) and AIA
        1700\AA{} (bottom panel), obtained by applying a Fourier transform to
        integrated emission from NOAA AR 11158 in discrete time increments of
        64 frames ($\sim$25.6 minutes) each. The dashed horizontal line marks the
        central frequency $\nu_{c}$ at $\sim$5.6 mHz, corresponding to a period
        of 3 minutes. The dotted horizontal lines on either side of $\nu_{c}$
        mark the edges of the frequency bandpass $\Delta\nu$ = 1 mHz. The
        vertical lines mark the flare, start, peak, and end times as determined
        by \textit{GOES}. The power is scaled logarithmically and over the same
        range in both channels.
        Note that the x-axis labels do not necessarily line up with the
        boundaries of the data columns.
        \label{wa}}
\end{figure*}

\begin{figure*}[htb!]\centering
    \includegraphics[draft=false,width=1.00\textwidth]{time-3minpower_flux.pdf}
    \includegraphics[draft=false,width=1.00\textwidth]{time-3minpower_maps.pdf}
    \caption{%
        Temporal evolution of the 3-minute power $P_{3min}(t)$ in
        AIA 1600\AA{} (blue curve) and AIA 1700\AA{} (red curve).
        Top: $P(t)$ obtained by applying a Fourier transform to the
        integrated flux from AR 11158.
        Bottom: $P(t)$ per unsaturated pixel, obtained by summing over power maps.
        Each point in time
        %represents the sum (in x and y) of each power map and
        is plotted as a function of the center
        of the time segment over which the Fourier transform was applied to
        obtain the power map over which the point was summed.
        %The axis labeled $T = 64$ is scaled to show the
        %length of each time segment relative to the full time series.
        The vertical dashed lines mark the \textit{GOES} start, peak, and end
        times of the flare at 01:44, 01:56, and 02:06 UT, respectively.
        \label{power_vs_time}}
\end{figure*}

\clearpage
\begin{figure*}[htb!]\centering
    \includegraphics[draft=false,width=0.80\textwidth]{preflare.pdf}\\
    \includegraphics[draft=false,width=0.80\textwidth]{postflare.pdf}
\end{figure*}


\subsection{Conclusions}
