\section{The solar chromosphere}

%% Chromosphere, in general {---



\myheading{The flaring chromosphere}
Response of chromosphere during flares:
Most focus is on corona, where energy
is stored and MR and CMEs are released.
But chromosphere dominates the
radiative energy budget, so somehow energy is transported down rapidly, and how
the layers respond can help us characterize energy transport and
conversion.% Yes this does work to keep mynote on same line in pdf :)
%
\mynote[8/14/18]
{Written on the back of post-it notes pad, no date.
Not sure if I was working on how to word something in paper/dissertation, or
just testing myself to see if I know the point of my own research.}




To study the chromospheric response to flares, it is necessary
to first understand the structure and dynamics of the chromosphere
during quiet, non-flaring periods.

The chromosphere got its name from Lockyer and Frankland because of the vivid
red color of the limb (from Hα emission) before and after eclipses.
According to Hudson (2007), the chromosphere is a ``well-defined layer''.

Globally, the chromosphere is a complex, dynamic interface that lies between
the visible disk of the photosphere and the hot corona.
About 500 km above the base of the photosphere at $\sim$5800 K, the temperature
decreases with distance as one would expect from basic physics,
until it reaches $\sim$4200 K.
This is known as the temperature minimum $T_{min}$, and
marks the lower boundary of the chromosphere.
The absolute
thickness of the chromosphere varies in the literature, usually between 2000
and 5000 km (close to the radius of the Earth).

At the upper boundary of the chromosphere, the temperature rapidly increases to
the order of a million K in the corona.
This temperature jump takes place in a region defined as the transition
region (TR), a relatively thin layer of plasma that separates the chromosphere
from the overlying corona.

The base of the photosphere (z=0) is defined as where
$\tau_{5000}$ = 1.
The temperature reaches 10000 K at z $\approx$ 2300 km, and then skyrockets
to a million K over the TR (Kneer \& Von Uexküll).


\mynote{
    What makes the chromosphere a distinct, separate layer of the
    solar atmosphere? It can be defined in multiple ways, including
    temperature profile (Tmin to rapid increase that defines the TR),
    structure (high to low $\beta$), or if you want to be really technical,
    the optical depth at the limb.
    Why is the atmosphere divided this way?
    What makes the chromosphere distinct from the other layers?}




\clearpage
\subsection{Magnetic field, structure, and the \textit{plasma-$\beta$}}



The lower region of the chromosphere reveals
a magnetic network similar to that of the photosphere, up to
$\sim$1300-1500 km. The pattern reflects that of the supergranular
pattern at the photosphere. Above this, the consistent, homogeneous
plasma gives way to complex structures (see \cite{Judge2006} for a
review). At the boundary between these two regions, the dominant
pressure changes from thermal gas pressure to magnetic pressure. This
property is quantified in a parameter called the \textit{plasma
$\beta$}, defined as the ratio of thermal gas pressure to magnetic
pressure:

\begin{equation}
    \beta \;=\; \frac{P_{thermal}}{P_{magnetic}}
    \;=\; \frac{\rho k_{B}T/\mu m_{amu}}{B^{2}/8\pi}
    \;\propto\; \frac{\rho T}{B^{2}}
\end{equation}

where $k_{B} = 1.38 \times 10^{-16}$ [erg K$^{-1}$] is Boltzmann's
constant, $\mu$ is the average mass per particle in atomic mass units
[amu], $m_{amu} = 1.67 \times 10^{-24}$ [g] is the nucleon mass, $\rho$
is the mass density, $T$ is the local temperature, and $B$ is the
magnetic field strength [gauss].

In practice, a plasma is described by expressing its $\beta$
relative to unity (rather than a specific numerical value). A plasma
with $\beta \gg 1$ is dominated by thermal pressure, and appears as a
fairly homogeneous, uniform distribution of material. A plasma with
$\beta \ll 1$ is dominated by magnetic pressure and is characterized
by complex structures, such as spicules and coronal loops.
Figure~\\ref{fig:plasma\_beta}
shows how $\beta$ varies with height
throughout the solar atmosphere. The balance between the two pressures
plays a vital role in the structure and dynamics of a plasma, rather
than the absolute strength of the magnetic field alone. At the
$\beta=1$ boundary, mode conversion can occur for propagating acoustic
waves.

\begin{framed}
    Physics sidebar: mode conversion, $\beta = 1$ surface.
\end{framed}

% Structures
Magnetic flux is generally concentrated in active regions, such as the umbrae
of sunspots. As this flux rises into the atmosphere, it gives rises to the
complex structures that are observed as the emission from particles that flow
along magnetic field lines (a result of the so-called ``frozen-in'' theorem).
Magnetic structures play an important role in diagnosing conditions in the
solar atmosphere. Field lines often serve as waveguides for oscillations in the
atmosphere, revealed indirectly by the manner in which the plasma responds to a
disturbance. Structures are also a potential means of transportation of mass
and energy to the corona.


The upper part of
the chromosphere contains structures called spicules that emit the distinct,
dark pink color of H$\alpha$, and can be seen around the disk of the sun during
solar eclipses. Above the base of the photosphere (where $T \approx 5800$~K),
the temperature falls off with distance until it reaches the so-called
\textit{temperature minimum}, at $T_{min} \approx 4200$~K. The height at which
$T = T_{min}$ marks the lower boundary of the chromosphere, about 500 km above
the base of the photosphere. This is where acoustic waves shock and dissipate,
causing a sharp increase in temperature. The absolute thickness of the
chromosphere varies roughly between 2000 and 5000 km, slightly less than the
radius of the Earth. The temperature reaches $\sim$10000 K at the upper
boundary of the chromosphere, at $z \approx 2300$ km, and then rapidly
increases to $T\!\gtrsim\!10^{6}$~K in the corona.
This temperature jump takes
place in a region defined as the \textit{transition region} (TR), a relatively
thin layer of plasma that separates the chromosphere from the overlying corona.
The temperature gradient in the solar atmosphere above the base of the
photosphere is shown in
Figure~\\ref{fig:solar\_atm}.



\subsection{Challenges in observing the chromosphere}




% ---}

%% 3-minute oscillations {---
\clearpage
\section{3-minute oscillations}

\mynote{
    When were 3mOs first observed and why? What was the motivation
    behind the research article that produced the first results?
}

3mOs were
first discovered via velocity observations in the quiet chromosphere
\citep{Jensen1963}.

Intensity oscillations with periods of 2-3 minutes have been observed in
the chromosphere starting in the 1960s.
They have been observed
via intensity variations in EUV lines.
In active regions, intensity oscillations were discovered by
\cite{Beckers1969} above sunspot umbrae in Ca$^{+}$,
and given the term ``umbral flashes'' to characterize
both their location and transient nature.
A short follow-up study from \cite{Wittmann1969} revealed
flashes with a period of 150 seconds (2.5 minutes).
Beckers and company continued to pursue these oscillations in
\citep{Beckers1972},
with more intensity oscillations and the addition of velocity measurements
of the umbrae at the photospheric level.
The highest power tended toward the area
directly above sunspot umbrae, with peak-to-peak velocities up to
1 km s$^{-1}$.
They found clear periods of
178 seconds (2.97 minutes) at the umbral center,
255 seconds (4.25 minutes) at the umbral edge, and
300 seconds (5 minutes) in the penumbra.
They did not find a connection between velocity oscillations in the photosphere
and intensity oscillations in the chromosphere.

In the quiet chromospheric internetwork regions,
the Fourier velocity power spectrum peaks at
$\sim$5.5 mHz (3 minutes)
\citep{Orrall1966}.

\cite{Giovanelli1972} reported umbral oscillations in both velocity and intensity.

Many authors have dedicated multiple publications to investigating
this phenomenon. Lites is the first author on many papers submitted
over the course of a decade or so, between 1982 and 1992.

\myheading{Initial interpretation}

3mOs were initially believed to exist in cavities as
trapped, standing waves
\citep{Scheuer1981}.
Chromospheric cavity between Tmin and TR
(\cite{Chae2015}
$\rightarrow$
Leibacher \& Stein (1981))


\subsection{Observations in intensity and velocity}
\subsection{Location relative to chromospheric network and active regions}


\subsection{Origin of the 3mOs}

Both the origin of the 3-minute oscillations and the reason for their
persistence remain unclear.

3mOs are currently interpreted as slow
magnetoacoustic waves propagating along magnetic field lines.

Results from \cite{Tian2014} supported the propagation of shock waves with peak-to-peak
velocity amplitude between 5 and 10 km s$^{-1}$,
$\lesssim$ the local sound speed.

Many other authors have contributed to the
current interpretation as slow propagating SMA waves.
Similar results were found by
\cite{
    Brynildsen1999b,
    Brynildsen2002,
    Brynildsen2004,
    Maltby1999,
    OShea2002,
    DeMoortel2000,
    Reznikova2012
}


\begin{itemize}
    \item The periodic pattern of
        sharp blueshifts followed by gradual redshifts
        in velocity observations reveal a
        sawtooth shape characteristic of propagating shock wave fronts.
    \item The shift toward the blue wing in emission lines (toward the
        observer) indicates that indicates an upward propagation through the
        atmosphere. The lack of accompanying redshift supports the
        interpretation of propagating waves and not standing waves reflected
        back toward the solar surface.
\end{itemize}

The previous two points (blueshift and sawtooth pattern)
have lead to
the interpretation of these patterns as manifestations of upward
propagating magnetoacoustic waves along magnetic field lines.



\cite{Bogdan2006}, page 319:
Various ionization stages (increasing T) $\rightarrow$ high space and time
coherence at different heights in the 3-min band.
High time cadence
$\rightarrow$ phase shifts, which agree with wavepackets propagating from
photosphere to the base of the chromosphere with vertical phase velocity
$\sim$cs (7-12 km per second).


\myheading{Phot $\rightarrow$ Chrom}

\textbf{What is the manner in which oscillations propagate from the photosphere to
the chromosphere?}

The photosphere was already known to oscillate at a dominant frequency of 5
minutes.
5-minute oscillations are observed in the chromosphere as well, mostly in
the network regions of the quiet sun, where magnetic field strength is
relatively high.
3mOs are possibly connected to the global 5-minute oscillations.

The 5-minute period in the photosphere has been
attributed to the global pulsations of the sun due to pressure modes
(p-modes) in the interior. In the photosphere, the Fourier power spectrum
peaks at $\sim$3 mHz (5 minutes) in the quiet sun.
This power is reduced by a
factor of 2-5 at the umbra of sunspots, though it is still the dominant
frequency \citep{Felipe2010; Bogdan2006 p. 323}.
The amplitudes of intensity variations in the photosphere tend to be relatively
small and difficult to measure, while those in the chromosphere are
dominated by the aforementioned umbral flashes.
The non-isothermal region where a compressive acoustic disturbance is
propagating along the magnetic field in
direction $\hat{k}$
Umbral $\vec{B}$ aligned with $\vec{g}$, so $H_{||} \rightarrow$
`true' $H_{P}$.

\begin{equation}
    k_{||}^{2} = \frac{\omega^{2}}{c^{2}} - \frac{1}{4H_{||}^{2}}
\end{equation}

where $\omega = 2\pi\nu$ and $H = H_{P}$,
the pressure scale height
%\citep{Cally2001; Crouch2003}.
(Cally 2001; Crouch \& Cally 2003).

Waves propagate if
$k_{||}^{2} \ge 0$
($ \omega \ge c/2H_{||} $)
threshold = acoustic cutoff frequency.
$\Omega = \Omega(z)$ = max value of 5.2 - 5.5 mHz (3.21 - 3.03 minutes) with
Tmin \mynote{Figure 2. of Cally et al. 1994}
3-minute oscillations at the high frequency tail of photospheric
oscillations.
Conservation of wave energy: amplitude of velocity oscillation
$\propto 1/\sqrt{\rho}$ for
waves with frequency greater than the cutoff.
For
waves with frequency less than the cutoff,
``osc do not give rise to prop. waves''.
Dispersion relation determines
decline in velocity predicted by the imaginary part of the wave number.

``The dominant power above SS umbrae changes from 5 minutes in the photosphere to
3 minutes in the chromosphere because of strong spatial damping of evanescent
waves with height, whereas 3mOs at the cutoff frequency are not damped.''
\mynote{
    \cite{Milligan2017} $\rightarrow$ Noyes \& Leighton 1963.}
Vertical velocity oscillations extending
into the chromosphere
(\cite{Judge2006} $\rightarrow$ Jensen \& Orrall (1963);
Noyes \& Leighton (1963)).


\myheading{Chrom $\rightarrow$ Corona}

\textbf{What is the manner in which oscillations propagate from the chromosphere
to the corona?}

Bogdan \& Judge (2006)
\mynote{
    and Tian et al. (2014)?}
: Intensity
oscillations (in emission lines) have been observed to ``disappear'' after
traveling from the chromosphere through the transition region toward the corona
(around T $\ge 10^{6}$ K).
The disappearance of a wave is generally attributed to
dissipation, damping, or mode conversion.
Thermal conduction is the principal
dissipation mechanism at coronal temperatures, but it doesn't affect 3mOs
until they travel a few thousand km into the corona.
Slow magnetoacoustic waves are not subject to Ohmic dissipation.
Viscosity can be neglected for a
collisionless plasma. They can’t be converted from slow to fast magnetoacoustic
waves because this phenomenon would take place at the coupling region
(where $\beta \approx 1$),
and the waves are observed to propagate at heights well above this.
This
observation also cannot be explained by reflection at some layer because that
would result in a redshift in the observations, which is not the case.
The best explanation is that the rapid increase in the temperature scale height
causes the waves to dissipate.

    \begin{equation}
        \frac{2\pi}{k_{||}} \gg H_{T}
    \end{equation}

A wavetrain passing through the formation height of line causes uniform lift,
compression, descent, and rarefaction over the entire layer over a single wave
period. Exception to this: strong resonance lines like
\ion{Ca}{2} H \& K and H$\alpha$, which
are optically thick and broad.
Coronal emission lines are optically thin.

Compression and rarefaction reduce the overall contributions to net
fluctuations in coronal emission line.
\ion{Fe}{16}:
Changes in emission line intensity time series as move upward in z
(and hence, in T)
\mynote{
    \cite{OShea2002}}
How high do they propagate? This question was posed by Reznikova et al. (2012).





There are two dominating theories of where the 3mOs originate.


\subsubsection{Slow, propagating magnetoacoustic waves from the photosphere}

One is that they are generated below the chromosphere and
propagate upward. The dominant peak at 3 minutes is explained by
the acoustic cutoff frequency in the chromosphere, which is around
5.5 mHz.
\mynote{
    See appendix or sidebar for physics details on the
    cutoff frequency? This is where my own plot should go!}
Waves with frequencies lower than the cutoff are unable to continue
propagating, therefore lower frequencies (longer periods) are not
observed at those heights.
\mynote{
    What heights? Where does $\omega_{0}$ reach a maximum?}


\subsubsection{Response of chromosphere at the cutoff frequency}

The second theory regarding the origination of 3-minute oscillations
is that the acoustic cutoff frequency is the frequency at which the
chromosphere naturally responds when a disturbance is introduced.
In other words, the oscillations are
generated within the chromospheric plasma, where they are observed.

This theory has been investigated and predicted by several authors.
\cite{Fleck1991} found similar behavior in
both isothermal and non-isothermal models, suggesting that the temperature
gradient throughout the chromosphere does not have an effect on its oscillatory
response. \cite{Kalkofen1994} simulated disturbances from both a single impulse
and a continuous jostling (i.e. a periodic piston), with frequencies above and
below the cutoff. The response to the single pulse tended toward the cutoff
value, analogous to a bell sounding at a particular frequency when it is
struck. A series of papers analytically examined adiabatic wave excitations in
a gravitationally stratified atmosphere \citep{Sutmann1995a, Sutmann1995b,
Sutmann1998}. These studies also supported a similar response to a disturbance,
regardless of whether they were impulsively or continuously introduced into the
atmosphere. \cite{Chae2015} used simulations of a gravitationally stratified
medium to show that 3-minute oscillations naturally arise when such a medium is
disturbed.

While many studies have predicted this behavior,
actual observations of this phenomenon are quite rare.
\cite{Kwak2016}
observed enhanced oscillations in response to a downflow
event attributed to a plume-like feature above the umbra of a sunspot.


\begin{framed}
    Physics sidebar: \textbf{What are slow magnetoacoustic waves?}

    Like standard acoustic waves, magnetoacoustic waves are longitudinal,
    compressible, and anisotropic waves in a magnetized plasma.
    They perturb the plasma density, allowing intensity variations to be
    observed.
    They also perturb the parallel component of the magnetic field
    \mynote{
        or velocity?}.
    MA waves exist as either slow or fast MA waves, both of which can be either
    standing or propagating.
    Slow magnetoacoustic waves tend to damp out
    quickly (after 1-3 oscillations) due to energy losses from
    radiation/thermal conduction.
    They travel sub-sonically; their
    speed (phase velocity) is lower than the local sound speed.
    They are polarized perpendicular to the magnetic field.
    They are observed
    \mynote{
        at least in the corona}
    via both intensity variations (in EUV lines) and Doppler shifts.
    (See \cite{Nakariakov2005} for a general review of MHD waves and coronal
    seismology).
    Speed close to sound speed
    allows interpretation of slow, propagating MA waves.
\end{framed}


% ---}

%% Cutoff frequency {---
\newpage
\section{The acoustic cutoff frequency}
In the interior of the sun, acoustic waves have frequencies higher than the
cutoff, so it is of less importance there. In the solar atmosphere, however,
the cutoff plays an important role in the behavior of waves and oscillatory
phenomena.
The cutoff peaks in the chromosphere, at a value of about 5.5 mHz,
or roughly three minutes. This is also the dominant frequency of the
oscillations observed everywhere at all times in the chromosphere.

If waves with frequencies lower than the cutoff are unable to propagate
at a certain height, then one would expect to see a continuously high power
spectrum at frequencies higher than this, and then a drop to zero at
frequencies lower than this. However, what we see is a peak at the cutoff,
but the power drops again for higher frequencies.

\mynote{Where are the higher frequencies?}

This may be because waves at higher frequencies simply aren't generated
in the first place.

\mynote{If the cutoff frequency didn't exist, how would power at lower
frequencies compare to the 3-minute power? Higher or lower? Or do these
waves even exist to begin with?}

\mynote{What types of waves are generated in the atmosphere, and why?}

Figure~\\ref{fig:Stangalini2011} shows power maps from
\cite{Stangalini2011} illustrating the power distribution of the
3-minute oscillations at the photospheric Fe~6173~\AA{} line and the
chromospheric Ca~8542~\AA{} line. Also shown is the power distribution
of the 5-minute oscillations. The 5-minute period has been attributed
to the global pulsations of the sun due to pressure modes
(\textit{p}-modes) in the interior. The image shows the suppression of
the 5-minute period in the umbra of sunspots, the opposite of the
3-minute behavior in the chromosphere above sunspots. It is commonly
observed in the photosphere that the 5-minute power above sunspot
umbrae is \emph{reduced} compared to the quiet sun by a factor of 2-5,
though it is still the dominant frequency
\citep{Felipe2010, Bogdan2006}.
\cite{Reznikova2012} also observed lowering of the cutoff frequency
due to inclination of magnetic field lines.
The presence of the 5-minute oscillations in the diffuse magnetic regions
surrounding the umbra is likely caused by the decrease in effective
cutoff frequency due to the inclination of the magnetic field. Waves
that are generated in the lower atmosphere propagate upward until
their frequency no longer exceeds the local cutoff frequency and they
are reflected \citep{DeMoortel2000, Brynildsen1999}. The cutoff
frequency at the base of the chromosphere is $\sim$5.5 mHz, or roughly
3 minutes, which agrees with the acoustic cutoff period of a gas with
the composition and pressure scale height of the upper photosphere
\citep{Kalkofen1994}.

Waves with a frequency lower than the cutoff will:
\begin{itemize}
    \item be reflected back down
    \item shock and dissipate
    \item strongly damp out
\end{itemize}


\mynote{What is the `natural frequency', and what is the significance
of the chromosphere oscillating at its natural frequency
(as opposed to\ldots)? Milligan paper presented evidence supporting that
the chromosphere oscillates at its natural frequency in response to a
disturbance, but never stated why this is significant or the implications
for bigger picture science.}

\clearpage
\subsection{Derivation of the cutoff frequency}

The cutoff frequency defines the minimum frequency a wave must have in order to
propagate through a medium.
It is an intrinsic property of the medium,
determined by local macro parameters, such as pressure and density.
As these properties vary throughout the interior of the sun and its atmosphere,
the cutoff frequency itself is an indirect function of height.

\mynote{Plot of cutoff frequency with height goes here.}

Acoustic waves obey the Klein Gordon equation:
\begin{equation}
    \frac{\partial^{2}u}{\partial t^{2}}
    - \left( c_{s}^{2} \right)
    \frac{\partial^{2}u}{\partial z^{2}}
    + \Omega^{2}u
    = 0
\end{equation}

where
$u$ = velocity amplitude,
$t$ = time,
$c_{s}$ = local sound speed,
$z$ = height above the photosphere, and
where $\Omega = 2\pi\nu_{ac}$
is the cutoff frequency in radians per second
($\nu_{ac}$ = cutoff frequency in oscillations per second),


The solution for $u$ is of the form:

\begin{equation}
    u = u_{0}e^{i\left( k_{z}z - \omega t \right)}
\end{equation}

where $\omega = 2 \pi f$ is the oscillation frequency of the wave
in radians s$^{-1}$ ($f$ = frequency in Hz), and $k_{z} = 2\pi/\lambda$
is the wavenumber in the $z$-direction.
Using this solution, we eventually get:
\begin{equation}
    c_{s}^{2} k^{2} = \omega^{2} - \Omega^{2}
\end{equation}
If $\omega < \Omega$, then $k^{2} < 0$ and $k$ is imaginary.
In this case, the wave cannot propagate and standing oscillations are
observed.

\begin{equation}
    \Omega = \frac{c_{s}}{2H}
\end{equation}
$H$ = scale height.

\begin{align}
    \Omega &= \frac{c_{s}}{2H} = 2\pi\nu_{ac}\\
    \nu_{ac} &= \frac{c_{s}}{4 \pi H}\\
    \nu_{ac} &= \sqrt{ \frac{\gamma k T }{ 16 \pi^{2} H^{2} \mu m_{amu} } }
\end{align}

where $c_{s}$ is given by:
\begin{equation}
    c_{s} = \sqrt{ \frac{ \gamma P }{ \rho }}
    = \sqrt{ \frac{ \gamma k T }{ \mu m_{amu} }}
\end{equation}

Setting $\gamma$ to $5/3$, $\mu = 1.2$, $H = 150$ km,
and $ T = 4200 $ K (the approximate temperature minimum),
the acoustic cutoff frequency $\nu_{ac}$ comes out to roughly
5.6 mHz, the frequency at which the 3-minute chromospheric
period oscillates.


%% ???
%The non-isothermal region where a compressive acoustic disturbance
%is propagating along the magnetic field in direction $\hat{k}$.
%Umbral $\vec{B} \overrightarrow{B}$ aligned with
%$\vec{g}$, so $H_{||} \rightarrow$ `true'\mynote{??} $H_{P}$.
%\[
%    k_{||}^{2} = \frac{\omega^{2}}{c^{2}} - \frac{1}{4H_{||}^{2}}
%    \]
%where $\omega = 2\pi\nu$ and $H = H_{P}$, the pressure scale height.
%(Cally 2001; Crouch \& Cally 2003).
%Waves propagate if $k_{||}^{2} \geq 0 ( \omega \geq c/2H_{||} )$,
%threshold = acoustic cutoff frequency.
%$\Omega = \Omega(z)$ = max value of 5.2 - 5.5 mHz
%(3.21 - 3.03 minutes) with $T_{min}$\mynote{
%    (Fig 2. of Cally et al. 1994)}
%3-minute oscillations at the high frequency tail of photospheric
%oscillations. Conservation of wave energy: amp of velocity osc.
%$\propto 1/\sqrt{\rho}$ for waves with frequency greater than the
%cutoff. For freq less than cutoff, ``osc do not give rise to prop.
%waves''. Dispersion relation determines decline in velocity predicted
%by the \emph{imaginary} part of the wave number.

%%\subsection{Physical background for the cutoff frequency}

\mynote[Mon Feb 18 17:15:24 MST 2019]{%
I assume the following is very similar to the above text, but don't feel like
looking at it closely right now\ldots}

The cutoff frequency of a medium is determined by local macro parameters,
such as pressure and density. Waves that oscillate with a frequency higher
than this will propagate through the medium, whereas waves with a frequency
lower than the cutoff will shock and dissipate. Acoustic waves obey the
Klein Gordon equation:
\begin{equation}
    \frac{\partial^{2}u}{\partial t^{2}} - \left( c_{s}^{2} \right)
    \frac{\partial^{2}u}{\partial z^{2}} + \Omega^{2}u = 0
\end{equation}
where $\Omega = c_{s}/2H_{P}$ is the cutoff frequency
($c_{s}$ is the local sound speed and
$H_{P}$ = $ kT / \mu g $ is the pressure scale height),
$t$ is time, $z$ is height
above the photosphere, and $u$ is the velocity amplitude.
The solution for $u$ is of the form:
\begin{equation}
    u = u_{0}e^{i\left( k_{z}z - \omega t \right)}
\end{equation}
where $\omega = 2 \pi \nu_{ac}$ is the oscillation frequency of the wave
in radians s$^{-1}$ ($\nu_{ac}$ = acoustic wave frequency in Hz),
and $k_{z} = 2\pi/\lambda$
is the wavenumber in the $z$-direction.
Using this solution yields:
\begin{equation}
    c_{s}^{2} k^{2} = \omega^{2} - \Omega^{2}
\end{equation}
If $\omega < \Omega$, then $k^{2} < 0$ and $k$ is imaginary.
In this case, the wave cannot propagate and standing oscillations are
observed.
\begin{align}
    2\pi\nu_{ac} &= \frac{c_{s}}{2H_{P}} = \Omega \\
    \nu_{ac} &= \frac{c_{s}}{4 \pi H_{P}}
    = \sqrt{ \frac{\gamma k T }{ 16 \pi^{2} H_{P}^{2} \mu m_{amu} } }
\end{align}
where $c_{s}$ is given by:
\begin{equation}
    c_{s}
    = \sqrt{ \frac{ \gamma P }{ \rho }}
    = \sqrt{ \frac{ \gamma k T }{ \mu m_{amu} }}\\
\end{equation}
Setting $\gamma = 5/3$, $\mu = 1.2$ amu, $H_{P} = 150$ km,
and $ T = T_{min} \approx 4200$~K,
the acoustic cutoff frequency $\nu_{ac}$ comes out to roughly
5.5 mHz, which corresponds to $\sim$3 minutes.


%% ???
%The non-isothermal region where a compressive acoustic disturbance
%is propagating along the magnetic field in direction $\hat{k}$.
%Umbral $\vec{B} \overrightarrow{B}$ aligned with
%$\vec{g}$, so $H_{||} \rightarrow$ `true'\mynote{??} $H_{P}$.
%\[
%    k_{||}^{2} = \frac{\omega^{2}}{c^{2}} - \frac{1}{4H_{||}^{2}}
%    \]
%where $\omega = 2\pi\nu$ and $H = H_{P}$, the pressure scale height.
%(Cally 2001; Crouch \& Cally 2003).
%Waves propagate if $k_{||}^{2} \geq 0 ( \omega \geq c/2H_{||} )$,
%threshold = acoustic cutoff frequency.
%$\Omega = \Omega(z)$ = max value of 5.2 - 5.5 mHz
%(3.21 - 3.03 minutes) with $T_{min}$\mynote{
%    (Fig 2. of Cally et al. 1994)}
%3-minute oscillations at the high frequency tail of photospheric
%oscillations. Conservation of wave energy: amp of velocity osc.
%$\propto 1/\sqrt{\rho}$ for waves with frequency greater than the
%cutoff. For freq less than cutoff, ``osc do not give rise to prop.
%waves''. Dispersion relation determines decline in velocity predicted
%by the \emph{imaginary} part of the wave number.

The numerical value of the cutoff
frequency depends on physical parameters of the ambient environment,
such as composition, temperature, and pressure scale height.
% ---}


% Wed Feb 13 03:58:54 MST 2019
%   Moved "Physics background" to appendix.tex.


%% Flares {---
\newpage

\section{Flares}

\mynote{Sun Nov 18 18:17:21 MST 2018}{
    Do charged particles travel
    along magnetic field lines via the Lorentz force?
    Is particle beam composed of both electrons and ions, or just electrons?
    Are ions only visible in gamma rays? Do they hit the chromosphere?
}

\paragraph{Rocket analogy}
The episodic, rapid release of magnetic energy in the solar atmosphere
associated with flares, CMEs, and SEP events has the potential to wipe us out.
Thus the field of space weather emerged and is devoted to understanding and
preparing for these events. Much like a rocket booster, this outward burst of
energy needs something to push against, and the same acceleration occurs in the
opposite direction: down toward the lower layers of the atmosphere. Most
research has focused on the rocket: the energy and material released outward
into space. However, it is impossible to understand how the rocket works
without studying what it left behind, how it got there, and how the energy may
have been converted through the various stages to its final form.

\subsection{Flare evolution}

Need to know general evolution of a flare, particularly the role of the
chromosphere as we currently understand it.
This is necessary in order
to interpret the results from before, during, and after a flare;
provides some context for what is physically happening.

An understanding of the physical processes that take place
during solar flares
\mynote[18 October 2018]{
    (when they happen, where they happen, the various structures involved,
    static physical conditions such as temperature, density, and composition,
    how much and why those values then change, etc.)}
is crucial if one desires to characterize the role played by the
relatively cool, thin atmospheric layer known as the chromosphere.
These stages, as described by the standard model,
provide vital context for the questions proposed, and then
investigated for this dissertation,
particularly the temporal behavior
of the chromospheric 3-minute oscillations before, during, and after flares.

%Figure~\ref{fig:flare_emission} shows the typical pattern of several bandpasses throughout the duration of a flare.
The development and evolution
of the various dynamical phenomena that ultimately
%lead to the burst of emission known as a solar flare
result in a solar flare
takes place throughout the solar atmosphere
in three primary stages:
precursor, impulsive, and gradual.
Each of these are characterized by specific observations that are interpreted
in the context of energy storage, release, and dissipation.
The official time at which the event transitions from one stage to the next
is determined by the SXR emission from the \textit{GOES} satellite.
GOES has two photometers that observe wavelength ranges
1-8\AA{} (1.5-10 keV) and 0.5-4\AA{}.

\mynote[18 October 2018]{
    Energy range for 0.5-4\AA{}? Need source!}

%(similar to the X-ray curve in Figure~\ref{fig:flare_emission})
These points in time are
characterized by the initial increase, peak, and final return
to background levels of the observed SXR flux.
The times at which these features are observed in the
lightcurves of other wavelength regimes in the EM spectrum
tend to deviate from the \textit{GOES} times
since different processes in the course of flare development
produce different types of emission during different stages.

Both the SXR and H$\alpha$ curves are observed to peak twice.
a sharp increase at the onset (the impulsive phase),
followed by a slower, smoother climb and subsequent decay (the gradual phase).
Each phase is characterized by specific observations that are interpreted in
the context of energy storage, release, conversion, and dissipation. These
stages are important in understanding the role of the chromosphere, and provide
the basis for the proposed investigation into the temporal localization of the
chromospheric 3-minute oscillations.

Our current understanding of flares involves their development and evolution
throughout the entire solar atmosphere. This is broken down into three main
phases: the precursor, impulsive, and decay phases. Each of these are
characterized by specific observations that are interpreted in the context of
energy storage, release, and dissipation. These stages are important in
understanding the role of the chromosphere, and will provide the basis for the
proposed investigation into the temporal of the excitation of the chromospheric
3-minute oscillations.

\textbf{Pre-flare: the zero-th stage}\quad
According to the standard flare model \mynote{(source!)},
the phenomenon we observe as a flare is the result of the rapid
release of magnetic energy after
days of slowing building to critical levels in the corona.

\cite{Woods2017} investigated the pre-flare stages of
the March 29 flare in 2014
with IRIS and \textit{Hinode}/EIS.
\mynote[18 October 2018]{\ldots And?}

\mynote[18 October 2018]{
    The following cute story was copied from Evernote, though
    probably came from proposal originally:}

Our journey begins during the ``pre-flare'' stage, during which magnetic energy
builds to critical levels.
This takes place over timescales from several days up to a few weeks.
\cite{Woods2017}
conducted a study to constrain the trigger mechanism
of flares, for which there are several possibilities. They found plasma flows
with velocities on the order of several hundred km s$^{-1}$.
This is a key time period for space weather prediction.


\mynote[Thu Oct 18 14:58:21 EDT 2018]{
    Testing \texttt{:r! date} to insert current date. Works!
    Although I'd prefer a different format\ldots}


\textbf{Precursor: the first stage}\quad
There are several possible mechanisms that could be responsible for triggering
the release of the stored magnetic energy (see Woods et al. (2017) for an
example).

The precursor phase, sometimes known as the pre-eruptive stage,
begins when the release of magnetic energy is triggered.
This stage typically lasts $\sim$3 minutes.
The eruption starts slowly, as
the energy leaks slowly and the
highly sheared magnetic field loses
stability and begins to rearrange itself.
The magnetic
structure surrounding prominence evolves through series of force-free
equilibria until it loses equilibrium and goes unstable.
The plasma around the magnetic reconnection site heats up,
and is observed as a
slow increase in soft X-ray (SXR) and UV emission from active regions.

A sudden increases in speed occurs,
most likely because the onset of magnetic
reconnection releases field lines that were holding the prominence down,
and cut it loose, hence the
``activation of the prominence'' (T-H and Emslie book).


This onset is important (e.g. SXR and UV before ``cataclysmic onset of impulsive
phase''). The magnetic field becomes unstable, starts to readjust. The emergence
of new flux is now a thing. Prominence activation (Martin and Ramsey 1972) and
heating of plasma (Poland et al. 1982; Hernandez et al. 1986; Klein et al.
1987; Machado et al. 1988a\ldots from T-H \& Emslie chapter 6?).

\textbf{Impulsive: the second stage}\quad

Once the magnetic configuration loses stability, an eruption occurs and stage
two, the impulsive stage, begins. This stage only lasts for a minute or less.
Magnetic reconnection begins, whose rapid onset is thought to be the cause of
the sudden drop in evolutionary timescales from the precursor to the impulsive
stage. There is a sudden release of energy that is then converted into various
forms, including kinetic energy of accelerating particles, thermal energy that
heats the plasma, bulk acceleration of fluid, and enhanced radiation fields.
Intense radio, HXR, and gamma rays are emitted in the form of intense, rapidly
fluctuating bursts. The jump in emission after it increased slowly during the
precursor phase is thought to be caused by the rapid onset of MR. This process
releases the prominence previously held down by the magnetic field. The
prominence is the matter that is ejected outward in the form of a CME.
\cite{Priest2017}
distinguish two phases of magnetic reconnection. The first
takes place during the impulsive phase, along the loop arcade, and is called
``zipper reconnection''.

Once the magnetic configuration loses stability, an eruption occurs
and stage two, the \textbf{impulsive} stage, begins. The onset of
magnetic reconnection releases field lines previously holding the
prominence down \citep{Priest2017}, and energy starts to be released
at a much faster rate.
Some of this energy is converted to kinetic energy of
non-thermal particles, which are then accelerated down until they hit the
chromosphere and inject the plasma with an impulsive burst of high energy
(E $\gtrsim$ 10 keV).
Energy is then lost to radiation in the form of
intense, rapidly fluctuating bursts of radio, hard X-ray (HXR),
and gamma rays.
The HXR Bremsstrahlung radiation at the loop footpoints is indicative
of energization of non-thermal particles, and this is
where the initial heating of the chromospheric plasma takes place
\citep{Hoyng1981, Fletcher2013b}.

This entire process takes less than a minute.
Many of the processes that take place during the impulsive phase
are inferred from the response of the surrounding plasma,
but cannot be observed directly due to the extremely short timescales
over which they occur.

\myheading{NT particle acceleration}
Magnetic energy released via magnetic reconnection is converted to
kinetic energy of charged particles that are accelerated downward
with energies exceeding 1 MeV.

After being injected into the denser plasma of the chromosphere,
the interaction between NT electrons and the ambient particles causes
the emission of Bremsstrahlung HXR radiation.
This emission appears as ribbon-like features along the footpoints
of the reconnected loops as they contract down into the lower atmosphere.

Chromospheric evaporation also occurs during this time.
After the plasma is heated, some of the excess energy is shed by
radiation or thermal conduction, but if these mechanisms are not enough
to return the plasma to thermal equilibrium, the high gas pressure causes
the plasma to rise up into the post-flare loops.
Explosive evaporation occurs when the upward force is accompanied by
an equal and opposite force in the opposite direction.
This is revealed by the presence of redshifts in emission lines in
addition to blueshifts. The downward movement of plasma is then
referred to as chromospheric ``condensation''.
Velocities of plasma movement via explosive evaporation can be
$\gtrsim$100 km s$^{-1}$, and $F_{\ell} > 10^{10}$ erg s$^{-1}$ cm$^{-2}$.

Gentle evaporation refers to
when the continuous, lower energy input from thermal conduction
pushes plasma up the loops (evaporation), but not down (no condensation),
so only blueshifts are observed.
This process continues as long as a temperature gradient exists.
Velocities of plasma movement via gentle evaporation are
$\gtrsim$10 km s$^{-1}$, and $F_{\ell} < 10^{10}$ erg s$^{-1}$ cm$^{-2}$.

Explosive can turn into gentle if conductive flux out of explosively heated
plasma becomes comparable to energy flux in the electron beam:
\[
    F_{cond} \approx F_{NTe^{-}s}; \quad T \gtrsim 1 - \gtrsim 10s
\]



\textbf{Gradual: the third and final stage}\quad
Once all the magnetic energy has been released and dissipated,
non-thermal particles are no longer being accelerated downward,
to drive chromospheric evaporation, and
the gradual stage has begun.
There remains a strong temperature gradient between the hot flare loops and
the surrounding, cooler chromospheric plasma.
Some of the energy that
was not lost to radiation continues to heat the plasma via thermal
conduction, which continues to drive evaporation of plasma upward
through post-flare loops \citep{Battaglia2015}, though more slowly
than during the impulsive phase. This final phase is characterized by
a gradual buildup and decay of SXR emission from evaporation of hot
post-flare loops. The loops continue to evolve by cooling and
draining, and eventually retreat back down and become visible in
H$\alpha$ \citep{Hudson2007}. This emission reveals the morphology of
the so-called ``flare ribbons''. These are the footpoints of newly
connected field lines, and are one of the primary observable features
of flares. This stage typically lasts $\sim$30 minutes.

\subsubsection{Quasi-periodic pulsations (QPPs)}

\mynote[18 October 2018]{
    So I've been reluctant to discuss the following issue because I don't
    actually know much about it, just that a few authors have quoted these words:
    ``There have been few studies of QPPs in thermal emission.''
    However, just found a paper from {} that explains this, which is helpful
    not only because it provides me with an supplementary description,
    but it's another source, so I'm not just quoting Milligan's paper over
    and over.
}

\mynote[18 October 2018]{
    Why do we care about QPPs? This needs to be clear to the
    reader/listener right away.
}

Thermal emission from the lower atmosphere during flares
provides a potentially useful
way to probe the chromospheric dynamics and extract
information about the transportation and conversion of energy in this
region.

\mynote{Brief description of what QPPs are}

Embedded within the global shape of flare emission
are temporal fluctuations known as quasi-periodic pulsations (QPPs),
with periods ranging from $\sim$1 second to several minutes.
They have been observed during all three flare phases
in all wavelength bands.
(See \cite{VanDoorsselaere2016} for a recent review.)


Since they are clearly connected to the flare process,
QPPs are considered to be an intrinsic
property of flares, thereby providing a direct probe into the
reconnection site and the affected regions beyond.
QPPs are ``directly linked to the properties of the flare reocnnection
region and flare acceleration sites'' \citep{Inglis2015},
so a lot can be learned about flares by studying these.


\mynote[18 February 2019]{Copied from E-note. I think this was in my proposal,
but probably took it out.}
There is no strict definition for this kind of pattern, but it is generally
described, as quoted from \cite{Inglis2015}, as
``variations in the flux from a flare\ldots as a function of time,
which appear to include periodic components with characteristic timescales
ranging from one second up to several minutes.''


Constraining the physical mechanism(s) responsible for the
generation of QPPs is an ongoing investigation.
There are two leading theories that describe them as signatures of one of two
things:
\begin{enumerate}
    \item the energy release process (MR),
        i.e. the rate at which magnetic field lines are reconnected
        and accelerate non-thermal particles.
        This is often referred to as a ``load/unload'' mechanism.
    \item MHD oscillations induced in the ambient plasma by
        \begin{itemize}
            \item the magnetic reconnection itself, which would give the resulting
                QPPs the same ``built-in'' periodicity as the reconnection rate.
            \item the same trigger responsible for inducing magnetic reconnection
                in the first place \citep{Nakariakov2009}.
        \end{itemize}
        Theoretically, MHD oscillations would be induced in post-flare loops,
        due to the nature of these types of oscillations in the vicinity of a
        magnetic field.
\end{enumerate}

Each of the above mechanisms can produce similar observational signatures,
adding to the difficulty in identifying which is the cause, if not both
\citep{Brosius2016}.


Their small periodicities can also make them difficult to extract from the main
lightcurve \citep{VanDoorsselaere2016}.
\mynote[Mon Feb 18 03:15-ish MST 2019]{%
Modulation depth? I think that has more to do with the amplitude than
the period. Maybe it's because flare timescales are close to QPP periods,
so everything gets mushed together and it's more difficult to extract any one
individual signal.}
Sometimes these periodicies can be on the order of a few seconds, making
them difficult to probe with current available instrumental cadence.
\mynote[Mon Feb 18 03:37:31 MST 2019]{Oh.}


% Observations of QPPs in thermal emission during flares
Currently, there are few studies of QPPs in thermal emission.
Non-thermal emission in HXR and microwave bands has a higher modulation depth,
particularly during the impulsive phase of flares, when the global emission level
increases substantially
\citep{Hayes2016}.


\citep{Brosius2016} studied emission from the \ion{C}{1} line, and found
a period of $\sim$170 seconds. They
attributed this periodicity to the rate of injection of
non-thermal electrons.
Further investigation of QPPs in thermal emission was carried out by
\cite{Milligan2017}, who found enhancements of the 3-minute oscillations
in Lyman continuum and Lyman-$\alpha$ line emission.
However, no signature of a 3-minute oscillation was found in
the X-ray data generated by the flare, indicating
that the chromospheric response was not a
reflection of the rate of energy input.
These findings supported the idea that the chromosphere naturally
responds at its cutoff frequency, regardless of the periodicity of the
energy injection rate.
This result did not support the conclusion from
\cite{Sych2009}, who
proposed that the 3-minute waves leaked from sunspots into the upper
atmosphere by propagating along magnetic field lines, and triggered
the magnetic reconnection and subsequent energy release and particle
acceleration.


To investigate the transfer of energy during flares, \cite{Monsue2016}
conducted a pilot study of the spatial and temporal flare response of acoustic
oscillations in H$\alpha$ emission.
They were able to preserve both the two-dimensional spatial information
and the temporal information in the original data
with the power and frequency of the output of the Fourier transforms
by incorporating a technique called ``frequency-filtered amplitude movies''
from \cite{Jackiewicz2013}.
They reported a suppression of frequencies
between 1 and 8 mHz during the main phase, and an enhancement of frequencies
between 1 and 2 mHz before and after the flare. They suggested that the
suppression of lower frequencies could be evidence of the conversion of energy
from acoustic to thermal, and that the pre-flare enhancement may have been
indicative of an instability in the chromosphere.
In the conclusion of this study, they encouraged
the further investigation of
earlier time frames before the flare precursor, along with the
inclusion of additional cases to increase the statistical significance of the
findings.

% Temporal dependence of QPPs (flare phases). Science question - Time!!!
% Move this to methodology?
\cite{Hayes2016} examined oscillations during
both the impulsive and decay phase of an X-class flare, and considered
a combination of possible mechanisms for producing QPPs, where the
impulsive phase involved rapid injections from non-thermal particle
acceleration and subsequent heating of the surrounding plasma, while
thermal signatures persisted throughout the decay phase.
Propagating slow magnetoacoustic waves, such as the chromospheric
3-minute oscillations, show characteristic quasi-periodic patterns.
Several recent papers discuss the relationship between QPPs and
chromospheric oscillations.



% Move this to methodology?
The observations were acquired using
Lyman continuum data from \textit{SDO}/EVE,
Lyman-$\alpha$ line emission from \textit{GOES}/EUVS,
1600\AA{} and 1700\AA{} continuum from \textit{SDO}/AIA, and
HXR emission from \textit{RHESSI}.
EVE observes the sun as a star, with disk-integrated data.
Figure~\ref{fig:Milligan2017} shows their
results from the analysis on AIA data.


\subsection{Flare classes}

Flare are classified according to their peak
soft x-ray flux (1-8\AA{}), as
measured at Earth from the \textit{GOES} satellite.
The classes are as follows:
\begin{description}
    \item [A] 10$^{−4}$ < F < 10$^{−3}$
    \item [B] 10$^{−3}$ < F < 10$^{−2}$
    \item [C] 10$^{−2}$ < F < 10$^{−1}$
    \item [M] 10$^{−1}$ < F < 1
    \item [X] 1 < F < 10
\end{description}
where F = flux (erg s$^{-1}$ cm$^{−2}$).
These are further subdivided from 1 through 9.
This is a linear relationship so, for example, an X2 flare is twice as powerful
as an X1 flare.

Typical flare temperatures are 10$^{7}$ K.
Emission lines at these temperatures are blueshifted, while those at
chromospheric or TR temperatures are redshifted
\citep{Brosius2016}.
Typical energies are around 10$^{27}−10^{32}$ erg.
The energy for the largest flares (i.e. of class > X10) has
been capped at a few $\times 10^{32}$ erg.

% ---}
