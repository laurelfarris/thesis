\section{Introduction}

\paragraph{``Overview'' from proposal:}

Embedded in the main lightcurve of a solar flare are low amplitude, high
frequency oscillations called quasi-periodic pulsations (QPPs). The
chromosphere persistently oscillates at its acoustic cutoff frequency of 3
minutes, in both quiet and active regions. A recent study of QPPs revealed an
enhancement in the 3-minute oscillatory power in thermal emission during the
main phase of a flare \citep{Milligan2017}. Here I propose to probe the
oscillatory response of the chromosphere to flares in greater detail by
addressing the following science questions:
\begin{enumerate}
    \item In which part(s) of the active region does the enhancement of the
        3-minute power occur? Is it global or local?
    \item When does the enhancement of the 3-minute power occur, relative to
        the phases of flare development and evolution? At what rate does the
        oscillatory power increase, and how long does the enhancement last?
    \item How do chromospheric oscillations in velocity compare to oscillations
        in intensity during flares?
    \item Do flares of different sizes show the same oscillatory behavior in
        the chromosphere?
\end{enumerate}

The current hypothesis is that the oscillations reflect the natural
response of the chromosphere to a disturbance, rather than a reflection of
the periodicity of the disturbance itself.
The questions above will be addressed by a dissertation project,
to be carried out in three phases:
\begin{enumerate}%[label=\Roman*. ]
    \item \textbf{Reproduce and augment the results from \cite{Milligan2017} by
        studying the spatial and temporal localization of the 3-minute power
        during the 2011 February 15 flare.}
        Data from the Helioseismic and Magnetic Imager (HMI) and from the
        Atmospheric Imaging Assembly (AIA), two instruments on board the
        \textit{Solar Dynamics Observatory (SDO)},
        will be analyzed, using the techniques of Fast Fourier Transforms (FFTs)
        and wavelet analysis.
        The preliminary results will then be expanded by
        isolating distinct components of the active regions to constrain the
        spatial variation and extent of the oscillatory reponse. Finally, a
        longer data set will be used to include time before and after the
        flare.
    \item \textbf{Expand the methodology to include analysis of spectroscopic data from
        the \textit{Interface Region Imaging Spectrograph (IRIS)}.}
        The spatial
        and spectroscopic capabilities of \textit{IRIS} were designed to probe
        the chromosphere and transition region and address questions similar to
        those put forth here. Previous studies have found chromospheric
        oscillations to be more prominent in velocity than in intensity, and
        the inclusion of spectroscopic data from \textit{IRIS} in Phase II will
        benefit this study by removing some of the ambiguities present in
        intensity images (such as those with multiple spectral lines in a
        single bandpass), and providing information from multiple heights in
        the lower atmosphere.
    \item \textbf{Apply the techniques from the previous two phases to multiple flares
        with a range of sizes.}
        The methodology developed in Phases I and II
        will be used to study multiple flares of various sizes in Phase III.
        The primary goal of this phase is to go beyond analysis of a single
        event and create a more general contribution to the flare model as a
        whole.
\end{enumerate}

%The results from the time period before each flare begins will be
%of particular interest, with the opportunity to contribute to models for space
%weather prediction.
%
%
%During the course of this dissertation I will learn to use spectroscopy and
%imaging together, and continue to develop useful scientific and computational
%skills for which there are many applications beyond this project alone.
%
%the enhancement
%of the 3-minute oscillations appears to be concentrated in the center of
%sunspot umbrae, though not in all sunspots contained in the active region.

%\S\ref{background}
%provides background information on flares and the chromosphere in general,
%followed by a description of the relevant physics in \S\ref{physics}.
%Plans for carrying out the steps listed above, along with some preliminary
%results and predicted outcomes, are described in
%\S\ref{project}.
%A dissertation timeline is presented in
%\S\ref{timeline}.

\clearpage

\begin{framed}
\paragraph{single\_page\_freewriting.gdoc}

    \mynote{Copied from proposal, probably needs some revising.}

The overarching theme of my research is the investigation of the role of the
chromosphere in flares. The field of solar physics and space weather is unique
in that is has a dirct impact on humans and our way of life.
\begin{enumerate}
    \item How does understanding details of the flare process help with space
        weather prediction?
    \item What is some common work done to contribute to learning about flares?
        Most work has focused on the corona
    \item Importance of chrom in general (maybe\ldots seems like this just ends
        up being a bullet underneath flares…)
\end{enumerate}

\begin{itemize}
    \item Importance of chromosphere's role in flares:
        \begin{itemize}
            \item Light we see is emitted from this layer
            \item Why are there gaps? Chromosphere can be hard to study\ldots
        \end{itemize}
\end{itemize}

\mynote{From current paper (09 October 2018):
Flare dynamics can be probed via
quasi-periodic pulsations (QPPs), low-amplitude variations
that have been observed in flare emission
during all phases and across all wavelength
regimes, with periods between 1 and 10 minutes.
The cause of long-period
($\gtrsim$ 1 minute) QPPs
is generally thought to be one of two possible mechanisms.
One is that they reflect the periodic buildup and release of magnetic
energy through cycles of magnetic reconnection.
The other is that they are triggered externally by the surrounding plasma.
Many studies of QPPs have concentrated
on non-thermal emission, but there are
few reports of QPPs in thermal emission from the chromosphere
associated with flares.}

Flare emission contains QPPs, small scale oscillations within the main light
curve. Most flare studies have been of non-thermal emission at the extreme ends
of the spectrum (radio \& x-ray). Chromospheric oscillations are revealed in
thermal emission, of which not many studies have been carries out. A recent
study of a solar flare (Milligan et al. 2017) revealed an enhancement of the
3-minute oscillatory power in thermal emission from the chromosphere, but not
in x-ray emission. This indicates that the 3mOs were not excited according to
the rate of energy injection, but rather that this was a response of the
chromosphere at its own natural frequency, or cutoff frequency.
\begin{enumerate}
    \item Motivation:
        \begin{itemize}
            \item ``Few reports of QPPs in thermal emission in response to flare
                energization''.
            \item 3mOs do not depend on rate of energy injection, but may
                transport a significant amount of mechanical energy, which
                needs to be included in the flare energy budget.
        \end{itemize}
    \item Methods:
        \begin{itemize}
            \item Lyman alpha emission
            \item X-rays
            \item AIA 1600, 1700
        \end{itemize}
    \item Main results/conclusions:
        \begin{itemize}
            \item 3m Enhancement in thermal emission, but not X-rays\\
                $\rightarrow$ Not signature of flare energy, but natural
                response of chromosphere
            \item Flare injected energy, causing chromosphere to oscillate at
                the cutoff frequency
        \end{itemize}
\end{enumerate}

For my project, I will probe the spatial dependence and temporal patterns of
the oscillatory behavior of the chromosphere, using both images and spectra.
I will then apply the developed methodology to multiple flares to compare
results for flares of various sizes, contributing to the general field of
solar flares and space weather prediction.

\mynote[Thu Oct 18 16:11:32 EDT 2018]{
    From Monsue2016:
    ``\ldots H$\alpha$ observations of chromospheric
    oscillations in the p-mode band can provide information about the physical
    processes occurring in flaring regions. In particular,
    \textbf{variations in the
    oscillatory power as a function of frequency, spatial position, and time can be
    used to probe energy transport at different heights within a flare.''}
    (emphasis added).
}

\end{framed}
