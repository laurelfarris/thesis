%% Cutoff frequency {---
\subsection{The acoustic cutoff frequency}
\subsubsection{Derivation of the cutoff frequency}

Acoustic waves obey the Klein Gordon equation:
\begin{equation}
    \frac{\partial^{2}u}{\partial t^{2}}
    - \left( c_{s}^{2} \right)
    \frac{\partial^{2}u}{\partial z^{2}}
    + \Omega^{2}u
    = 0
\end{equation}

where
$u$ = velocity amplitude,
$t$ = time,
$c_{s}$ = local sound speed,
$z$ = height above the photosphere, and
where $\Omega = 2\pi\nu_{ac}$
is the cutoff frequency in radians per second
($\nu_{ac}$ = cutoff frequency in oscillations per second),


The solution for $u$ is of the form:

\begin{equation}
    u = u_{0}e^{i\left( k_{z}z - \omega t \right)}
\end{equation}

where $\omega = 2 \pi f$ is the oscillation frequency of the wave
in radians s$^{-1}$ ($f$ = frequency in Hz), and $k_{z} = 2\pi/\lambda$
is the wavenumber in the $z$-direction.
Using this solution, we eventually get:
\begin{equation}
    c_{s}^{2} k^{2} = \omega^{2} - \Omega^{2}
\end{equation}
If $\omega < \Omega$, then $k^{2} < 0$ and $k$ is imaginary.
In this case, the wave cannot propagate and standing oscillations are
observed.

\begin{equation}
    \Omega = \frac{c_{s}}{2H}
\end{equation}
$H$ = scale height.

\begin{align}
    \Omega &= \frac{c_{s}}{2H} = 2\pi\nu_{ac}\\
    \nu_{ac} &= \frac{c_{s}}{4 \pi H}\\
    \nu_{ac} &= \sqrt{ \frac{\gamma k T }{ 16 \pi^{2} H^{2} \mu m_{amu} } }
\end{align}

where $c_{s}$ is given by:
\begin{equation}
    c_{s} = \sqrt{ \frac{ \gamma P }{ \rho }}
    = \sqrt{ \frac{ \gamma k T }{ \mu m_{amu} }}
\end{equation}

Setting $\gamma$ to $5/3$, $\mu = 1.2$, $H = 150$ km,
and $ T = 4200 $ K (the approximate temperature minimum),
the acoustic cutoff frequency $\nu_{ac}$ comes out to roughly
5.6 mHz, the frequency at which the 3-minute chromospheric
period oscillates.


%% ???
%The non-isothermal region where a compressive acoustic disturbance
%is propagating along the magnetic field in direction $\hat{k}$.
%Umbral $\vec{B} \overrightarrow{B}$ aligned with
%$\vec{g}$, so $H_{||} \rightarrow$ `true'\mynote{??} $H_{P}$.
%\[
%    k_{||}^{2} = \frac{\omega^{2}}{c^{2}} - \frac{1}{4H_{||}^{2}}
%    \]
%where $\omega = 2\pi\nu$ and $H = H_{P}$, the pressure scale height.
%(Cally 2001; Crouch \& Cally 2003).
%Waves propagate if $k_{||}^{2} \geq 0 ( \omega \geq c/2H_{||} )$,
%threshold = acoustic cutoff frequency.
%$\Omega = \Omega(z)$ = max value of 5.2 - 5.5 mHz
%(3.21 - 3.03 minutes) with $T_{min}$\mynote{
%    (Fig 2. of Cally et al. 1994)}
%3-minute oscillations at the high frequency tail of photospheric
%oscillations. Conservation of wave energy: amp of velocity osc.
%$\propto 1/\sqrt{\rho}$ for waves with frequency greater than the
%cutoff. For freq less than cutoff, ``osc do not give rise to prop.
%waves''. Dispersion relation determines decline in velocity predicted
%by the \emph{imaginary} part of the wave number.

%%\subsection{Physical background for the cutoff frequency}

\mynote[Mon Feb 18 17:15:24 MST 2019]{%
I assume the following is very similar to the above text, but don't feel like
looking at it closely right now\ldots}

The cutoff frequency of a medium is determined by local macro parameters,
such as pressure and density. Waves that oscillate with a frequency higher
than this will propagate through the medium, whereas waves with a frequency
lower than the cutoff will shock and dissipate. Acoustic waves obey the
Klein Gordon equation:
\begin{equation}
    \frac{\partial^{2}u}{\partial t^{2}} - \left( c_{s}^{2} \right)
    \frac{\partial^{2}u}{\partial z^{2}} + \Omega^{2}u = 0
\end{equation}
where $\Omega = c_{s}/2H_{P}$ is the cutoff frequency
($c_{s}$ is the local sound speed and
$H_{P}$ = $ kT / \mu g $ is the pressure scale height),
$t$ is time, $z$ is height
above the photosphere, and $u$ is the velocity amplitude.
The solution for $u$ is of the form:
\begin{equation}
    u = u_{0}e^{i\left( k_{z}z - \omega t \right)}
\end{equation}
where $\omega = 2 \pi \nu_{ac}$ is the oscillation frequency of the wave
in radians s$^{-1}$ ($\nu_{ac}$ = acoustic wave frequency in Hz),
and $k_{z} = 2\pi/\lambda$
is the wavenumber in the $z$-direction.
Using this solution yields:
\begin{equation}
    c_{s}^{2} k^{2} = \omega^{2} - \Omega^{2}
\end{equation}
If $\omega < \Omega$, then $k^{2} < 0$ and $k$ is imaginary.
In this case, the wave cannot propagate and standing oscillations are
observed.
\begin{align}
    2\pi\nu_{ac} &= \frac{c_{s}}{2H_{P}} = \Omega \\
    \nu_{ac} &= \frac{c_{s}}{4 \pi H_{P}}
    = \sqrt{ \frac{\gamma k T }{ 16 \pi^{2} H_{P}^{2} \mu m_{amu} } }
\end{align}
where $c_{s}$ is given by:
\begin{equation}
    c_{s}
    = \sqrt{ \frac{ \gamma P }{ \rho }}
    = \sqrt{ \frac{ \gamma k T }{ \mu m_{amu} }}\\
\end{equation}
Setting $\gamma = 5/3$, $\mu = 1.2$ amu, $H_{P} = 150$ km,
and $ T = T_{min} \approx 4200$~K,
the acoustic cutoff frequency $\nu_{ac}$ comes out to roughly
5.5 mHz, which corresponds to $\sim$3 minutes.


%% ???
%The non-isothermal region where a compressive acoustic disturbance
%is propagating along the magnetic field in direction $\hat{k}$.
%Umbral $\vec{B} \overrightarrow{B}$ aligned with
%$\vec{g}$, so $H_{||} \rightarrow$ `true'\mynote{??} $H_{P}$.
%\[
%    k_{||}^{2} = \frac{\omega^{2}}{c^{2}} - \frac{1}{4H_{||}^{2}}
%    \]
%where $\omega = 2\pi\nu$ and $H = H_{P}$, the pressure scale height.
%(Cally 2001; Crouch \& Cally 2003).
%Waves propagate if $k_{||}^{2} \geq 0 ( \omega \geq c/2H_{||} )$,
%threshold = acoustic cutoff frequency.
%$\Omega = \Omega(z)$ = max value of 5.2 - 5.5 mHz
%(3.21 - 3.03 minutes) with $T_{min}$\mynote{
%    (Fig 2. of Cally et al. 1994)}
%3-minute oscillations at the high frequency tail of photospheric
%oscillations. Conservation of wave energy: amp of velocity osc.
%$\propto 1/\sqrt{\rho}$ for waves with frequency greater than the
%cutoff. For freq less than cutoff, ``osc do not give rise to prop.
%waves''. Dispersion relation determines decline in velocity predicted
%by the \emph{imaginary} part of the wave number.

The numerical value of the cutoff
frequency depends on physical parameters of the ambient environment,
such as composition, temperature, and pressure scale height.
% ---}
