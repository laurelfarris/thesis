%% Cutoff frequency {---
\newpage
\subsection{The acoustic cutoff frequency}
In the interior of the sun, acoustic waves have frequencies higher than the
cutoff, so it is of less importance there. In the solar atmosphere, however,
the cutoff plays an important role in the behavior of waves and oscillatory
phenomena.
The cutoff peaks in the chromosphere, at a value of about 5.5 mHz,
or roughly three minutes. This is also the dominant frequency of the
oscillations observed everywhere at all times in the chromosphere.

If waves with frequencies lower than the cutoff are unable to propagate
at a certain height, then one would expect to see a continuously high power
spectrum at frequencies higher than this, and then a drop to zero at
frequencies lower than this. However, what we see is a peak at the cutoff,
but the power drops again for higher frequencies.

\mynote{Where are the higher frequencies?}

This may be because waves at higher frequencies simply aren't generated
in the first place.

\mynote{If the cutoff frequency didn't exist, how would power at lower
frequencies compare to the 3-minute power? Higher or lower? Or do these
waves even exist to begin with?}

\mynote{What types of waves are generated in the atmosphere, and why?}

Figure~\\ref{fig:Stangalini2011} shows power maps from
\cite{Stangalini2011} illustrating the power distribution of the
3-minute oscillations at the photospheric Fe~6173~\AA{} line and the
chromospheric Ca~8542~\AA{} line. Also shown is the power distribution
of the 5-minute oscillations. The 5-minute period has been attributed
to the global pulsations of the sun due to pressure modes
(\textit{p}-modes) in the interior. The image shows the suppression of
the 5-minute period in the umbra of sunspots, the opposite of the
3-minute behavior in the chromosphere above sunspots. It is commonly
observed in the photosphere that the 5-minute power above sunspot
umbrae is \emph{reduced} compared to the quiet sun by a factor of 2-5,
though it is still the dominant frequency
\citep{Felipe2010, Bogdan2006}.
\cite{Reznikova2012} also observed lowering of the cutoff frequency
due to inclination of magnetic field lines.
The presence of the 5-minute oscillations in the diffuse magnetic regions
surrounding the umbra is likely caused by the decrease in effective
cutoff frequency due to the inclination of the magnetic field. Waves
that are generated in the lower atmosphere propagate upward until
their frequency no longer exceeds the local cutoff frequency and they
are reflected \citep{DeMoortel2000, Brynildsen1999}. The cutoff
frequency at the base of the chromosphere is $\sim$5.5 mHz, or roughly
3 minutes, which agrees with the acoustic cutoff period of a gas with
the composition and pressure scale height of the upper photosphere
\citep{Kalkofen1994}.

Waves with a frequency lower than the cutoff will:
\begin{itemize}
    \item be reflected back down
    \item shock and dissipate
    \item strongly damp out
\end{itemize}


\mynote{What is the `natural frequency', and what is the significance
of the chromosphere oscillating at its natural frequency
(as opposed to\ldots)? Milligan paper presented evidence supporting that
the chromosphere oscillates at its natural frequency in response to a
disturbance, but never stated why this is significant or the implications
for bigger picture science.}

\clearpage
\subsubsection{Derivation of the cutoff frequency}

The cutoff frequency defines the minimum frequency a wave must have in order to
propagate through a medium.
It is an intrinsic property of the medium,
determined by local macro parameters, such as pressure and density.
As these properties vary throughout the interior of the sun and its atmosphere,
the cutoff frequency itself is an indirect function of height.

\mynote{Plot of cutoff frequency with height goes here.}

Acoustic waves obey the Klein Gordon equation:
\begin{equation}
    \frac{\partial^{2}u}{\partial t^{2}}
    - \left( c_{s}^{2} \right)
    \frac{\partial^{2}u}{\partial z^{2}}
    + \Omega^{2}u
    = 0
\end{equation}

where
$u$ = velocity amplitude,
$t$ = time,
$c_{s}$ = local sound speed,
$z$ = height above the photosphere, and
where $\Omega = 2\pi\nu_{ac}$
is the cutoff frequency in radians per second
($\nu_{ac}$ = cutoff frequency in oscillations per second),


The solution for $u$ is of the form:

\begin{equation}
    u = u_{0}e^{i\left( k_{z}z - \omega t \right)}
\end{equation}

where $\omega = 2 \pi f$ is the oscillation frequency of the wave
in radians s$^{-1}$ ($f$ = frequency in Hz), and $k_{z} = 2\pi/\lambda$
is the wavenumber in the $z$-direction.
Using this solution, we eventually get:
\begin{equation}
    c_{s}^{2} k^{2} = \omega^{2} - \Omega^{2}
\end{equation}
If $\omega < \Omega$, then $k^{2} < 0$ and $k$ is imaginary.
In this case, the wave cannot propagate and standing oscillations are
observed.

\begin{equation}
    \Omega = \frac{c_{s}}{2H}
\end{equation}
$H$ = scale height.

\begin{align}
    \Omega &= \frac{c_{s}}{2H} = 2\pi\nu_{ac}\\
    \nu_{ac} &= \frac{c_{s}}{4 \pi H}\\
    \nu_{ac} &= \sqrt{ \frac{\gamma k T }{ 16 \pi^{2} H^{2} \mu m_{amu} } }
\end{align}

where $c_{s}$ is given by:
\begin{equation}
    c_{s} = \sqrt{ \frac{ \gamma P }{ \rho }}
    = \sqrt{ \frac{ \gamma k T }{ \mu m_{amu} }}
\end{equation}

Setting $\gamma$ to $5/3$, $\mu = 1.2$, $H = 150$ km,
and $ T = 4200 $ K (the approximate temperature minimum),
the acoustic cutoff frequency $\nu_{ac}$ comes out to roughly
5.6 mHz, the frequency at which the 3-minute chromospheric
period oscillates.


%% ???
%The non-isothermal region where a compressive acoustic disturbance
%is propagating along the magnetic field in direction $\hat{k}$.
%Umbral $\vec{B} \overrightarrow{B}$ aligned with
%$\vec{g}$, so $H_{||} \rightarrow$ `true'\mynote{??} $H_{P}$.
%\[
%    k_{||}^{2} = \frac{\omega^{2}}{c^{2}} - \frac{1}{4H_{||}^{2}}
%    \]
%where $\omega = 2\pi\nu$ and $H = H_{P}$, the pressure scale height.
%(Cally 2001; Crouch \& Cally 2003).
%Waves propagate if $k_{||}^{2} \geq 0 ( \omega \geq c/2H_{||} )$,
%threshold = acoustic cutoff frequency.
%$\Omega = \Omega(z)$ = max value of 5.2 - 5.5 mHz
%(3.21 - 3.03 minutes) with $T_{min}$\mynote{
%    (Fig 2. of Cally et al. 1994)}
%3-minute oscillations at the high frequency tail of photospheric
%oscillations. Conservation of wave energy: amp of velocity osc.
%$\propto 1/\sqrt{\rho}$ for waves with frequency greater than the
%cutoff. For freq less than cutoff, ``osc do not give rise to prop.
%waves''. Dispersion relation determines decline in velocity predicted
%by the \emph{imaginary} part of the wave number.

%%\subsection{Physical background for the cutoff frequency}

\mynote[Mon Feb 18 17:15:24 MST 2019]{%
I assume the following is very similar to the above text, but don't feel like
looking at it closely right now\ldots}

The cutoff frequency of a medium is determined by local macro parameters,
such as pressure and density. Waves that oscillate with a frequency higher
than this will propagate through the medium, whereas waves with a frequency
lower than the cutoff will shock and dissipate. Acoustic waves obey the
Klein Gordon equation:
\begin{equation}
    \frac{\partial^{2}u}{\partial t^{2}} - \left( c_{s}^{2} \right)
    \frac{\partial^{2}u}{\partial z^{2}} + \Omega^{2}u = 0
\end{equation}
where $\Omega = c_{s}/2H_{P}$ is the cutoff frequency
($c_{s}$ is the local sound speed and
$H_{P}$ = $ kT / \mu g $ is the pressure scale height),
$t$ is time, $z$ is height
above the photosphere, and $u$ is the velocity amplitude.
The solution for $u$ is of the form:
\begin{equation}
    u = u_{0}e^{i\left( k_{z}z - \omega t \right)}
\end{equation}
where $\omega = 2 \pi \nu_{ac}$ is the oscillation frequency of the wave
in radians s$^{-1}$ ($\nu_{ac}$ = acoustic wave frequency in Hz),
and $k_{z} = 2\pi/\lambda$
is the wavenumber in the $z$-direction.
Using this solution yields:
\begin{equation}
    c_{s}^{2} k^{2} = \omega^{2} - \Omega^{2}
\end{equation}
If $\omega < \Omega$, then $k^{2} < 0$ and $k$ is imaginary.
In this case, the wave cannot propagate and standing oscillations are
observed.
\begin{align}
    2\pi\nu_{ac} &= \frac{c_{s}}{2H_{P}} = \Omega \\
    \nu_{ac} &= \frac{c_{s}}{4 \pi H_{P}}
    = \sqrt{ \frac{\gamma k T }{ 16 \pi^{2} H_{P}^{2} \mu m_{amu} } }
\end{align}
where $c_{s}$ is given by:
\begin{equation}
    c_{s}
    = \sqrt{ \frac{ \gamma P }{ \rho }}
    = \sqrt{ \frac{ \gamma k T }{ \mu m_{amu} }}\\
\end{equation}
Setting $\gamma = 5/3$, $\mu = 1.2$ amu, $H_{P} = 150$ km,
and $ T = T_{min} \approx 4200$~K,
the acoustic cutoff frequency $\nu_{ac}$ comes out to roughly
5.5 mHz, which corresponds to $\sim$3 minutes.


%% ???
%The non-isothermal region where a compressive acoustic disturbance
%is propagating along the magnetic field in direction $\hat{k}$.
%Umbral $\vec{B} \overrightarrow{B}$ aligned with
%$\vec{g}$, so $H_{||} \rightarrow$ `true'\mynote{??} $H_{P}$.
%\[
%    k_{||}^{2} = \frac{\omega^{2}}{c^{2}} - \frac{1}{4H_{||}^{2}}
%    \]
%where $\omega = 2\pi\nu$ and $H = H_{P}$, the pressure scale height.
%(Cally 2001; Crouch \& Cally 2003).
%Waves propagate if $k_{||}^{2} \geq 0 ( \omega \geq c/2H_{||} )$,
%threshold = acoustic cutoff frequency.
%$\Omega = \Omega(z)$ = max value of 5.2 - 5.5 mHz
%(3.21 - 3.03 minutes) with $T_{min}$\mynote{
%    (Fig 2. of Cally et al. 1994)}
%3-minute oscillations at the high frequency tail of photospheric
%oscillations. Conservation of wave energy: amp of velocity osc.
%$\propto 1/\sqrt{\rho}$ for waves with frequency greater than the
%cutoff. For freq less than cutoff, ``osc do not give rise to prop.
%waves''. Dispersion relation determines decline in velocity predicted
%by the \emph{imaginary} part of the wave number.

The numerical value of the cutoff
frequency depends on physical parameters of the ambient environment,
such as composition, temperature, and pressure scale height.
% ---}
