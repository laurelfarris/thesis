\clearpage
\section{The solar chromosphere}

%% Chromosphere, in general {---



\myheading{The flaring chromosphere}
Response of chromosphere during flares:
Most focus is on corona, where energy
is stored and MR and CMEs are released.
But chromosphere dominates the
radiative energy budget, so somehow energy is transported down rapidly, and how
the layers respond can help us characterize energy transport and
conversion.% Yes this does work to keep mynote on same line in pdf :)
%
\mynote[8/14/18]
{Written on the back of post-it notes pad, no date.
Not sure if I was working on how to word something in paper/dissertation, or
just testing myself to see if I know the point of my own research.}




To study the chromospheric response to flares, it is necessary
to first understand the structure and dynamics of the chromosphere
during quiet, non-flaring periods.

The chromosphere got its name from Lockyer and Frankland because of the vivid
red color of the limb (from Hα emission) before and after eclipses.
According to Hudson (2007), the chromosphere is a ``well-defined layer''.

Globally, the chromosphere is a complex, dynamic interface that lies between
the visible disk of the photosphere and the hot corona.
About 500 km above the base of the photosphere at $\sim$5800 K, the temperature
decreases with distance as one would expect from basic physics,
until it reaches $\sim$4200 K.
This is known as the temperature minimum $T_{min}$, and
marks the lower boundary of the chromosphere.
The absolute
thickness of the chromosphere varies in the literature, usually between 2000
and 5000 km (close to the radius of the Earth).

At the upper boundary of the chromosphere, the temperature rapidly increases to
the order of a million K in the corona.
This temperature jump takes place in a region defined as the transition
region (TR), a relatively thin layer of plasma that separates the chromosphere
from the overlying corona.

The base of the photosphere (z=0) is defined as where
$\tau_{5000}$ = 1.
The temperature reaches 10000 K at z $\approx$ 2300 km, and then skyrockets
to a million K over the TR (Kneer \& Von Uexkull).


\mynote{
    What makes the chromosphere a distinct, separate layer of the
    solar atmosphere? It can be defined in multiple ways, including
    temperature profile (Tmin to rapid increase that defines the TR),
    structure (high to low $\beta$), or if you want to be really technical,
    the optical depth at the limb.
    Why is the atmosphere divided this way?
    What makes the chromosphere distinct from the other layers?}




%\clearpage  % does this cause problems, since clearpage is already
% done when using \input command?
\newpage
\subsection{Magnetic field, structure, and the \textit{plasma-$\beta$}}



The lower region of the chromosphere reveals
a magnetic network similar to that of the photosphere, up to
$\sim$1300-1500 km. The pattern reflects that of the supergranular
pattern at the photosphere. Above this, the consistent, homogeneous
plasma gives way to complex structures (see \cite{Judge2006} for a
review). At the boundary between these two regions, the dominant
pressure changes from thermal gas pressure to magnetic pressure. This
property is quantified in a parameter called the \textit{plasma
$\beta$}, defined as the ratio of thermal gas pressure to magnetic
pressure:

\begin{equation}
    \beta \;=\; \frac{P_{thermal}}{P_{magnetic}}
    \;=\; \frac{\rho k_{B}T/\mu m_{amu}}{B^{2}/8\pi}
    \;\propto\; \frac{\rho T}{B^{2}}
\end{equation}

where $k_{B} = 1.38 \times 10^{-16}$ [erg K$^{-1}$] is Boltzmann's
constant, $\mu$ is the average mass per particle in atomic mass units
[amu], $m_{amu} = 1.67 \times 10^{-24}$ [g] is the nucleon mass, $\rho$
is the mass density, $T$ is the local temperature, and $B$ is the
magnetic field strength [gauss].

In practice, a plasma is described by expressing its $\beta$
relative to unity (rather than a specific numerical value). A plasma
with $\beta \gg 1$ is dominated by thermal pressure, and appears as a
fairly homogeneous, uniform distribution of material. A plasma with
$\beta \ll 1$ is dominated by magnetic pressure and is characterized
by complex structures, such as spicules and coronal loops.
Figure~\\ref{fig:plasma\_beta}
shows how $\beta$ varies with height
throughout the solar atmosphere. The balance between the two pressures
plays a vital role in the structure and dynamics of a plasma, rather
than the absolute strength of the magnetic field alone. At the
$\beta=1$ boundary, mode conversion can occur for propagating acoustic
waves.

\begin{framed}
    Physics sidebar: mode conversion, $\beta = 1$ surface.
\end{framed}

% Structures
Magnetic flux is generally concentrated in active regions, such as the umbrae
of sunspots. As this flux rises into the atmosphere, it gives rises to the
complex structures that are observed as the emission from particles that flow
along magnetic field lines (a result of the so-called ``frozen-in'' theorem).
Magnetic structures play an important role in diagnosing conditions in the
solar atmosphere. Field lines often serve as waveguides for oscillations in the
atmosphere, revealed indirectly by the manner in which the plasma responds to a
disturbance. Structures are also a potential means of transportation of mass
and energy to the corona.


The upper part of
the chromosphere contains structures called spicules that emit the distinct,
dark pink color of H$\alpha$, and can be seen around the disk of the sun during
solar eclipses. Above the base of the photosphere (where $T \approx 5800$~K),
the temperature falls off with distance until it reaches the so-called
\textit{temperature minimum}, at $T_{min} \approx 4200$~K. The height at which
$T = T_{min}$ marks the lower boundary of the chromosphere, about 500 km above
the base of the photosphere. This is where acoustic waves shock and dissipate,
causing a sharp increase in temperature. The absolute thickness of the
chromosphere varies roughly between 2000 and 5000 km, slightly less than the
radius of the Earth. The temperature reaches $\sim$10000 K at the upper
boundary of the chromosphere, at $z \approx 2300$ km, and then rapidly
increases to $T\!\gtrsim\!10^{6}$~K in the corona.
This temperature jump takes
place in a region defined as the \textit{transition region} (TR), a relatively
thin layer of plasma that separates the chromosphere from the overlying corona.
The temperature gradient in the solar atmosphere above the base of the
photosphere is shown in
Figure~\\ref{fig:solar\_atm}.



\subsection{Observing the chromosphere}

\subsubsection{Characteristic spectral lines/emission}

% Table - Spectral lines in atmosphere
\begin{deluxetable}{c c c c}
    \tablewidth{\textwidth} % default = \pagewidth, natural width = 0pt
    \tablecaption{
        Characteristic spectral lines in the solar atmosphere.
        \label{tab:comp}}
    \tablehead{
        \colhead{Species} &
        \colhead{Wavelength [\AA{}]} &
        \colhead{Layer} &
        \colhead{Temperature [K]}
    }
    \startdata
        \myion{Cl}{I} & \nodata & Low to middle Chrom. & \nodata \\
        \myion{O}{I} & \nodata & Low to middle Chrom. & \nodata \\
        \myion{C}{I} & \nodata & Low to middle Chrom. & \nodata \\

        \myion{Mg}{II} k line& 2796 & Chromosphere & \nodata \\
        \myion{Mg}{II} wing & 2830 & \nodata & \nodata \\
        \myion{Ca}{II} H & 3969 & Chromosphere & \nodata \\
        \myion{Ca}{II} K & 3934 & Chromosphere & \nodata \\

        \myion{C}{II} & 1334 & Upper Chrom, low TR & \nodata \\
        \myion{C}{II} TR line & 1335 & Upper Chrom, low TR & \nodata \\
        \myion{Si}{IV} & 1394/1403 & Transition Region & 65000 \\
        \myion{Si}{IV} TR line & 1400 & Transition Region & 65000 \\

        \myion{O}{IV} & $\sim$1401 & \nodata & 150000 \\
        \myion{Fe}{XIII} & 10747 & Corona & \nodata \\
    \enddata
\end{deluxetable}


\subsubsection{Challenges}

Here are some challenges.



% ---}

%% 3-minute oscillations {---
\clearpage
\section{3-minute oscillations}



\subsection{History of 3mOs}

\mynote{%
    When were 3mOs first observed and why? What was the motivation
    behind the research article that produced the first results?
}

The famous 3-minute oscillations of the chromosphere
came to light as results from several studies in the 1960s and early
1970s were published.
They have been discovered independently in different locations,
observation methods, and behavioral forms.
The various forms
are often used interchangeably in the literature, which can cause
some confusion.

3mOs were first discovered via velocity observations in the quiet chromosphere
by \cite{Jensen1963}.
In the internetwork regions,
the Fourier velocity power spectrum peaks at
$\sim$5.5 mHz (3 minutes)
\citep{Orrall1966}.


Perhaps the most commonly cited source for 3mOs in active regions is the work from
\cite{Beckers1969}.
From intensity observations in Ca$^{+}$ above sunspot umbrae,
they discovered intermittently occurring flashes in intensity,
a phenomenon they designated
``umbral flashes'' to characterize
both their location and transient nature.
A short follow-up study from \cite{Wittmann1969} revealed
flashes with a period of 150 seconds (2.5 minutes).

Three years later, a similar study revealed
persistent oscillations in velocity above sunspot umbrae at the photospheric level,
also with a period of 3 minutes \citep{Beckers1974}.
The highest power tended toward the area
directly above sunspot umbrae, with peak-to-peak velocities up to
1 km s$^{-1}$.
They found clear periods of
178 seconds (2.97 minutes) at the umbral center,
255 seconds (4.25 minutes) at the umbral edge, and
300 seconds (5 minutes) in the penumbra.

At the time, these oscillations were concluded to be unrelated to the umbral
flashes detected in the chromosphere 3 years earlier.


\cite{Giovanelli1972} reported umbral oscillations in both velocity and intensity.

Many authors have dedicated multiple publications to investigating
this phenomenon. Lites is the first author on many papers submitted
over the course of a decade or so, between 1982 and 1992.

\subsection{Observations and physical interpretation}

\mynote{Figures here: at least one for each observation type.}

%% Inclination of magnetic field
Figure~\\ref{fig:Stangalini2011} shows power maps from
\cite{Stangalini2011} illustrating the power distribution of the
3-minute oscillations at the photospheric Fe~6173~\AA{} line and the
chromospheric Ca~8542~\AA{} line. Also shown is the power distribution
of the 5-minute oscillations. The 5-minute period has been attributed
to the global pulsations of the sun due to pressure modes
(\textit{p}-modes) in the interior. The image shows the suppression of
the 5-minute period in the umbra of sunspots, the opposite of the
3-minute behavior in the chromosphere above sunspots. It is commonly
observed in the photosphere that the 5-minute power above sunspot
umbrae is \emph{reduced} compared to the quiet sun by a factor of 2-5,
though it is still the dominant frequency
\citep{Felipe2010, Bogdan2006}.
\cite{Reznikova2012} also observed lowering of the cutoff frequency
due to inclination of magnetic field lines.
The presence of the 5-minute oscillations in the diffuse magnetic regions
surrounding the umbra is likely caused by the decrease in effective
cutoff frequency due to the inclination of the magnetic field.


Velocity observations reveal a sawtooth
pattern of periodic sharp blueshifts and gradual redshifts.
This pattern is characteristic of propagating shock wave fronts.
The shift toward the blue wing in emission lines indicates an
upward propagation through the atmosphere.
The lack of accompanying redshift supports the interpretation of propagating
waves and not standing waves reflected back toward the solar surface.
Shocks occur when the velocity of a wave exceeds the local
sound speed.
If the ``certain value'' of velocity at which umbral flashes occur is near
the local sound speed, it seems reasonable to assume that these flashes
are an observable signature of a shock wave propagating through the
atmosphere.

The sawtooth pattern and shift toward the blue wing in velocity observations
support the current
interpretation of 3mOs as manifestations of upward propagating
magnetoacoustic waves along magnetic field lines.

3mOs were initially believed to exist in cavities as
trapped, standing waves \citep{Scheuer1981}.
Chromospheric cavity between Tmin and TR
(\cite{Chae2015} $\rightarrow$ Leibacher \& Stein (1981))
This theory has been largely abandoned after studies have shown a cavity
is not needed to produce the observations
\citep{Fleck1991}.


Results from \cite{Tian2014} supported the propagation of shock waves
with peak-to-peak velocity amplitude between 5 and 10 km s$^{-1}$,
$\lesssim$ the local sound speed.

Many other authors have contributed to the
current interpretation as slow propagating SMA waves.
Similar results were found by
\cite{Brynildsen2004, Maltby1999, DeMoortel2000, OShea2002,
Brynildsen1999b, Brynildsen2002, Reznikova2012}

%%
%\cite{Bogdan2006}, page 319:
%Various ionization stages (increasing T) $\rightarrow$ high space and time
%coherence at different heights in the 3-min band.
%High time cadence
%$\rightarrow$ phase shifts, which agree with wavepackets propagating from
%photosphere to the base of the chromosphere with vertical phase velocity
%$\sim$cs (7-12 km per second).
%%

\myheading{Phot $\rightarrow$ Chrom}

The question that follows, then, regards the manner in which oscillations
are generated in the lower atmosphere and then
propagate upward into the chromosphere.

3mOs are possibly connected to the global 5-minute oscillations.
(see discussion above and figures from Stangalini et al. 2011).

The photosphere was already known to oscillate at a dominant frequency of 5
minutes. 5-minute oscillations are observed in the chromosphere as well, mostly
in the network regions of the quiet sun, where magnetic field strength is
relatively high. The 5-minute period in the photosphere has been attributed to
the global pulsations of the sun due to pressure modes (p-modes) in the
interior. In the photosphere, the Fourier power spectrum peaks at $\sim$3 mHz
(5 minutes) in the quiet sun. This power is reduced by a factor of 2-5 at the
umbra of sunspots, though it is still the dominant frequency \citep{Felipe2010;
Bogdan2006 p. 323}. The amplitudes of intensity variations in the photosphere
tend to be relatively small and difficult to measure, while those in the
chromosphere are dominated by the aforementioned umbral flashes.

%The non-isothermal region where a compressive acoustic disturbance is
%propagating along the magnetic field in
%direction $\hat{k}$
%Umbral $\vec{B}$ aligned with $\vec{g}$, so $H_{||} \rightarrow$
%`true' $H_{P}$.
%
%\begin{equation}
%    k_{||}^{2} = \frac{\omega^{2}}{c^{2}} - \frac{1}{4H_{||}^{2}}
%\end{equation}
%
%where $\omega = 2\pi\nu$ and $H = H_{P}$,
%the pressure scale height
%%\citep{Cally2001; Crouch2003}.
%(Cally 2001; Crouch \& Cally 2003).
%
%Waves propagate if
%$k_{||}^{2} \ge 0$
%($ \omega \ge c/2H_{||} $)
%threshold = acoustic cutoff frequency.
%$\Omega = \Omega(z)$ = max value of 5.2 - 5.5 mHz (3.21 - 3.03 minutes) with
%Tmin \mynote{Figure 2. of Cally et al. 1994}
%3-minute oscillations at the high frequency tail of photospheric
%oscillations.
%Conservation of wave energy: amplitude of velocity oscillation
%$\propto 1/\sqrt{\rho}$ for
%waves with frequency greater than the cutoff.
%For
%waves with frequency less than the cutoff,
%``osc do not give rise to prop. waves''.
%Dispersion relation determines
%decline in velocity predicted by the imaginary part of the wave number.
%
%``The dominant power above SS umbrae changes from 5 minutes in the photosphere to
%3 minutes in the chromosphere because of strong spatial damping of evanescent
%waves with height, whereas 3mOs at the cutoff frequency are not damped.''
%\mynote{
%    \cite{Milligan2017} $\rightarrow$ Noyes \& Leighton 1963.}
%Vertical velocity oscillations extending
%into the chromosphere
%(\cite{Judge2006} $\rightarrow$ Jensen \& Orrall (1963);
%Noyes \& Leighton (1963)).
%
%
%\myheading{Chrom $\rightarrow$ Corona}
%
%\textbf{What is the manner in which oscillations propagate from the chromosphere
%to the corona?}
%
%Bogdan \& Judge (2006)
%\mynote{
%    and Tian et al. (2014)?}
%: Intensity
%oscillations (in emission lines) have been observed to ``disappear'' after
%traveling from the chromosphere through the transition region toward the corona
%(around T $\ge 10^{6}$ K).
%The disappearance of a wave is generally attributed to
%dissipation, damping, or mode conversion.
%Thermal conduction is the principal
%dissipation mechanism at coronal temperatures, but it doesn't affect 3mOs
%until they travel a few thousand km into the corona.
%Slow magnetoacoustic waves are not subject to Ohmic dissipation.
%Viscosity can be neglected for a
%collisionless plasma. They can’t be converted from slow to fast magnetoacoustic
%waves because this phenomenon would take place at the coupling region
%(where $\beta \approx 1$),
%and the waves are observed to propagate at heights well above this.
%This
%observation also cannot be explained by reflection at some layer because that
%would result in a redshift in the observations, which is not the case.
%The best explanation is that the rapid increase in the temperature scale height
%causes the waves to dissipate.
%
%    \begin{equation}
%        \frac{2\pi}{k_{||}} \gg H_{T}
%    \end{equation}
%
%A wavetrain passing through the formation height of line causes uniform lift,
%compression, descent, and rarefaction over the entire layer over a single wave
%period. Exception to this: strong resonance lines like
%\ion{Ca}{2} H \& K and H$\alpha$, which
%are optically thick and broad.
%Coronal emission lines are optically thin.
%
%Compression and rarefaction reduce the overall contributions to net
%fluctuations in coronal emission line.
%\ion{Fe}{16}:
%Changes in emission line intensity time series as move upward in z
%(and hence, in T)
%\mynote{
%    \cite{OShea2002}}
%How high do they propagate? This question was posed by Reznikova et al. (2012).


\section{Coming around to the purpose of my research\ldots}
\subsection{Driving mechanism of 3mOs}

The origin and reason for the persistence of 3mOs remains unclear.
There are two dominating theories of where the 3mOs originate.

\subsubsection{Propagating waves from photosphere}

\mynote{Plot of cutoff frequency with height goes here.}

The prevailing theory is that they are generated below the chromosphere and
propagate upward.
The reason for the dominant period of 3 minutes
is the acoustic cutoff frequency, which peaks
at the base of the chromosphere at a value of $\sim$5.6 mHz,
or roughly 3 minutes, which agrees with the acoustic cutoff period of a gas with
the composition and pressure scale height of the upper photosphere
\citep{Kalkofen1994}.

The cutoff frequency defines the minimum frequency a wave must have in order to
propagate through a medium.
It is an intrinsic property of the medium,
determined by local macro parameters, such as pressure and density.
As these properties vary throughout the interior of the sun and its atmosphere,
the cutoff frequency itself is an indirect function of height.
In the interior of the sun, acoustic waves have frequencies higher than the
cutoff, so it is of less importance there. In the solar atmosphere, however,
the cutoff plays an important role in the behavior of waves and oscillatory
phenomena.

Waves that are generated in the lower atmosphere
and travel upward are able to propagate until
their frequency no longer exceeds the local cutoff frequency
\citep{DeMoortel2000, Brynildsen1999}.
When this happens, the wave effectively hits a wall, and will either be
reflected back down, shock and dissipate, or strongly damp out.
This is why there is a drop observed in the power spectrum at frequencies
lower than this above a certain height in the solar atmosphere.
\mynote{
    What height? Where does $\omega_{0}$ reach a maximum?}


It is still uncertain why there is also a drop in power for
frequencies higher than the cutoff.
\mynote{%
\textbf{Where are the higher frequencies?}
Maybe waves at higher frequencies simply aren't generated
in the first place.
\textbf{What types of waves are generated in the atmosphere, and why?}
}


\subsubsection{%
    Natural response of chromosphere at the acoustic cutoff frequency}

The second theory regarding the origination of 3-minute oscillations
is that the acoustic cutoff frequency is the frequency at which the
chromosphere naturally responds when a disturbance is introduced.
In other words, the oscillations are
generated within the chromospheric plasma, where they are observed.

\mynote{%
What is the significance of the chromosphere oscillating at its natural
frequency Milligan paper presented evidence supporting that the chromosphere
oscillates at its natural frequency in response to a disturbance, but never
stated why this is significant or the implications for bigger picture science.
}

This theory has been investigated and predicted by several authors.
\cite{Fleck1991}
found similar behavior in both isothermal and non-isothermal models, suggesting
that the temperature gradient throughout the chromosphere does not have an
effect on its oscillatory response.
\cite{Kalkofen1994}
simulated disturbances from both a single impulse and a continuous jostling
(i.e. a periodic piston), with frequencies above and below the cutoff. The
response to the single pulse tended toward the cutoff value, analogous to a
bell sounding at a particular frequency when it is struck. A series of papers
analytically examined adiabatic wave excitations in a gravitationally
stratified atmosphere \citep{Sutmann1995a, Sutmann1995b, Sutmann1998}. These
studies also supported a similar response to a disturbance, regardless of
whether they were impulsively or continuously introduced into the atmosphere.
\cite{Chae2015}
used simulations of a gravitationally stratified medium to show that 3-minute
oscillations naturally arise when such a medium is disturbed.

While many studies have predicted this behavior,
actual observations of this phenomenon are quite rare.
\cite{Kwak2016}
observed enhanced oscillations in response to a downflow
event attributed to a plume-like feature above the umbra of a sunspot.

% ---}
