\clearpage
\section{The Solar Dynamics Observatory (SDO)}

\subsection{The Atmospheric Imaging Assembly (AIA)}

The Atmospheric Imaging Assembly (AIA; \cite{Lemen2012, Boerner2012})
on board the
\textit{Solar Dynamics Observatory} (SDO; \cite{Pesnell2012})
obtains full disk images throughout the solar atmosphere,
using narrow band filters centered on 10 different wavelengths.
It primarily collects EUV emission that originates in the corona.
In addition it obtains thermal UV continuum data in
two bandpasses centered on 1600\AA{} and 1700\AA{}.
Two of these channels provide measurements of
thermal UV emission from the chromosphere.
AIA 1700\AA{} samples the
lower atmosphere around $T = 10^{3.7}$ K, close to $T_{min}$.
AIA 1600\AA{}
samples the transition region and upper chromosphere, and contains emission
from the \ion{C}{4} 1548\AA{} spectral emission line, as well as UV continuum.
Both UV channels have a cadence of
24 seconds and spatial size scale of 0.6 arcseconds per pixel.

These data allow the computation of spatially resolved power maps centered
on the frequency of interest.

\paragraph{Ambiguity in formation height}

% Table - AIA channels


% Add vertical space: data & data \\ [dimen]
\begin{deluxetable}{l l l l}
    \tablewidth{\textwidth} % default = \pagewidth, natural width = 0pt
    \tablecaption{
        Relevant properties of AIA channels used in this dissertation
        (information obtained from \cite{Lemen2012}).
        \label{table_aia}
    }
    \tablehead{
        \colhead{central $\lambda$ (\AA{})} &
        \colhead{Formation T (K)} &
        \colhead{Source of emission}
    }
    \startdata
    1600 & $10^{5.0}$
    & upper photosphere continuum \& TR (\ion{C}{4}) emission line \\
    1700 & $10^{3.7}$
    & photosphere \& T$_{min} continuum$ \\
    \enddata
\end{deluxetable}


Table~\ref{table_aia} gives the standard values for the
formation temperatures and primary source of emission for each
of the UV channels on AIA.
However, due to the difficulties in observing the chromosphere
(described in the chromosphere background section),
there is a certain ambiguity in the formation height of the emission
in these channels, especially during flares.



\subsection{The Helioseismic and Magnetic Imager (HMI)}

The Helioseismic and Magnetic Imager (HMI; \cite{Scherrer2012}),
is another instrument on board
the \textit{Solar Dynamics Observatory (SDO)}.
HMI obtains full disk data in the form of four types of filtergrams:
line-of-sight magnetograms, vector magnetograms,
Doppler velocity, and continuum intensity images.
Each of these filtergrams is
measured around the \ion{Fe}{1} absorption line at 6173\AA{}
with a passband width of 0.076\AA{}.
Each data product has a cadence of 45 seconds (with the exception
of the vector magnetograms, at 135 seconds),
and {spatial size scale} of 0.5 arcseconds per pixel
\citep{Schou2012}.

Data from HMI is
used along with data from AIA for several reasons.
One is to provide context for the magnetic field configuration in the
photospheric counterpart, which sunspots are part of a pair, what
the polarity of each sunspot is, etc.
This data is also
used to study potential
correlations between magnetic field strength and oscillatory
behavior in the chromosphere.
