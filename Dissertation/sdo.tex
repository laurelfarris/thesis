\section{The Solar Dynamics Observatory (SDO)}

\subsection{The Atmospheric Imaging Assembly (AIA)}

The Atmospheric Imaging Assembly (AIA; \cite{Lemen2012}) on board the
\textit{Solar Dynamics Observatory} (SDO; \cite{Pesnell2012})
obtains full disk images throughout the solar atmosphere,
using narrow band filters centered on 10 different wavelengths.
Two of these channels provide measurements of
thermal UV emission from the chromosphere.
The 1700\AA{} channel mostly contains continuum emission from the
temperature minimum, and
the 1600\AA{} channel covers
both continuum emission and the \ion{C}{4} spectral line in the upper
photosphere and transition region.
Both UV channels have a cadence of
24 seconds and spatial size scale of 0.6 arcseconds per pixel.

These data allow the computation of spatially resolved power maps centered
on the frequency of interest.

\paragraph{Ambiguity in formation height}


\mynote{From propopsal:}
The Atmospheric Imaging Assembly (AIA) \citep{Lemen2012, Boerner2012}
on board \textit{SDO} primarily collects EUV emission
that originates in the corona.
In addition it obtains thermal UV continuum data in
two bandpasses centered on 1600\AA{} and 1700\AA{}. AIA 1700\AA{} samples the
lower atmosphere around $T = 10^{3.7}$ K, close to $T_{min}$. AIA 1600\AA{}
samples the transition region and upper chromosphere, and contains emission
from the \ion{C}{4} 1548\AA{} line, as well as UV continuum. Both data sets
are obtained at a cadence of 24 seconds, with a spatial resolution of $\sim$0.6
arcseconds per pixel.

\mynote{09 October 2018 - From current state of article (Phase 1):}
The data used to analyze this flare is provided by
AIA \citep{Lemen2012},
one of the instruments on board
SDO \citep{Pesnell2012}.
AIA obtains full disk images throughout the solar atmosphere using narrow band
filters centered on 10 different wavelengths, two of which provide measurements
of thermal UV emission from the chromosphere.
The 1700\AA{} channel mostly contains
continuum emission from the temperature minimum, and the
1600\AA{} channel covers both continuum emission and the
\ion{C}{4} spectral line in the upper photosphere and transition region.
Both channels have a cadence of 24 seconds and
spatial size scale of 0.6 arcseconds per pixel.


\subsection{The Helioseismic and Magnetic Imager (HMI)}

The Helioseismic and Magnetic Imager (HMI)
is one of three instruments on board
the \textit{Solar Dynamics Observatory (SDO)}. It obtains four types of
filtergrams around the \ion{Fe}{1} line at 6173\AA{}. These filtergrams are
in the form of dopplergrams, vector magnetograms, line-of-sight magnetograms,
and continuum intensity images \citep{hmi}.


Data from
the Helioseismic and Magnetic Imager (HMI; \cite{Scherrer2012}),
also on board \textit{SDO}, is used to study potential
correlations between magnetic field strength and oscillatory
behavior in the chromosphere.
HMI obtains full disk data in the form of
line-of-sight magnetograms, vector magnetograms,
Doppler velocity, and continuum intensity,
measured at the \ion{Fe}{1} absorption line at 6173\AA{}
with a passband width of 0.076\AA{}.
Each data product has a cadence of 45 seconds (with the exception
of the vector magnetograms, at 135 seconds),
and {spatial size scale} of 0.5 arcseconds per pixel
\citep{Schou2012}.
