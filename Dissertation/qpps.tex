\subsection{Quasi-periodic pulsations (QPPs)}


\begin{comment}

\mynote[18 October 2018]{
    So I've been reluctant to discuss the following issue because I don't
    actually know much about it, just that a few authors have quoted these words:
    ``There have been few studies of QPPs in thermal emission.''
    However, just found a paper from {} that explains this, which is helpful
    not only because it provides me with an supplementary description,
    but it's another source, so I'm not just quoting Milligan's paper over
    and over.
}

\mynote[18 October 2018]{
    Why do we care about QPPs? This needs to be clear to the
    reader/listener right away.
}

Thermal emission from the lower atmosphere during flares
provides a potentially useful
way to probe the chromospheric dynamics and extract
information about the transportation and conversion of energy in this
region.

\mynote{Brief description of what QPPs are}

Embedded within the global shape of flare emission
are temporal fluctuations known as quasi-periodic pulsations (QPPs),
with periods ranging from $\sim$1 second to several minutes.
They have been observed during all three flare phases
in all wavelength bands.
(See \cite{VanDoorsselaere2016} for a recent review.)


Since they are clearly connected to the flare process,
QPPs are considered to be an intrinsic
property of flares, thereby providing a direct probe into the
reconnection site and the affected regions beyond.
QPPs are ``directly linked to the properties of the flare reocnnection
region and flare acceleration sites'' \citep{Inglis2015},
so a lot can be learned about flares by studying these.


\mynote[18 February 2019]{Copied from E-note. I think this was in my proposal,
but probably took it out.}
There is no strict definition for this kind of pattern, but it is generally
described, as quoted from \cite{Inglis2015}, as
``variations in the flux from a flare\ldots as a function of time,
which appear to include periodic components with characteristic timescales
ranging from one second up to several minutes.''


Constraining the physical mechanism(s) responsible for the
generation of QPPs is an ongoing investigation.
There are two leading theories that describe them as signatures of one of two
things:
\begin{enumerate}
    \item the energy release process (MR),
        i.e. the rate at which magnetic field lines are reconnected
        and accelerate non-thermal particles.
        This is often referred to as a ``load/unload'' mechanism.
    \item MHD oscillations induced in the ambient plasma by
        \begin{itemize}
            \item the magnetic reconnection itself, which would give the resulting
                QPPs the same ``built-in'' periodicity as the reconnection rate.
            \item the same trigger responsible for inducing magnetic reconnection
                in the first place \citep{Nakariakov2009}.
        \end{itemize}
        Theoretically, MHD oscillations would be induced in post-flare loops,
        due to the nature of these types of oscillations in the vicinity of a
        magnetic field.
\end{enumerate}

Each of the above mechanisms can produce similar observational signatures,
adding to the difficulty in identifying which is the cause, if not both
\citep{Brosius2016}.


Their small periodicities can also make them difficult to extract from the main
lightcurve \citep{VanDoorsselaere2016}.
\mynote[Mon Feb 18 03:15-ish MST 2019]{%
Modulation depth? I think that has more to do with the amplitude than
the period. Maybe it's because flare timescales are close to QPP periods,
so everything gets mushed together and it's more difficult to extract any one
individual signal.}
Sometimes these periodicies can be on the order of a few seconds, making
them difficult to probe with current available instrumental cadence.
\mynote[Mon Feb 18 03:37:31 MST 2019]{Oh.}


% Observations of QPPs in thermal emission during flares
Currently, there are few studies of QPPs in thermal emission.
Non-thermal emission in HXR and microwave bands has a higher modulation depth,
particularly during the impulsive phase of flares, when the global emission level
increases substantially
\citep{Hayes2016}.


\citep{Brosius2016} studied emission from the \ion{C}{1} line, and found
a period of $\sim$170 seconds. They
attributed this periodicity to the rate of injection of
non-thermal electrons.
Further investigation of QPPs in thermal emission was carried out by
\cite{Milligan2017}, who found enhancements of the 3-minute oscillations
in Lyman continuum and Lyman-$\alpha$ line emission.
However, no signature of a 3-minute oscillation was found in
the X-ray data generated by the flare, indicating
that the chromospheric response was not a
reflection of the rate of energy input.
These findings supported the idea that the chromosphere naturally
responds at its cutoff frequency, regardless of the periodicity of the
energy injection rate.
This result did not support the conclusion from
\cite{Sych2009}, who
proposed that the 3-minute waves leaked from sunspots into the upper
atmosphere by propagating along magnetic field lines, and triggered
the magnetic reconnection and subsequent energy release and particle
acceleration.


To investigate the transfer of energy during flares, \cite{Monsue2016}
conducted a pilot study of the spatial and temporal flare response of acoustic
oscillations in H$\alpha$ emission.
They were able to preserve both the two-dimensional spatial information
and the temporal information in the original data
with the power and frequency of the output of the Fourier transforms
by incorporating a technique called ``frequency-filtered amplitude movies''
from \cite{Jackiewicz2013}.
They reported a suppression of frequencies
between 1 and 8 mHz during the main phase, and an enhancement of frequencies
between 1 and 2 mHz before and after the flare. They suggested that the
suppression of lower frequencies could be evidence of the conversion of energy
from acoustic to thermal, and that the pre-flare enhancement may have been
indicative of an instability in the chromosphere.
In the conclusion of this study, they encouraged
the further investigation of
earlier time frames before the flare precursor, along with the
inclusion of additional cases to increase the statistical significance of the
findings.

% Temporal dependence of QPPs (flare phases). Science question - Time!!!
% Move this to methodology?
\cite{Hayes2016} examined oscillations during
both the impulsive and decay phase of an X-class flare, and considered
a combination of possible mechanisms for producing QPPs, where the
impulsive phase involved rapid injections from non-thermal particle
acceleration and subsequent heating of the surrounding plasma, while
thermal signatures persisted throughout the decay phase.
Propagating slow magnetoacoustic waves, such as the chromospheric
3-minute oscillations, show characteristic quasi-periodic patterns.
Several recent papers discuss the relationship between QPPs and
chromospheric oscillations.



% Move this to methodology?
The observations were acquired using
Lyman continuum data from \textit{SDO}/EVE,
Lyman-$\alpha$ line emission from \textit{GOES}/EUVS,
1600\AA{} and 1700\AA{} continuum from \textit{SDO}/AIA, and
HXR emission from \textit{RHESSI}.
EVE observes the sun as a star, with disk-integrated data.
%Figure~\ref{fig:Milligan2017} shows their results from the analysis on AIA data.

\end{comment}

% ---}
