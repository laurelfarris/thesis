% Table - BDA
\begin{deluxetable}{c c c c}
    \tablewidth{\textwidth} % default = \pagewidth, natural width = 0pt
    \tablecaption{
        Pre-flare, X-flare, and post-flare things that happened
        \label{tab:bda}}
    \tablehead{
        \colhead{Phase} &
        \colhead{Indices} &
        \colhead{Time range (UT)} &
        \colhead{What's happening}
    }
    \startdata
        Before  & 0:63 & 00:00--00:25 & blip at very beginning\\
        \nodata & 16:79 & 00:07--00:32 & Quiet\\
        \nodata & 27:90 & 00:11--00:36 & pre-flare through C-flare peak\\
        \nodata & 80:143 & 00:32--00:57 & Centered on C-flare\\
        \nodata & 90:153 & 00:36--01:01 & C-flare peak (close to previous t-seg; start to peak $\sim$4 min!\\
        \nodata & 147:200 & 00:59--01:24 & Through where em. back to pre-Cflare levels\\
        \nodata & \nodata & \nodata & also centered on brief flux increase at $\sim$1:01--1:05 (2b)\\
        \nodata & 175:238 & 01:10--01:35 & Quiet, cleared of obvious emission increase\\
        \nodata & 197:260 & 01:19--01:44 & BP in (1a), been there since 00:30,\\
        \nodata & \nodata & \nodata & though no noticable increase in flux from this SS\\
        During  & \nodata & 01:45--02:30 & \nodata\\
        After   & \nodata & 02:30--04:59 & \nodata\\
    \enddata
\end{deluxetable}
