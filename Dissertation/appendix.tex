%\appendix




%% --- Moved "physics background" here from background.tex
%%  (Wed Feb 13 03:02:53 MST 2019) --- %%



%% Physics background: waves and coronal seismology {---



\clearpage
\section{Physics background}

\subsection{Waves}

1D:
\[
    \frac{\partial^{2}\psi}{\partial x^{2}}
    - \frac{1}{v^{2}}\frac{\partial^{2}\psi}{\partial t^{2}}
    = 0
    \]

3D:
\[
    \frac{\partial^{2}\psi}{\partial x^{2}} \rightarrow
    \nabla^{2}\psi
    \]


propagation speed $v$:
\[
    v = \sqrt{ \frac{F_{r}}{\rho}}
    \]
$F_{r}$ = restoring ``force'';
$\rho$ = mass ``density''

Get solutions to wave equation by specifying:
\begin{itemize}
    \item ICs - prescribe amplitude and phase
    \item BCs - solutions are standing waves, or harmonics
\end{itemize}


\[
    \left( \nabla^{2} - \frac{1}{c^{2}} \frac{\partial^{2}}{\partial t^{2}} \right)
    \psi\left( \vec{r},t \right)
    = 0
    \]

\subsection{Coronal seismology}\label{cs}

\paragraph{Ideal MHD}\label{idealMHD}
Conditions of ``ideal MHD'' involve timescales between collision that are much
smaller than those of other processes, allowing the particles to take a
Maxwellian distribution. They describe the evolution of the plasma densities
(mass, energy, etc.) over time. Under conditions of ideal MHD, plasma acts as a
perfect conductor, and is completely frozen-in to the magnetic field lines.

The equations of ideal MHD are shown in Table~ref\{tab:idealMHD\}.
\\input{table\_idealMHD}

Figure~ref\{disp\} shows the location of the different MHD modes relative to
characteristic speeds both external to and inside a cylindrical volume of
plasma.

\paragraph{Characteristic speeds}
Relations between the (internal and external)
characteristic speeds (Alfv\'en, sound, and tube speeds)
determine properties of MHD modes guided by the tube.

\begin{enumerate}
    \item Sound speed: $C_{s} = \sqrt{\frac{\gamma p_{0}}{\rho_{0}}}
            \approx 166 T_{o}^{1/2}$ m s$^{-1}$ = 200 km s$^{-1}$

            (\emph{All} sound speeds $\approx \sqrt{ \frac{ED}{\rho_{I}} } $ ),
            where $\rho_{I}$ = inertial mass density.

            ED of magnetosphere = $\frac{B^{2}}{2\mu_{0}} $
            = $\frac{P}{\rho} $

            $C_{s} = \sqrt{ \frac{P\gamma}{\rho}  } $

    \item Alfv\'en speed = ``speed at which hydrodynamic waves can be propagated
    in a magnetically dominated plasma.''
        \mynote[07/17/17 15:21]{Source?}

        $C_{A} = \frac{B_{0}}{\sqrt{\mu_{0}\rho_{0}}}$
        \[
            V_{A} = 2.18\times10^{12}\frac{B_o}{\sqrt{n_o}}
            \textrm{m}\ \textrm{s}^{-1} = 3000 \textrm{km}\ \textrm{s}^{-1}
            \]
        (B$_{o} \sim$ 100 G; n$_{o}$ $\sim$ 10$^{16}$ m$^{-3}$)
    \item Tube/Cusp speed:
        $C_{T} = C_{s}C_{A}/\left(C_{A}^{2} + C_{s}^{2}\right)^{1/2}$
        = combination of sound and Alfv\'en speeds
\end{enumerate}
Audible hearing limit: $C_{s}$ = 20 Hz. In space, $\nu \approx 10^{-7}$ Hz.


%The behavior of linear perturbations of the form
%\[
%    \delta P_{tot} \left( r \right)
%    \textrm{exp} \left[ i \left( k_{z}z + m\phi - \omega t \left) \right]
%    \]
%is governed by  the following system of first order differential equations and
%algebraic equations:
%\[
%    D\frac{\mathrm{d}}{\mathrm{d}r} \left( r\xi_{r} \right)
%    = \left( C_{A}^{2} + C_{s}^{2})\ldots
%    \]

\subsubsection{Kink waves}

\paragraph{Speed}
In the long wavelength limit, the phase speed of all but sausage fast modes
(see below) tends to the so-called kink speed, which corresponds to the density
weighed average Alfv\'en speed.
Phase speed for kink waves is equal to this kink speed, $c_{k}$:
\[
    v_{ph}
    = c_{k}
    = \sqrt{\frac{\rho_{o}V^{2}_{Ao}+\rho_{e}V^{2}_{Ae}} {\rho_o+\rho_e}}
    \approx V_{A}\sqrt{\frac{2}{1 + \frac{\rho_{e}}{\rho_{o}}}}
    \]
in the low-$\beta$ plasma.

Frequency:
\[
    w_{K} = \sqrt{ \frac{2k_{z}B^{2}}{\mu(\rho_{i}+\rho_{e})}  }
    \]
Slab phase speed:
\[
    V_{A_{e}}
    \]
Tube kink speed:
\[
    c_{K} = \frac{B_{i}^{2} + B_{e}^{2}}{\sqrt{\mu(\rho_{i}+\rho_{e})}}
    \]
(mean Alfv\'enic speed). Slab (or tube) is moved laterally; little variation in
cross-sec, density, or intensity.

\paragraph{Period}
\[
    P = \frac{2\ell}{V_A}\sqrt{\frac{1+\rho_e/\rho_o}{2}}
    \]
where $\lambda=2\ell$ ($\ell$ is the loop length).
Typically, $\ell \approx 60-600$ Mm in the corona.

Period of global kink mode:
\[
    P = \frac{2\ell}{c_{K}}
    \]
where $P \approx $ few seconds - minutes

\newpage
\subsection{Thermal and non-thermal energy}
Thermal energy is radiated by matter in \textit{thermal equilibrium}
The extreme ends of the spectrum (radio and HXR) are dominated by
radiation from non-thermal particles in the early stages of flare
energy release, but the intermediate regimes (visible, EUV, and SXR)
are dominated by radiation from thermal particles during the decay
period, over longer timescales.
Figure~\\ref{fig:nt} shows a comparison
between the two types of energy across the electromagnetic spectrum.
Non-thermal particles are revealed by observed energies that, if
caused by thermal particles, would indicate temperatures far higher
than observed, according to
\begin{equation}
    E_{thermal} = \frac{3}{2}k_{B}T
\end{equation}
where $k_{B} = 1.38 \times 10^{-16}$ = Boltzmann's constant in erg
K$^{-1}$, $T$ = temperature in Kelvin, and $ E_{thermal} $ = thermal
energy in erg. The distinction between thermal and non-thermal
particles is important when diagnosing observations. They result from
different processes and provide clues to the conditions in their
environment.
% ---}


\section{Normalization of Power Spectrum}

Figure~\ref{norm_vs_not} shows power spectra for each AIA channel
during the X-class flare and during the pre-flare period.
These were calculated with and without normalizing to see which
method was best for the type of analysis utilized in this work.

\begin{figure*}\centering
    \includegraphics[draft=false,width=0.8\textwidth]{norm_vs_not.pdf}
    \caption{Fourier power spectrum for quiet vs. flaring times,
    comparing calculations without normalizing (left panels) to
    those with normalizing (right panels) for both AIA 1600\AA{}
    (top) and AIA 1700\AA{} (bottom).
    All spectra were obtained by applying a Fourier transform
    to the integrated emission from AR 11158.
    \label{norm_vs_not}}
\end{figure*}

