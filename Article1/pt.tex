\subsection{Temporal evolution of 3-minute power}

\begin{figure*}[htb!]\centering
    %\includegraphics[draft=false,width=1.00\textwidth]{time-3minpower_flux.pdf}
    \includegraphics[draft=false,width=1.00\textwidth]{time-3minpower_maps.pdf}
    \caption{%
        Temporal evolution of the 3-minute power $P(t)$ in
        AIA 1600\AA{} (green curve) and AIA 1700\AA{} (purple curve).
        %Top: $P(t)$ obtained by applying a Fourier transform to the
        %integrated flux from AR 11158.
        %Bottom:
        $P(t)$ per unsaturated pixel, obtained by summing over power maps
        whose pixel values = 0 if saturated, then divided by number of
        unsaturated pixels.
        Each point in time
        %represents the sum (in x and y) of each power map and
        is plotted as a function of the center
        of the time segment over which the Fourier transform was applied to
        obtain the power map over which the point was summed.
        %The axis labeled $T = 64$ is scaled to show the
        %length of each time segment relative to the full time series.
        The vertical dashed lines mark the \textit{GOES} start, peak, and end
        times of the flare at 01:44, 01:56, and 02:06 UT, respectively.
        \label{power_vs_time}}
\end{figure*}


The evolution of the 3-minute power with time
was calculated from the power maps
by summing over $x$ and $y$ in each map $P(x,y,t_{i})$, and taking the total
to be the 3-minute power of the active region during each time segment.
This is shown in Figure~\ref{power_vs_time}.
Each point is plotted as a function of the center of the time segment
over which the Fourier transform was applied to obtain that point.
The axis labelled $T=64$ shows the scale for this length of time.

Power as a function of time obtained from total flux and power maps
is shown to check for possible contradictions between the two.
Integrating flux over the AR
before applying the Fourier transform
has the potential effect of reducing or canceling signal from
pixels whose intensity variations are out of phase.

The reason for the apparent periodicity in the plots of 3-minute power with time
is unclear.
The calculations were repeated for power centered on the
5-minute period and the 2-minute period,
and resulted in the same trend,
with the additional observation
that the length of the central period
scaled with the period in the plot of power vs. time.
This pattern is attributed to computational effects.


%Since each point in the temporal plots represents
%the behavior of a much longer time segment than the plot seems to imply,
%a running average over the length of time segment $T$
%was computed from
%the data obtained from summing the power maps,
%and is shown in Figure~\ref{power_vs_time}.
%This greatly smoothed the periodicity in the power,
%but with further loss of temporal resolution.


The persistence of the 3-minute power toward the end of the gradual phase
in AIA 1700\AA{} is consistent with the results of the wavelet analysis
carried out by \cite{Milligan2017}.
