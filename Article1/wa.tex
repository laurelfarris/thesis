\subsection{Discrete wavelet analysis}

\begin{figure*}[htb!]\centering
    \includegraphics[draft=false,width=0.9\textwidth]{wa.pdf}
    \caption{
        Time-frequency power plots from AIA 1600\AA{} (top panel) and AIA
        1700\AA{} (bottom panel), obtained by applying a Fourier transform to
        integrated emission from NOAA AR 11158 in discrete time increments of
        64 frames ($\sim$25.6 minutes) each. The dashed horizontal line marks the
        central frequency $\nu_{c}$ at $\sim$5.6 mHz, corresponding to a period
        of 3 minutes. The dotted horizontal lines on either side of $\nu_{c}$
        mark the edges of the frequency bandpass $\Delta\nu$ = 1 mHz. The
        vertical lines mark the flare, start, peak, and end times as determined
        by \textit{GOES}. The power is scaled logarithmically and over the same
        range in both channels.
        \label{wa}}
\end{figure*}


The technique described in \S\ref{analysis} was applied
to the integrated flux from AR 11158 at
discrete intervals of $T$ = 64 images with no overlap
(i.e. for start time $t_{0}$, then $t_{1} = t_{0}+T$, etc.).
This provided a ``quick and dirty'' way
to compare the spectral power at a range of frequencies.
This method produces similar results to those
obtained with wavelet analysis, though
at lower resulting frequency and time resolution.
The results are are shown in
Figure~\ref{wa}
for frequencies between 2.5 and 20.0 mHz (400 and 50 seconds, respectively).
The central frequency $\nu_{c} = 5.56$ mHz
and the frequency bandpass $\Delta\nu$ at 5 and 6 mHz
are marked by the horizontal dashed lines.
The power at all frequencies
appears to be enhanced during the X-flare compared to their
non-flaring power before and after.
%This implies that the chromospheric plasma oscillates at a range
%of frequencies in response to energy injection.
During the small events before and after the flare,
the power at lower frequencies is enhanced,
but the power at higher frequencies is suppressed relative to
the same frequencies for adjacent time segments.

At all points in time when power enhancement occurs for any frequency,
there appears to be a correlation with flux increase.
