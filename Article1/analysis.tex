The technique used to calculate power maps as functions of
space and time is similar to that \correction{employed} by
\cite{Jackiewicz2013} and \correction{further employed} by \cite{Monsue2016}.
The general method is as follows:
For a data set of $N$ images, each power map
$P(x,y,t_{i})$ is generated by
applying a Fourier transform to every pixel at
$(x,y)$ in the temporal direction,
from $t_{i}$ to $t_{i}+T$, where $T$ is the length of the time segment.
The power is averaged over a frequency band $\Delta{\nu}$
of user-defined width, centered on the frequency
of interest. This process is repeated at every timestep, for starting
times from $t_{0}$ to $t_{N-N_{T}}$.

The data set for AR 11158 consisted of $N$ = 749 images (5 hours)
of AIA observations in each channel.
Each time
segment $T$ was set to 64 images ($\sim$25.6 minutes).
The value of $T$ was chosen based on
a balance between sufficient length to obtain frequencies close to
that of the 3-minute period and not so long as to lose information on
timescales over which the 3-minute power was previously observed to change.
Each Fourier transform was applied without detrending the data since
the frequency of interest was well outside the global flare signal.
(As a check on this, a Fourier filter was applied with a cutoff period
above 400 seconds. The power spectra for the periods of interest did not change.)
If a saturated pixel was encountered in any segment $t_{i}$ to $t_{i}+T$,
it was excluded from the power map for that time segment,
and the location ($x, y$) of that pixel was set to zero.

The frequency bandwidth $\Delta\nu$ was set to 1 mHz
\correction{centered on $\nu \sim 5.6$ mHz}.
This is consistent with similar techniques applied in previous studies.
For instance, \cite{Stangalini2011} used a 1-mHz frequency bandpass between 4.8 mHz
(208.3 seconds) and 5.8 mHz (172.4 seconds) when calculating power maps
\correction{around 5.6 mHz} for the chromosphere and photosphere.
\cite{Tripathy2018} also used a band of 1 mHz
over 0.1-mHz steps from 1 to 10.5 mHz.
\cite{Reznikova2012} used a bandpass of only 0.4 mHz\ldots

With these input parameters,
a frequency resolution $\partial\nu$ of $\sim$0.65 mHz
was obtained.
Two frequencies were obtained within $\Delta\nu$ at
5.21 mHz (192.00 seconds) and
5.86 mHz (170.67 seconds).

The average power over $\Delta\nu$ for each unsaturated pixel in
time segment $T$ from
$t_{i}$ to $t_{i}+T$
was taken to be the 3-minute power in each power map.
Since only two frequencies were obtained within $\Delta\nu$,
and were centered around the frequency of interest,
the average was computed without the application of a filter.

Power maps representing the 3-minute power over NOAA AR 11158
in space and time
were obtained at every starting point in the time series (up to $N-T$)
by applying a Fourier transform to the signal from each pixel,
and averaging the power within the
1-mHz frequency bandwidth $\Delta\nu$ centered around 5.6 mHz (3 minutes).
