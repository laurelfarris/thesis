The amplitude of the small-scale variations in 3-minute
power is higher for AIA 1700\AA{} almost everywhere with
the exception of the main phase of the X-class flare.

When both plots are normalized between 0.0 and 1.0, the variation in
power is higher from AIA 1700\AA{} almost everywhere except the main phase of the
X-class flare, when the power from 1600\AA{} is slightly higher.

The power from AIA 1700 is higher than AIA 1600 by
1000 counts at all points throughout the time series.
Compared to 1700\AA{},
the 1600\AA{} 3-minute power appears to increase more
(relative to its own minimum) and at a faster rate.
The standard deviation for $P(t)$ from integrated flux
for 1600\AA{} is $4.9 \times 10^{4}$, and
for 1700\AA{} is $2.7 \times 10^{4}$.

If the emission from AIA 1600\AA{}
originates from a higher location in the atmosphere
than the 1700\AA{} emission,
a possible explanation for the higher, sharper increase
is that the energy from the non-thermal particle beam
dissipates as it travels through deeper layers of the chromosphere.
Because AIA 1700 originates from a deeper layer, it is probably more dense
as well, which would cause the plasma to cool at a slower rate.
If the 3-minute power originated from the transition region rather than the
upper photosphere, this may coincide with the levitation of hot chromospheric
plasma upward into post-flare loops.
Although the emission from AIA 1700\AA{} is generally thought to originate
in lower formation heights than emission from 1600\AA{},
the latter spans a broader temperature range, and contains
emission from \ion{C}{4} line.
Determination of the AIA 1600\AA{} formation height is more complicated
during flares because the \ion{C}{4} is more likely to be contributing
to the signal, and both channels may be sampling at deeper layers than
they are thought to during non-flaring times.
