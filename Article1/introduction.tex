%&<tex>
Most of the radiative energy associated with solar flares
is emitted from the chromosphere
in the form of optical and UV emission, but the mechanism of energy
transport from the magnetic reconnection site to the chromosphere
and subsequent conversion to other forms remains unclear.
The chromosphere has been observed
to oscillate in response to an injection of energy,
suggesting that the nature of such oscillations may reveal
something about the nature of energy deposition and conversion
associated with flares.
In this paper, we aim to characterize the oscillatory response of
the chromosphere before, during, and after an X-class flare
with the goal of
further investigating the ``flaring chromosphere'' and
helping to constrain the origin of the persistent 3-minute oscillations
in the chromosphere.

% 3-minute oscillations in (non-flaring) chromosphere
%
The 3-minute oscillations observed in the chromosphere
have been attributed to the upward propagation of slow
magnetoacoustic waves that originate in deeper layers
of the solar atmosphere.
The acoustic cutoff frequency at the base of the chromosphere,
$\nu_{0} \approx 5.6$ mHz ($\sim$3 minute period),
effectively creates a barrier across which waves can travel
only if their propagation frequency is higher than $\nu_{0}$.
%The 3-minute oscillations have also been attributed to
Another explanation is that the acoustic cutoff frequency
is the natural frequency at which the chromospheric plasma will
respond to a disturbance.
This was predicted and shown numerically by
a series of papers by \cite{Sutmann1995a,Sutmann1995b,Sutmann1998},
and other studies by \cite{Chae2015}.
Enhancement of oscillations close to a period of 3 minutes
has since been observed associated with a variety of phenomenae.
\cite{Kwak2016} observed
the response of the chromosphere to a downflow event
using high-resolution spectra from the
\textit{Interface Region Imaging Spectrograph}
(\textit{IRIS}; \cite{DePontieu2014}).
\cite{Kosovichev2007} detected a different type of oscillations above
an active region associated with a flare.

\cite{Milligan2017}
observed an enhancement in the 3-minute power from thermal emission
that was not present in X-ray emission associated with the flare.
supporting the prediction that the chromosphere naturally responds
to an impulsive disturbance at the acoustic cutoff frequency.

\cite{Kumar2006}
observed an enhancement in velocity oscillations between 5 and 6.5 mHz
in emission in close proximity to the source of HXR emission
in RHESSI images,
and interpreted this locally concentrated enhancement as energy injection by
non-thermal particles.


\cite{Brosius2015} studied UV stare spectra of an M-class flare
in \ion{Si}{4}, \ion{C}{1}, and \ion{O}{4} lines,
and reported four complete intensity fluctuations with periods
around 171 seconds, in furthor support of the
model of energy injection in the chromosphere by non-thermal particle beams.



% QPPs
The 3-minute period falls within the typical range of periods observed
in the form of
small-scale fluctuations known as quasi-periodic pulsations (QPPs).
QPPs have been observed in flare emission throughout all stages
and across all wavelength bands, and are considered to be intrinsic signatures of
flare dynamics.
In particular, QPPs in thermal emission from the chromosphere
may provide a tool for investigating the transportation of energy from
the flare site down into the chromosphere
\citep{Inglis2015}.


\cite{Sych2009} suggested that the leakage of umbral 3-minute oscillations
into the upper atmosphere was the cause of flaring QPPs, supported by
observations of a similar periodicity in the flare emission.


\cite{Monsue2016} observed
an enhancement of frequencies
between 1 and 8 mHz
associated with an M- and X-class flare
in H$\alpha$ emission integrated over the AR,
but investigation of subregions revealed an
enhancement at low frequencies (1-2 mHz) in inner flare regions
before and after the flare, and a
suppression of oscillatory power over all frequencies
between 1 and 8 mHz during the main phases.

\cite{Awasthi2018} found two distinct pre-flare phases,
beginning with non-thermal particles and evolving into a
thermal conduction front.
\cite{Fletcher2013b} studied both the thermal and non-thermal response
of the chromosphere during the early stages of an M-class flare,
and found the main flux to originate from a different location from
the initial brightenings.

The goal of the present study is to
investigate the location of power enhancement
before, during, and after a flare.
The extra time will allow comparison between flaring and non-flaring chromosphere
to distinguish whether the
plasma is oscillating at the natural frequency of the chromosphere
or responding to an impulsive injection of energy.
The location of power enhancement will help
probe the nature of the energy deposition at various phases.

Here we present the spatial and temporal evolution of 3-minute power in the
chromosphere during the
\textit{GOES} X-class flare that occurred on 15 February 2011.
The Atmospheric Imaging Assembly (AIA; \cite{Lemen2012}) on board the
\textit{Solar Dynamics Observatory} (SDO; \cite{Pesnell2012})
provides images
with a spatial size scale of 0.6" per pixel and 24-second cadence
in thermal UV emssion from
two channels that sample the lower atmosphere.
These data allow the computation of spatially resolved power maps centered
on the frequency of interest.
The flare, data, and methodology are described in \S\ref{data}.
Results are presented and discussed in \S\ref{results}.
We conclude in \S\ref{conclusions}
with key preliminary findings and plans for the continuation and
development of this work.
