%&<tex>
Most of the radiative energy associated with solar flares
is emitted from the chromosphere
in the form of optical and UV emission, but the mechanism of energy
transport from the magnetic reconnection site to the chromosphere
and subsequent conversion to other forms remains unclear.
The chromosphere has been observed
to oscillate in response to an injection of energy,
suggesting that the nature of such oscillations may reveal
something about the nature of energy deposition and conversion
associated with flares.
In this paper, we aim to characterize the oscillatory response of
the chromosphere before, during, and after an X-class flare
with the goal of
further investigating the ``flaring chromosphere'' and
helping to constrain the origin of the persistent 3-minute oscillations
in the chromosphere.

% QPPs
Embedded within the large-scale variations
in a typical flare lightcurve
are temporal fluctuations known as quasi-periodic pulsations (QPPs).
Typical oscillation periods of QPPs
range from $\sim$1 second to several minutes,
and have been observed throughout the duration of solar flares
in all wavelength bands.
QPPs are considered to be an intrinsic
property of flares, thereby providing an observable
probe into the reconnection site and surrounding plasma
\citep{Inglis2015}.
While the specific mechanism that generates QPPs remains uncertain,
there are two prevailing theories for the mechanism
that generates QPPs.
The first theory posits that magnetic field lines reconnect periodically and
QPPs reflect the rate of energy deposition via
non-thermal particles accelerated each time MR occurs.
The second theory explains QPPs as a more indirect signature of
magnetic reconnection,
wherein MHD waves are induced in the plasma
in the immediate vicinity of the reconnection site,
either by the same mechanism that
triggered the initial onset of magnetic reconnection,
or by the reconnection process itself after it had begun
\citep{Nakariakov2009}.
The QPPs would therefore be observational signatures of these MHD waves.
The difficulty in narrowing down the source of QPPs
lies in the similarity in observational signatures
between the two outcomes.
It is also possible that both ocurr simultaneously, or during
different flare phases \citep{Brosius2016}.
The small periodicities can be difficult to extract from the global
lightcurve \citep{VanDoorsselaere2016}.
Thermal emission from the lower atmosphere during flares provides a potentially
useful way to probe the chromospheric dynamics and extract information about
the transportation and conversion of energy in this region.


% 3-minute oscillations in (non-flaring) chromosphere
Observations of non-flaring active regions
in both intensity and velocity
have revealed oscillations in all regions of the chromosphere
with a dominant period around 3 minutes.
They are particularly strong above the umbra of sunspots,
as first discovered by \cite{Beckers1969}, as well as
internetwork regions in the quiet sun \citep{Orrall1966}.
\cite{Reznikova2012} found a concentration of 3-minute power
above sunspot umbra in AIA UV emission.
These observations are interpreted as the upward propagation of
slow magnetoacoustic waves generated below the chromosphere
\citep{Brynildsen2004}.
One prevailing theory for the dominant power at 3 minutes is that
the acoustic cutoff frequency at the base of the chromosphere,
$\nu_{0} \approx 5.6$ mHz
(which corresponds to a period around 3 minutes)
effectively creates a barrier across which waves can travel
only if they propagate with frequency higher than the cutoff.
Another theory attributes these oscillations to
the chromospheric plasma responding to disturbances
at its own natural frequency.
This was predicted and shown numerically by
a series of papers by \cite{Sutmann1995a,Sutmann1995b,Sutmann1998},
and other studies by \cite{Chae2015}
with similar results for both impulsive and continuous stimulation.
\cite{Sych2009} suggested that the leakage of umbral 3-minute oscillations
into the upper atmosphere was the cause of flaring QPPs, supported by
observations of a similar periodicity in the flare emission.


During the past decade, several studies have revealed enhanced oscillations
in the chromosphere associated with an injection of energy.
Using Dopplergrams from MDI on SOHO,
covering several X-class flares,
\cite{Kumar2006}
found enhancements of the 3-minute oscillations in velocity
that preceded the \textit{GOES} peak time of the flares.
These enhancements were locally concentrated around regions that
produced hard X-ray emission, indicating
that the enhancement was caused by energetic non-thermal particles.
\cite{Brosius2015} studied UV stare spectra of an M-class flare
in \ion{Si}{4}, \ion{C}{1}, and \ion{O}{4} lines,
and reported four complete intensity fluctuations with periods
around 171 seconds.
Their results showing periodic brightenings supported the
model of non-thermal particle beams injecting the chromosphere
with energy.
\cite{Kwak2016} observed
the response of the chromosphere to a downflow event
using high-resolution spectra from the
\textit{Interface Region Imaging Spectrograph}
(\textit{IRIS}; \cite{DePontieu2014}).

\cite{Monsue2016} observed H$\alpha$ emission from the GONG network
and preserved both temporal and spatial information
using a technique devised by \cite{Jackiewicz2013}.
They initially observed an enhancement of all frequencies between
1 and 8 mHz during the flare from the entire AR,
but upon further investigation of subregions within the active regions,
they revealed a suppression of oscillatory power between 1 and 8 mHz
during the main phases of an M- and X-class flare,
and enhancement was only observed
at lower frequencies (1-2 mHz) before and after the flare.
They interpreted some of the changes as conversion to
thermal energy in the chromosphere.
They suggested that the enhancement at low frequencies
prior to the precursor pahse
may be attributed to the presence of an instability in the chromosphere
that could potentially precede strong flares.

\cite{Milligan2017}
observed an enhancement of oscillations during the main phase of an
X-class flare at frequencies between 2 and 20 mHz
(500 and 50 seconds, respectively) in thermal emission.
The greatest increase in power for UV emission
occurred around the 120 second period
during the rise phase of the flare, which coincided with the
timescale of the peak power in the RHESSI X-ray spectrum.
The power at 180 seconds, while not as high, started to increase earlier
and the enhancement lasted until around 02:00, several minutes after
the flare peak. The X-ray emission did not show enhancement at this
period at all.
This supports the prediction that the chromosphere naturally responds
to an impulsive disturbance at the acoustic cutoff frequency.

Though the results from \cite{Milligan2017} revealed an
enhancement in oscillatory power from thermal emission,
they did not reveal where this enhancement occurred relative to the
active region where the flare took place.
One of the goals of the present study is to expand previous results to
include spatially resolved distribution of the 3-minute power.
The initial location of the 3-minute power enhancement
may help probe the nature of the energy deposition, which can be
either injection by non-thermal particles beams, or
thermal conduction in some cases.
For example,
\cite{Awasthi2018} found two distinct pre-flare phases,
beginning with non-thermal particles and evolving into a
thermal conduction front.
\cite{Fletcher2013b} studied both the thermal and non-thermal response
of the chromosphere during the early stages of an M-class flare,
and found the main flux to originate from a different location from
the initial brightenings.

The motivation behind including pre-flare data in this project
is twofold.
First, it will provide a comparison between the flaring and non-flaring
chromosphere.
This is necessary to interpret whether
the enhanced power reflects the natural response of the chromosphere
at the acoustic cutoff frequency,
or if it merely reflects the
exterior properties of the energy source.
\mynote{Need to discuss theories of response at natural frequency
vs. rate of energy injection.}
Second, if the chromosphere exhibits pre-flare signals,
this would contribute to the field of space weather prediction.

In the absence of flares,
the 3-minute oscillations cease to dominate below the chromosphere.
However, if the chromospheric layers are greatly disturbed, they may
push into the photospheric layers below, producing oscillations that are
not normally present.
See \cite{Simoes2018} and \cite{Tripathy2018}.

% Hypothesis
The rise in thermal emission from the chromosphere associated with flares
is caused by the rapid energy injection and subsequent heating of the plasma,
which then radiates some of this energy in the form of thermal UV emission.
If the plasma were to oscillate in response to this energy at the site
of energy injection, it is expected that the locations would be the same
at first.

Here we present the spatial and temporal evolution of 3-minute power in the
chromosphere during the
\textit{GOES} X-class flare that occurred on 15 February 2011.
The Atmospheric Imaging Assembly (AIA; \cite{Lemen2012}) on board the
\textit{Solar Dynamics Observatory} (SDO; \cite{Pesnell2012})
provides images
with a spatial size scale of 0.6" per pixel and 24-second cadence
in thermal UV emssion from
two channels that sample the lower atmosphere.
These data allow the computation of spatially resolved power maps centered
on the frequency of interest.
The inclusion of data before and after the flare allows the comparison
of the flaring and non-flaring chromosphere
to distinguish whether the
plasma is oscillating at the natural frequency of the chromosphere
or responding to an impulsive injection of energy.
The flare, data, and methodology are described in \S\ref{data}.
Results are presented and interpreted in \S\ref{results}.
Discussion of results, including potential relationship between
oscillations and magnetic field is in \S\ref{discussion}.
Conclusions and proposed future work are discussed in
\S\ref{conclusions}.
