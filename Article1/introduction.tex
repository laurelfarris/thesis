%&<tex>
Most of the radiative energy associated with solar flares
is emitted from the chromosphere
in the form of optical and UV emission, but the mechanism of energy
transport from the magnetic reconnection site to the chromosphere
and subsequent conversion to other forms remains unclear.
The chromosphere has been observed
to oscillate in response to an injection of energy,
suggesting that the nature of such oscillations may reveal
something about the nature of energy deposition and conversion
associated with flares.
In this paper, we aim to characterize the oscillatory response of
the chromosphere before, during, and after an X-class flare
with the goal of
further investigating the ``flaring chromosphere'' and
helping to constrain the origin of the persistent 3-minute oscillations
in the chromosphere.

% QPPs
Embedded within the large-scale variations
in a typical flare lightcurve
are temporal fluctuations known as quasi-periodic pulsations (QPPs).
Typical oscillation periods of QPPs
range from $\sim$1 second to several minutes,
and have been observed throughout the duration of solar flares
in all wavelength bands.
QPPs are considered to be an intrinsic
property of flares, thereby providing an observable
probe into the reconnection site and surrounding plasma
\citep{Inglis2015}.
There are two prevailing theories explaining the mechanism
that generates QPPs.
One theory is that they reflect the energy deposition rate
via non-thermal particles accelerated periodically by
magnetic reconnection.
The second theory is that they are observational signatures of MHD waves
induced in the plasma near the reconnection site,
by either magnetic reconnection itself or the mechanism that triggered
the onset of reconnection
\citep{Nakariakov2009}.
QPPs in thermal emission provides insight into
the transportation and conversion of energy in the
chromosphere during flares.


% 3-minute oscillations in (non-flaring) chromosphere
%
The dominant power at the 3-minute period in the chromosphere
has been attributed to
the acoustic cutoff frequency at the base of the chromosphere,
$\nu_{0} \approx 5.6$ mHz,
which corresponds to a period around 3 minutes.
This effectively creates a barrier across which waves can travel
only if their propagation frequency is higher than $\nu_{0}$.
Another theory attributes these oscillations to the response of
the chromospheric plasma to disturbances
at its own natural frequency.
This was predicted and shown numerically by
a series of papers by \cite{Sutmann1995a,Sutmann1995b,Sutmann1998},
and other studies by \cite{Chae2015}.
\cite{Sych2009} suggested that the leakage of umbral 3-minute oscillations
into the upper atmosphere was the cause of flaring QPPs, supported by
observations of a similar periodicity in the flare emission.
Using Dopplergrams from MDI on SOHO,
covering several X-class flares,
\cite{Kumar2006}
found enhancements of the 3-minute oscillations in velocity
that preceded the \textit{GOES} peak time of the flares.
These enhancements were locally concentrated around regions that
produced hard X-ray emission, indicating
that the enhancement was caused by energetic non-thermal particles.
\cite{Brosius2015} studied UV stare spectra of an M-class flare
in \ion{Si}{4}, \ion{C}{1}, and \ion{O}{4} lines,
and reported four complete intensity fluctuations with periods
around 171 seconds.
Their results showing periodic brightenings supported the
model of non-thermal particle beams injecting the chromosphere
with energy.
\cite{Kwak2016} observed
the response of the chromosphere to a downflow event
using high-resolution spectra from the
\textit{Interface Region Imaging Spectrograph}
(\textit{IRIS}; \cite{DePontieu2014}).

\cite{Milligan2017}
observed an enhancement in the 3-minute power from thermal emission
that was not present in X-ray emission associated with the flare.
supporting the prediction that the chromosphere naturally responds
to an impulsive disturbance at the acoustic cutoff frequency.

\cite{Monsue2016} observed
an enhancement of frequencies
between 1 and 8 mHz
associated with an M- and X-class flare
in H$\alpha$ emission integrated over the AR,
but investigation of subregions revealed an
enhancement at low frequencies (1-2 mHz) in inner flare regions
before and after the flare, and a
suppression of oscillatory power over all frequencies
between 1 and 8 mHz during the main phases.

\cite{Awasthi2018} found two distinct pre-flare phases,
beginning with non-thermal particles and evolving into a
thermal conduction front.
\cite{Fletcher2013b} studied both the thermal and non-thermal response
of the chromosphere during the early stages of an M-class flare,
and found the main flux to originate from a different location from
the initial brightenings.

The goal of the present study is to
investigate the location of power enhancement
before, during, and after a flare.
The extra time will allow comparison between flaring and non-flaring chromosphere
to distinguish whether the
plasma is oscillating at the natural frequency of the chromosphere
or responding to an impulsive injection of energy.
The location of power enhancement will help
probe the nature of the energy deposition at various phases.

Here we present the spatial and temporal evolution of 3-minute power in the
chromosphere during the
\textit{GOES} X-class flare that occurred on 15 February 2011.
The Atmospheric Imaging Assembly (AIA; \cite{Lemen2012}) on board the
\textit{Solar Dynamics Observatory} (SDO; \cite{Pesnell2012})
provides images
with a spatial size scale of 0.6" per pixel and 24-second cadence
in thermal UV emssion from
two channels that sample the lower atmosphere.
These data allow the computation of spatially resolved power maps centered
on the frequency of interest.
The flare, data, and methodology are described in \S\ref{data}.
Results are presented and interpreted in \S\ref{results}.
Discussion of results, including potential relationship between
oscillations and magnetic field is in \S\ref{discussion}.
Conclusions and proposed future work are discussed in
\S\ref{conclusions}.
