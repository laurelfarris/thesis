The origin of the 3-minute oscillations of the chromosphere
has been attributed to both
slow magnetoacoustic waves
%with frequencies higher than the acoustic cutoff
propagating from the photosphere,
and to
oscillations generated
within the chromosphere itself
at its natural frequency
as a response to a disturbance.
Here we present an investigation of the
spatial and temporal
behavior of the chromospheric 3-minute oscillations
in NOAA AR \mynote{spell this out here?} 11158
before, during, and after the
SOL2011-02-15T01:56 X2.2 flare.
Ultraviolet emission at 1600 and 1700 Angstroms
obtained at 24-second cadence
from the Atmospheric Imaging Assembly
on board the Solar Dynamics Observatory
was used to create power maps as functions of both space and time.
A Fourier transform was applied to
the intensity signal from individual pixels
starting at each observation time
over time segments 64 frames (25.6 minutes) in length.
We detect an increase in the 3-minute power
during the X-class flare, as well as during other smaller events before and
after the flare.
The enhancement is
concentrated in small areas,
supporting the injection of energy by nonthermal particles.
The potential correlation between 3-minute power and magnetic field strength
is discussed, along with formation height dependencies.
