%\subsection{The SOL2011-02-15T01:56 flare}
The 2011 February 15 X2.2 flare
occurred in NOAA active region (AR) 11158
close to disk center
during solar cycle 24 (SOL2011-02-15T01:56).
The AR was composed of a quadrupole:
two sunspot pairs (four sunspots total).
The X-flare occurred in a delta-spot composed of
the leading spot of the southern pair and
the trailing spot of the northern pair.
It started at 01:44UT, peaked at 01:56UT, and ended at 02:06UT,
as determined by the soft X-ray flux from the
\textit{Geostationary Operational Environmental Satellite}
(\textit{GOES}-15; \cite{Viereck2007}).
The impulsive phase lasted about 10 minutes.
Data covering 5 hours centered on this flare were used for the analysis.
This data includes a C-class flare that occurred between 00:30 and 00:45 UT
on 15 February 2011.


\textit{SDO}/AIA
obtains full disk images \correction{throughout}
the solar atmosphere, using narrow band filters centered on
10 different wavelengths, two of which provide measurements of
thermal UV emission from the \correction{chromosphere}.
The 1700\AA{} channel \correction{mostly contains} continuum emission from the
temperature minimum, and
the 1600\AA{} channel \correction{covers}
both continuum emission and the \ion{C}{4} spectral line in the upper
photosphere and transition region.
Both channels have a cadence of
24 seconds and \correction{spatial size scale} of 0.6 arcseconds per pixel.

Data from
the Helioseismic and Magnetic Imager (HMI; \cite{Scherrer2012}),
also on board \textit{SDO}, is used to \correction{study} potential
correlations between magnetic field strength and oscillatory
behavior in the chromosphere.
HMI obtains full disk data in the form of
line-of-sight magnetograms, vector magnetograms,
Doppler velocity, and continuum intensity,
measured at the \ion{Fe}{1} absorption line at 6173\AA{}
with a passband width of 0.076\AA{}.
Each \correction{data product} has a cadence of 45 seconds (with the exception
of the vector magnetograms, at 135 seconds),
and \correction{spatial size scale} of 0.5 arcseconds per pixel
\citep{Schou2012}.

The standard data reduction routine
\textit{aia\_prep.pro} from solarsoft was
\correction{applied to all data}.

\begin{figure*}[htb!]\centering
    \includegraphics[draft=false,width=1.0\textwidth]{lc.pdf}\\
    \includegraphics[draft=false,width=1.0\textwidth]{lc_goes.pdf}
    \caption{%
        Top: Light curves of the
        UV continuum emission from AIA 1600\AA{} (blue curve) and
        AIA 1700\AA{} (red curve),
        integrated over the flare region in AR 11158.
        %Bottom: Light curves from top panel normalized between 0.0 and 1.0,
        %overlaid with flux from the \textit{GOES} 1-8\AA{} channel.
        Bottom: Light curves from \textit{GOES-15}
        channels 1-8\AA{} (black curve) and 0.5-4\AA{} (pink curve),
        scaled as log(flux) to enable visibility of the increases
        during smaller events before and
        after the main X-flare.
        \label{lc}}
\end{figure*}

%\begin{figure*}[htb!]\centering
%    \includegraphics[draft=false,width=1.0\textwidth]{lc_log_20181128.pdf}\\
%    \includegraphics[draft=false,width=1.0\textwidth]{lc_goes.pdf}
%    \caption{%
%        Top:
%        Same as top panel of Figure~\ref{lc}, but with AIA emission in
%        log space to obtain a better comparison to the SXR emission from
%        \textit{GOES}.
%        \label{lc_log}}
%\end{figure*}

Figure~\ref{lc} shows light curves for the full 5-hour time series
from 00:00 to 04:59 on 2011-February-15.
The top panel shows both AIA channels.
The bottom panel shows both SXR channels from \textit{GOES-15} at
1-8\AA{} (black curve) and 0.5-4\AA{} (pink curve).
%overlaid and similarly scaled for comparison.
A small C-flare occurred before the X-flare between 00:30 and 00:45 UT, and
two small events occurred after the X-flare,
between 03:00 and 03:15, and between 04:25 and 04:45.

\begin{figure*}[htb!]\centering
    \includegraphics[draft=false,width=1.0\textwidth]{images.pdf}
    \caption{
        Images of active region 11158 in AIA 1600\AA{} (left panels),
        AIA 1700\AA{} (middle panels), and HMI LOS magnetogram (right panels),
        scaled to $\pm300$ Gauss.
        The top panels show the full disk,
        and the bottom panels show the region used for analysis in this study.
        \label{images}}
\end{figure*}

Pre-flare images of the full disk are shown in Figure~\ref{images},
along with a 300x198 arcsecond subset of the data centered on AR 11158.
\correction{This subset was extracted and aligned by cross correlation}
\citep{McAteer2003,McAteer2004}.
Images were scaled to improve contrast using the
\textit{aia\_intscale.pro} routine from \textit{sswidl}.
The magnetic configuration of the quadrupole is clear in the HMI magnetograms.
The northern pair will be designated as AR\_1
and the southern pair will be designated as AR\_2.
Sunspots in the northern pair will be designated as AR\_1a (positive polarity)
and AR\_1b (negative polarity).
Sunspots in the southern pair will be designated as AR\_2a (positive polarity)
and AR\_2b (negative polarity).

Both AIA channels saturated ($\geq\!15000$ counts) in the center
during the peak of the X-class flare, and a few pixels also saturated
during the smaller events before and after.
Affected pixels were \correction{all} contained within the
300x198 arcsecond \correction{subset of data}
throughout the duration of the time series.
Four images from the 1700\AA{} channel on AIA were missing, between
the images with start times at
00:59:53.12,
01:59:29.12,
02:59:05.12, and
03:58:41.12, and the following images, each with start times
48 seconds after the previous image.
Since the gaps in data were separated by an hour,
it was reasonable to approximate \correction{missing images}
by averaging the two adjacent images.
