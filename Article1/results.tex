
\subsection{Spatial distribution of oscillatory power}

Many of the general patterns in the spatial distribution of 3-minute power
were consistent throughout the time series.
Figure \ref{contours} is presented as an illustrative example,
showing post-flare
intensity images and corresponding power maps for
AIA 1600\AA{} and AIA 1700\AA{}, overlaid with
HMI $B_{LOS}$ contours at $\pm$300 Gauss
from the center of the time segment used to produce the power map
(around 3:12 UT).
Negative polarities are outlined in black and
positive polarities are outlined in white.

The location of power enhancement appears to be correlated with
intensity, though not all locations of high intensity are accompanied by
enhanced power.
These locations are also constrained to areas
directly around the active region, but do not appear to branch out into the quiescent
network/internetwork regions beyond.
Power enhancement occurs in relatively \emph{small} regions.
A few isolated regions of enhancement covered an area of
$\sim 6" \times 6"$.
The 3-minute power appears to be suppressed in most of the areas
directly over the umbra.


\subsubsection{Correlation with $B_{LOS}$}

In most power maps, the power enhancement
is located along the boundaries of magnetic field
strength = $\pm$300 Gauss,
which correlates with the outer boundary of the penumbra.
(It should be noted that the alignment procedures may have resulted in a slight
offset between the channels, in addition to any existing LOS affects.)

\subsection{Temporal evolution of oscillatory power}\label{time}

\subsubsection{Pre-flare}

Figure~\ref{before} shows the spatial distribution of 3-minute power for
various time segments of interest from 00:00 UT up to 01:44 UT.
This period includes a C-class flare that occurred an hour before the
X-class flare in AR\_2b.
A particularly prominent spot emerges in the lower edge of AR\_1a
and persists until the decay phase of the X-class flare.


\subsubsection{During the flare}

Figure~\ref{during}
shows the spatial distribution of 3-minute power for
various time segments of interest during the X-class flare.
Pixels that saturated were excluded from the maps
to improve contrast between the remaining pixels.
Bleeding that occurred at the edges of saturated areas could not
be excluded since these values started to approach those held
by pixels that were not saturated.

Maps (a)-(c) were produced from intensity images that
included pre-flare and precursor emission, before the channels
began to saturate.
They are as close in time as allowed by the instrumental cadence,
and there is still a noticeable increase in enhancement in the flare site
over AR\_1b and AR\_2a.

Maps (d)-(f) include emission from the impulsive phase,
and maps (g)-(i) include emission from AIA peak emission and
onward through the decay phase.


\subsubsection{Post-flare}

Figure~\ref{after}
shows the spatial distribution of 3-minute power for
various time segments of interest during the X-class flare.
There were two small events after the X-class flare.

\subsubsection{Discrete wavelet analysis}

The technique described in \S\ref{analysis} was applied
to the integrated flux from AR 11158 at
discrete intervals of $T$ = 64 images with no overlap
(i.e. for start time $t_{0}$, then $t_{1} = t_{0}+T$, etc.).
This provided a ``quick and dirty'' way
to compare the spectral power at a range of frequencies.
This method produces similar results to those
obtained with wavelet analysis, though
at lower resulting frequency and time resolution.
The results are are shown in
Figure~\ref{wa}
for frequencies between 2.5 and 20.0 mHz (400 and 50 seconds, respectively).
The central frequency $\nu_{c} = 5.56$ mHz
and the frequency bandpass $\Delta\nu$ at 5 and 6 mHz
are marked by the horizontal dashed lines.
The power at all frequencies
appears to be enhanced during the X-flare compared to their
non-flaring power before and after.
%This implies that the chromospheric plasma oscillates at a range
%of frequencies in response to energy injection.
During the small events before and after the flare,
the power at lower frequencies is enhanced,
but the power at higher frequencies is suppressed relative to
the same frequencies for adjacent time segments.

At all points in time when power enhancement occurs for any frequency,
there appears to be a correlation with flux increase.


The location of these enhancements does not move across the AR.
Rather, it remains in one place until it fades away.
One of the most prominent locations of enhanced power occurs before
the flare at the bottom of the leading sunspot in the northern pair.
The change in location with time of both flare intensity and 3-minute power
implies that the source of the beam of non-thermal particles changes as well.

% Damping behavior of 3-minute oscillations (maybe)
3-minute oscillations are interpreted as slow, propagating magnetoacoustic waves,
which have a characteristic excitation mechanism and damping rate,
depending on the local plasma conditions where they originate.
The timescales over which the oscillatory power is expected to change
depends on the nature of the oscillations themselves,
or maybe the cooling rate of the plasma.
The expected timescales would depend on the cooling rate of the plasma
(images) and the damping time/mechanisms of the 3-minute oscillations.

\subsubsection{AIA formation heights/temperatures}

The evolution of the 3-minute power with time
was calculated from the power maps
by summing over $x$ and $y$ in each map $P(x,y,t_{i})$, and taking the total
to be the 3-minute power of the active region during each time segment.
This is shown in Figure~\ref{power_vs_time}.
Each point is plotted as a function of the center of the time segment
over which the Fourier transform was applied to obtain that point.
The axis labelled $T=64$ shows the scale for this length of time.

Power as a function of time obtained from total flux and power maps
is shown to check for possible contradictions between the two.
Integrating flux over the AR
before applying the Fourier transform
has the potential effect of reducing or canceling signal from
pixels whose intensity variations are out of phase.

The reason for the apparent periodicity in the plots of 3-minute power with time
is unclear.
The calculations were repeated for power centered on the
5-minute period and the 2-minute period,
and resulted in the same trend,
with the additional observation
that the length of the central period
scaled with the period in the plot of power vs. time.
This pattern is attributed to computational effects.



The amplitude of the small-scale variations in 3-minute
power is higher for AIA 1700\AA{} almost everywhere with
the exception of the main phase of the X-class flare.

%Since each point in the temporal plots represents
%the behavior of a much longer time segment than the plot seems to imply,
%a running average over the length of time segment $T$
%was computed from
%the data obtained from summing the power maps,
%and is shown in Figure~\ref{power_vs_time}.
%This greatly smoothed the periodicity in the power,
%but with further loss of temporal resolution.

When both plots are normalized between 0.0 and 1.0, the variation in
power is higher from AIA 1700 almost everywhere except the main phase of the
X-class flare, when the power from 1600 is slightly higher.

The power from AIA 1700 is higher than AIA 1600 by
1000 counts at all points throughout the time series.
Compared to 1700\AA{},
the 1600\AA{} 3-minute power appears to increase more
(relative to its own minimum) and at a faster rate.
The standard deviation for $P(t)$ from integrated flux
for 1600\AA{} is $4.9 \times 10^{4}$, and
for 1700\AA{} is $2.7 \times 10^{4}$.

If the emission from AIA 1600\AA{}
originates from a higher location in the atmosphere
than the 1700\AA{} emission,
a possible explanation for the higher, sharper increase
is that the energy from the non-thermal particle beam
dissipates as it travels through deeper layers of the chromosphere.
Because AIA 1700 originates from a deeper layer, it is probably more dense
as well, which would cause the plasma to cool at a slower rate.
If the 3-minute power originated from the transition region rather than the
upper photosphere, this may coincide with the levitation of hot chromospheric
plasma upward into post-flare loops.
Although the emission from AIA 1700\AA{} is generally thought to originate
in lower formation heights than emission from 1600\AA{},
the latter spans a broader temperature range, and contains
emission from \ion{C}{4} line.
Determination of the AIA 1600\AA{} formation height is more complicated
during flares because the \ion{C}{4} is more likely to be contributing
to the signal, and both channels may be sampling at deeper layers than
they are thought to during non-flaring times.

The persistence of the 3-minute power toward the end of the gradual phase
in AIA 1700\AA{} is consistent with the results of the wavelet analysis
carried out by \cite{Milligan2017}.

\clearpage


\begin{figure*}[htb!]\centering
    \includegraphics[draft=false,width=0.48\textwidth]{aia1600big_image.pdf}
    \includegraphics[draft=false,width=0.48\textwidth]{aia1700big_image.pdf}\\
    \includegraphics[draft=false,width=0.48\textwidth]{aia1600big_map.pdf}
    \includegraphics[draft=false,width=0.48\textwidth]{aia1700big_map.pdf}
    \caption{%
        Top row:
        Post-flare images, shown as a composite product over
        the time range shown, overlaid with contours showing the approximate
        location of the $B_{LOS}$ at $\pm$300 Gauss.
        Bottom rows: Spatial distribution of 3-minute power, obtained
        by applying a pixel-by-pixel Fourier transform over the images
        included in the top row.
        The left column shows results from AIA 1600\AA{}, and
        the right column shows results from AIA 1700\AA{}.
        White and black contours represent positive and negative polarity,
        respectively.
    \label{contours}}
\end{figure*}

\begin{figure*}[htb!]\centering
    \includegraphics[draft=false,width=0.9\textwidth]{aia1600before_20181204.pdf}
    \includegraphics[draft=false,width=0.9\textwidth]{aia1700before_20181204.pdf}
    \caption{%
        Spatial distribution of 3-minute power before the X-class flare,
        in log scale to bridge large contrasts.
        Locations whose time segment included saturated pixels were set to zero.
        \label{before}}
\end{figure*}

\begin{figure*}[htb!]\centering
    \includegraphics[draft=false,width=0.9\textwidth]{aia1600during_20181204.pdf}
    \includegraphics[draft=false,width=0.9\textwidth]{aia1700during_20181204.pdf}
    \caption{%
        Spatial distribution of 3-minute power during the X-class flare,
        in log scale to bridge large contrasts.
        Locations whose time segment included saturated pixels were set to zero.
        \label{during}}
\end{figure*}

\begin{figure*}[htb!]\centering
    \includegraphics[draft=false,width=0.9\textwidth]{aia1600after_20181204.pdf}
    \includegraphics[draft=false,width=0.9\textwidth]{aia1700after_20181204.pdf}
    \caption{%
        Spatial distribution of 3-minute power after the X-class flare,
        in log scale to bridge large contrasts.
        Locations whose time segment included saturated pixels were set to zero.
        \label{after}}
\end{figure*}

\begin{figure*}[htb!]\centering
    \includegraphics[draft=false,width=1.00\textwidth]{time-3minpower_flux.pdf}
    \includegraphics[draft=false,width=1.00\textwidth]{time-3minpower_maps.pdf}
    \caption{%
        Temporal evolution of the 3-minute power $P(t)$ in
        AIA 1600\AA{} (green curve) and AIA 1700\AA{} (purple curve).
        Top: $P(t)$ obtained by applying a Fourier transform to the
        integrated flux from AR 11158.
        Bottom: $P(t)$ per unsaturated pixel, obtained by summing over power maps.
        Each point in time
        %represents the sum (in x and y) of each power map and
        is plotted as a function of the center
        of the time segment over which the Fourier transform was applied to
        obtain the power map over which the point was summed.
        %The axis labeled $T = 64$ is scaled to show the
        %length of each time segment relative to the full time series.
        The vertical dashed lines mark the \textit{GOES} start, peak, and end
        times of the flare at 01:44, 01:56, and 02:06 UT, respectively.
        \label{power_vs_time}}
\end{figure*}

\begin{figure*}[htb!]\centering
    \includegraphics[draft=false,width=0.9\textwidth]{wa.pdf}
    \caption{
        Time-frequency power plots from AIA 1600\AA{} (top panel) and AIA
        1700\AA{} (bottom panel), obtained by applying a Fourier transform to
        integrated emission from NOAA AR 11158 in discrete time increments of
        64 frames ($\sim$25.6 minutes) each. The dashed horizontal line marks the
        central frequency $\nu_{c}$ at $\sim$5.6 mHz, corresponding to a period
        of 3 minutes. The dotted horizontal lines on either side of $\nu_{c}$
        mark the edges of the frequency bandpass $\Delta\nu$ = 1 mHz. The
        vertical lines mark the flare, start, peak, and end times as determined
        by \textit{GOES}. The power is scaled logarithmically and over the same
        range in both channels.
        \label{wa}}
\end{figure*}
