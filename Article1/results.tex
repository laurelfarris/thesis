%%\subsection{Power maps} ... ? maybe.

The following section first discusses powermap results that were
consistent throughout the time series, followed by before, during, and after the
X-class flare.



Figures~\ref{before}, \ref{during}, and \ref{after}
each show intensity images and corresponding
spatial distibution of the 3-minute power
in AIA 1600\AA{} and AIA 1700\AA{}
before, during, and after the X-class flare, respectively.
The contours overlaid on each image represent the
HMI $B_{LOS}$ magnetogram at $\pm$300 Gauss,
which corresponds
closely with the boundary of the outer penumbra.
This also helps to identify possible correlations between
oscillatory power and magnetic field strength.
Both the contour data and the intensity maps were
obtained by averaging over the same time segment 
from which the power map was computed.
(It should be noted that the alignment procedures may have resulted in a slight
offset between the channels, in addition to any existing LOS affects.)

The particular time segments presented here
were chosen based on the presence of interesting features
in the power maps
or distinct patterns in the light curve.
Several key observations in the spatial distribution of
3-minute power
were relatively consistent throughout the time series.


% Power contained in AR, not network/internetwork
The power appears constrained to the immediate vicinity of the
active region; there is very little enhancement beyond this
(examination of the isolated power maps outside the active region
confirmed that this was not merely a visual scaling effect).


% Suppression of power over umbra center.
The 3-minute power tends to be suppressed directly over the center of the umbra,
with a ring-like pattern of enhanced power around the edges of the umbra,
similar to results from \cite{Reznikova2012} for AIA UV emission.

% SMALL regions of enhanced power
Power enhancement occurs in relatively small regions.
For example, in Figure~\ref{before}, there is a distinct bright spot
with high power along the lower edge of the penumbra boundary
of AR\_1p, with an area of $\sim6"\times6"$.

The concentration of power enhancement in small areas
located in the vicinity of sunspot umbrae suggests that
\correction{these small areas of} the chromosphere
are responding directly to the injection of energy by
a beam of non-thermal particles.
%\mynote{Umbral brightenings/dots?}


% Power and Intensity - same location
Locations of enhanced power tended to have high intensity in
the corresponding intensity image.
However, enhanced power did not necessarily appear in
every location of high intensity.
This is most clearly visible in AIA 1600\AA{}.


% Power correlation with magnetic field strength
The regions of enhanced 3-minute power
were usually located along the boundaries of magnetic field
strength at $\pm$300 Gauss,
approximately over the outer penumbra boundary.

% Spots appear and disappear in one place, no navigating.
The location of these enhancements does not move across the AR.
Rather, it remains in one place until it fades away.
One of the most prominent locations of enhanced power occurs before
the flare at the bottom of the leading sunspot in the northern pair.
The change in location with time of both flare intensity and 3-minute power
implies that the source of the beam of non-thermal particles changes as well.


%\subsection{Correlation with $B_{LOS}$}
% Uncomment this sub-section to include followup results of magnetic field
% strength correlation with osc. power.
% Visual location of power relative to HMI contours isn't really enough...
%%

\subsection{Pre-flare, 00:00-01:44 UT}


\begin{figure*}[htb!]\centering
    \includegraphics[draft=false,width=0.9\textwidth]{before_20190221.pdf}
    \caption{%
        Intensity and spatial distribution of 3-minute power
        immediately prior to the X-class flare between
        01:19 and 01:44 UT on 15 February 2011,
        in log scale to bridge large contrasts.
        Locations whose time segment included saturated pixels were set to zero.
        Contours indicate the approximate position of
        HMI B$_{LOS}$ at $\pm$300 Gauss. White and black contours represent
        positive and negative polarities, respectively.
        The dimensions of each image are the same as labeled
        on the axes in Figure~\ref{images}.
        \label{before}}
\end{figure*}

%\mynote{Provide some context here before describing maps?
%What's happening in intensity, which part of AR it's coming from,
%magnetic configuration, timeline of where the ribbons are
%(Schrijver2011), etc.}

Although this time segment took place during the precursor phase
of the X-class flare, it also took place
\emph{after} a C-class flare that occurred an hour before the X-flare,
between 00:30 and 00:45 UT,
as well as another event the day before on 10 February 2011.
%\mynote{How might C-flare have affected these results?
%Technically post-flare\ldots}
The emission from the C-flare originated from AR\_2n.

Figure~\ref{before}
shows pre-flare intensity images and corresponding power maps for
the time segment immediately prior to the \textit{GOES}
start time.
The ``J-shaped'' ribbon feature can be seen in the center of the
intensity images and the power map
along the left
%(\mynote{East/West?})
edge of the boundary of AR\_1n,
most visible in AIA 1600\AA{}.
The prominent region of enhanced power along the lower penumbral
boundary of AR\_1p
emerged earlier and persisted until the decay phase of the X-flare.
Interestingly, the C-flare occurred on the opposite side of
AR 11158.
Movies of the active region intensity leading up to the X-flare reveal
intermittent appearances of high intensity here. Due to the length
of $T$, the enhancement in power was consistently present in the power maps.
any activity in this region.
Future work will involve the investigation into
the magnetic configuration of this region and
potential connections between this spot and earlier events.
Locations of enhanced power are, for the most part, located along
umbra/penumbra boundaries, though a ribbon-like shape through the
center of AR\_1n is also observed.


\subsection{During the flare, 01:44-02:30 UT}

% Flare phases, if time:
%\subsubsection{Precursor}
%\subsubsection{Impulsive}
%\subsubsection{Gradual}


\begin{figure*}[htb!]\centering
    \includegraphics[draft=false,width=0.9\textwidth]{during_20190221.pdf}
    \caption{%
        Same as Figure~\ref{before},
        during the X-class flare between
        01:45 and 02:10 UT on 15 February 2011.
        \label{during}}
\end{figure*}


Figure~\ref{during}
shows intensity images and corresponding power maps for
the time segment between 01:45 and 02:10 UT,
just after the \textit{GOES} start time (01:44 UT) until
a few minutes after the \textit{GOES} end time (02:06 UT).
This covered the maximum amount of
flare emission: from the start of the impulsive phase through as much
of the decay phase as allowed by $T$.
Pixels that saturated were excluded from the maps
to improve contrast between the remaining pixels.
%Bleeding that occurred at the edges of saturated areas could not
%be excluded since these values started to approach those held
%by pixels that were not saturated.
%

Due to the short timescales over which flare dynamics typically occur,
the instrumental sampling rate here is
too low to obtain sufficient time resolution.
The sampling time $T\approx$25 minutes used here
did not allow the isolation of any one flare phase
as it was a few minutes longer than the entire duration between
the official \textit{GOES} start and end times.
(about 22 minutes).
While the choice of $T$ was necessary to obtain sufficient frequency resolution,
this does come
at the expense of temporal resolution.
The enhanced emission at the 3-minute period in integrated emission
observed by \cite{Milligan2017} appeared to start around 01:44 and faded
beyond the 99\% confidence limit by $\sim$02:02, with slightly enhanced
power beyond this until $\sim$02:10 in AIA 1700\AA{}.
These timescales over which the oscillatory power changed
were much shorter than the sample time length used here.

Due to the large number of pixels that saturated during the flare,
not much spatial information can be extracted from power maps obtained
during the time that enhanced chromospheric emission was previously observed.
peak flare emission, particularly
from the AR center and parts of the outside sunspots, where
emission reaches further into as the flare develops.
According to the review from \cite{Inglis2009}, QPPs in flare emission
often don't last longer than three or four periods, so
extracting these oscillations would be difficult even without
saturation with the current methods.

It can be observed that the spot in AR\_1p fades away in the power maps
obtained from gradual phase only (no emission from precursor or impulsive
phases contributing to power map).


For the current data set, the peak in AIA emission occurred around
01:53 UT (01:52:41.12 for 1600\AA{}, and 01:52:55.71 for 1700\AA{}),
about 3 minutes before the \textit{GOES} peak time (01:56 UT).
Both UV light curves show a slight increase at the \textit{GOES} end time
(02:06), as well as a few other times during the decay phase.



\subsection{Post-flare, 02:30-05:00}

\begin{figure*}[htb!]\centering
    \includegraphics[draft=false,width=0.9\textwidth]{after_20190221.pdf}
    \caption{%
        Same as Figure~\ref{before},
        after the X-class flare between
        03:00 and 03:25 UT on 15 February 2011.
        \label{after}}
\end{figure*}

Figure~\ref{after}
shows post-flare intensity images and corresponding power maps for
the time segment between 03:00 and 03:25,
well into the gradual phase of the x-flare.
This time period includes the first of two small events that occurred
after the X-class flare within the five-hour time series.

This particular time segment was chosen to present as the post-flare
oscillatory power because of the presence of several distinct, small
regions of enhancement in the center,
right around AR\_2p.

%Enhanced intensity and 3-minute power are observed along the lower
%boundary of AR\_2n, as well as along the lower boundary
%of the AR center (AR\_1n and AR\_2p).


\subsection{Discrete wavelet analysis}

\begin{figure*}[htb!]\centering
    \includegraphics[draft=false,width=0.9\textwidth]{wa.pdf}
    \caption{
        Time-frequency power plots from AIA 1600\AA{} (top panel) and AIA
        1700\AA{} (bottom panel), obtained by applying a Fourier transform to
        integrated emission from NOAA AR 11158 in discrete time increments of
        64 frames ($\sim$25.6 minutes) each. The dashed horizontal line marks the
        central frequency $\nu_{c}$ at $\sim$5.6 mHz, corresponding to a period
        of 3 minutes. The dotted horizontal lines on either side of $\nu_{c}$
        mark the edges of the frequency bandpass $\Delta\nu$ = 1 mHz. The
        vertical lines mark the flare, start, peak, and end times as determined
        by \textit{GOES}. The power is scaled logarithmically and over the same
        range in both channels.
        \label{wa}}
\end{figure*}


The technique described in \S\ref{analysis} was applied
to the integrated flux from AR 11158 at
discrete intervals of $T$ = 64 images with no overlap
(i.e. for start time $t_{0}$, then $t_{1} = t_{0}+T$, etc.).
This provided a ``quick and dirty'' way
to compare the spectral power at a range of frequencies.
This method produces similar results to those
obtained with wavelet analysis, though
at lower resulting frequency and time resolution.
The results are are shown in
Figure~\ref{wa}
for frequencies between 2.5 and 20.0 mHz (400 and 50 seconds, respectively).
The central frequency $\nu_{c} = 5.56$ mHz
and the frequency bandpass $\Delta\nu$ at 5 and 6 mHz
are marked by the horizontal dashed lines.
The power at all frequencies
appears to be enhanced during the X-flare compared to their
non-flaring power before and after.
%This implies that the chromospheric plasma oscillates at a range
%of frequencies in response to energy injection.
During the small events before and after the flare,
the power at lower frequencies is enhanced,
but the power at higher frequencies is suppressed relative to
the same frequencies for adjacent time segments.

At all points in time when power enhancement occurs for any frequency,
there appears to be a correlation with flux increase.


\subsection{Temporal evolution of 3-minute power}

\begin{figure*}[htb!]\centering
    %\includegraphics[draft=false,width=1.00\textwidth]{time-3minpower_flux.pdf}
    \includegraphics[draft=false,width=1.00\textwidth]{time-3minpower_maps.pdf}
    \caption{%
        Temporal evolution of the 3-minute power $P(t)$ in
        AIA 1600\AA{} (green curve) and AIA 1700\AA{} (purple curve).
        %Top: $P(t)$ obtained by applying a Fourier transform to the
        %integrated flux from AR 11158.
        %Bottom:
        $P(t)$ per unsaturated pixel, obtained by summing over power maps
        whose pixel values = 0 if saturated, then divided by number of
        unsaturated pixels.
        Each point in time
        %represents the sum (in x and y) of each power map and
        is plotted as a function of the center
        of the time segment over which the Fourier transform was applied to
        obtain the power map over which the point was summed.
        %The axis labeled $T = 64$ is scaled to show the
        %length of each time segment relative to the full time series.
        The vertical dashed lines mark the \textit{GOES} start, peak, and end
        times of the flare at 01:44, 01:56, and 02:06 UT, respectively.
        \label{power_vs_time}}
\end{figure*}


The evolution of the 3-minute power with time
was calculated from the power maps
by summing over $x$ and $y$ in each map $P(x,y,t_{i})$, and taking the total
to be the 3-minute power of the active region during each time segment.
This is shown in Figure~\ref{power_vs_time}.
Each point is plotted as a function of the center of the time segment
over which the Fourier transform was applied to obtain that point.
The axis labelled $T=64$ shows the scale for this length of time.

Power as a function of time obtained from total flux and power maps
is shown to check for possible contradictions between the two.
Integrating flux over the AR
before applying the Fourier transform
has the potential effect of reducing or canceling signal from
pixels whose intensity variations are out of phase.

The reason for the apparent periodicity in the plots of 3-minute power with time
is unclear.
The calculations were repeated for power centered on the
5-minute period and the 2-minute period,
and resulted in the same trend,
with the additional observation
that the length of the central period
scaled with the period in the plot of power vs. time.
This pattern is attributed to computational effects.


%Since each point in the temporal plots represents
%the behavior of a much longer time segment than the plot seems to imply,
%a running average over the length of time segment $T$
%was computed from
%the data obtained from summing the power maps,
%and is shown in Figure~\ref{power_vs_time}.
%This greatly smoothed the periodicity in the power,
%but with further loss of temporal resolution.


The persistence of the 3-minute power toward the end of the gradual phase
in AIA 1700\AA{} is consistent with the results of the wavelet analysis
carried out by \cite{Milligan2017}.

%The amplitude of the small-scale variations in 3-minute
power is higher for AIA 1700\AA{} almost everywhere with
the exception of the main phase of the X-class flare.

When both plots are normalized between 0.0 and 1.0, the variation in
power is higher from AIA 1700\AA{} almost everywhere except the main phase of the
X-class flare, when the power from 1600\AA{} is slightly higher.

The power from AIA 1700 is higher than AIA 1600 by
1000 counts at all points throughout the time series.
Compared to 1700\AA{},
the 1600\AA{} 3-minute power appears to increase more
(relative to its own minimum) and at a faster rate.
The standard deviation for $P(t)$ from integrated flux
for 1600\AA{} is $4.9 \times 10^{4}$, and
for 1700\AA{} is $2.7 \times 10^{4}$.

If the emission from AIA 1600\AA{}
originates from a higher location in the atmosphere
than the 1700\AA{} emission,
a possible explanation for the higher, sharper increase
is that the energy from the non-thermal particle beam
dissipates as it travels through deeper layers of the chromosphere.
Because AIA 1700 originates from a deeper layer, it is probably more dense
as well, which would cause the plasma to cool at a slower rate.
If the 3-minute power originated from the transition region rather than the
upper photosphere, this may coincide with the levitation of hot chromospheric
plasma upward into post-flare loops.
Although the emission from AIA 1700\AA{} is generally thought to originate
in lower formation heights than emission from 1600\AA{},
the latter spans a broader temperature range, and contains
emission from \ion{C}{4} line.
Determination of the AIA 1600\AA{} formation height is more complicated
during flares because the \ion{C}{4} is more likely to be contributing
to the signal, and both channels may be sampling at deeper layers than
they are thought to during non-flaring times.



The small periodicities can be difficult to extract from the global
lightcurve \citep{VanDoorsselaere2016}.

If the location of power enhancement does reveal sites of energy injection,
then \correction{it remains possible that} changes in energy source coincides
with the location of power changes.

%\mynote{
%    Compare my results from integrated emission to others.
%    Then compare my results from spatially resolved data.
%    Does it contradict those from integrated emission?
%    Do the \emph{relative} changes in the two AIA channels match?
%}


Almost all events before and after the main X-class flare
occurred in AR\_2n, including briefly at the beginning of the time series.



% Damping behavior of 3-minute oscillations (maybe)
3-minute oscillations are interpreted as slow, propagating magnetoacoustic waves,
which have a characteristic excitation mechanism and damping rate,
depending on the local plasma conditions where they originate.
The timescales over which the oscillatory power is expected to change
depends on the nature of the oscillations themselves,
or maybe the cooling rate of the plasma.
The expected timescales would depend on the cooling rate of the plasma
(images) and the damping time/mechanisms of the 3-minute oscillations.
