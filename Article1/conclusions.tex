In this work, we have used spatially resolved data from
SDO/AIA and SDO/HMI
to constrain the location and time that enhancement in the power
of the 3-minute oscillations occurred before, during, and after
an X-class flare.
Results support the theory of energy
injection by acceleration of non-thermal particles, and the response
of the chromosphere to this injection in thermal wavelengths.

The preliminary results of this work support the following conclusions:
\begin{enumerate}
%    \item The increase in 3-minute power just before the flare onset,
%        specifically before the increase in SXR emission, suggests that the
%        chromosphere is responding to a disturbance at its natural frequency.
    \item The enhancement of 3-minute
        power is concentrated in small areas that coincide with locations
        of enhanced flare emission. This supports the theory of
        energy injection by a beam of accelerated non-thermal particles.
        It also shows that the chromospheric plasma does not oscillate
        globally as one body across the active region.
    \item The 3-minute power changes more for
        AIA 1600\AA{} than for 1700\AA{}, which suggests that the 1600\AA{}
        emission originates from higher layers.
%    \item The temporal variation in 3-minute power appears to correlate with
%        the input flux, except after the flare peak where there is a second,
%        smaller increase in power as the flare flux is still steadily decreasing.
%        This may be attributable to two different sources of energy,
%        occurring at different times.
%    \item The 3-minute power appears to have a quasi-periodic variability
%        of its own. This is likely due to the computational effects of
%        shifting the input time series forward 24 seconds for each calculation.
\end{enumerate}

%The overall magnetic configuration of this active region is not clear in these
%data. Combining magnetograms from HMI with these results will show whether the
%location of enhancement is correlated with magnetic field strength. By the
%Lorentz force, the force with which charged particles can flow along magnetic
%field lines is proportional to the magnetic field strength, so if this was in
%fact governing the amount of energy that these particles injected into the
%chromosphere, it is expected that the areas of high magnetic field strength
%would be co-spatial with the areas of enhancement. Where are the main AR
%features (umbra, penumbra) relative to the enhanced areas in the power maps? As
%time goes on (especially before and after the flare), are the 3-minute
%oscillations enhanced in different places, or always in the same spot?

%The overall magnetic configuration of this active region is not clear in these
%data. Combining magnetograms from HMI with these results will show whether the
%location of enhancement is correlated with magnetic field strength. By the
%Lorentz force, the force with which charged parti- cles can flow along magnetic
%field lines is proportional to the magnetic field strength, so if this was in
%fact governing the amount of energy that these particles injected into the
%chromosphere, it is expected that the areas of high mag- netic field strength
%would be co-spatial with the areas of enhancement. Where are the main AR
%features (umbra, penumbra) relative to the enhanced areas in the power maps? As
%time goes on (especially before and after the flare), are the 3-minute
%oscillations enhanced in different places, or always in the same spot?


%It is possible that important spatial information is contained
%in location of the flare core,
%but cannot be extracted due to the saturation in the pixels.
%Saturation does not occur as often for flares
%of less powerful classes.
%\mynote{
%    remember proposal\ldots make sure you understand the difference
%    between ``strong'' and ``weak'' flares
%    vs. ``large'' and ``small'' flares.}

%The methods used in the current work were reasonably
%successful in determining the spatial location and distribution
%of concentrated power within a timespan of about 15 minutes.

%Future work will address the question of whether the chromospheric response
%reflects the rate and magnitude of energy input, or if it is more
%characteristic of the ambient plasma than of the nature of the disturbance.

%The pre-flare data shown in this work may not be the best representation
%since another, smaller flare took place in the middle of it.
%It may be worthwhile to obtain more data at earlier times

There are several possibilities for the continuation of this work.

Here we focused on the oscillations centered around the 3-minute period,
but the inclusion of other periods in the typical range of QPPs periods
will be helpful to see how the behavior of the 3-minute oscillations differs
from others, to set it apart from the range of frequencies excited due
to energy injection.

Since several of the data images saturated during the main phase of the flare,
spatial information cannot be obtained at the core location of the flare.
It may be worthwhile to apply these methods to a less powerful flare.

The temporal behavior of oscillations during the main flare remains
inconclusive due to the necessary balance between temporal and frequency
resolution.
Techniques to improve temporal resolution,
such as the standard wavelet analysis
presented by \cite{Torrence1998},
will allow study of chromospheric behavior on timescales comparable to those
over which flare dyanmics are known to occur.

Indeed, the timescales over which the oscillatory power changed in the study by
\cite{Milligan2017}
were much shorter than the sample time length used here.
The choice of $T$ was necessary to obtain sufficient frequency resolution
with the techniques utilized here, at the expense of temporal resolution.
