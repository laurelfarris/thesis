%% Conclusions

In this work, we have presented spatial distribution of the 3-minute oscillations
associated with the X-class flare that occurred in AR11158 on 15 February 2011.
Key points are as follows:
\begin{enumerate}
    \item Small, distinct regions of enhanced power show that
        the chromospheric plasma does not oscillate as one body.
    \item Location relative to AR same as enhanced intensity associated
        with the flare, which supports the theory of oscillation in response
        to energy injection via accelerated non-thermal particles.
    \item Variation in enhancement location throughout flare phases
        indicates a possible change in source of energy input.
\end{enumerate}



%% Future work

Future work will involve smaller flares to avoid saturation issues.
It is possible that important spatial information is contained
in location of the flare core,
but cannot be extracted due to the saturation in the pixels.
Saturation does not occur as often for flares
of less powerful classes.
Since several of the data images saturated during the main phase of the flare,
spatial information cannot be obtained at the core location of the flare.
It may be worthwhile to apply these methods to a less powerful flare.

The temporal behavior of oscillations during the main flare remains
inconclusive due to the necessary balance between temporal and frequency
resolution.
Techniques to improve temporal resolution,
such as the standard wavelet analysis
presented by \cite{Torrence1998},
will allow study of chromospheric behavior on timescales comparable to those
over which flare dyanmics are known to occur.

The pre-flare data shown in this work may not be the best representation
since another, smaller flare took place in the middle of it.
It may be worthwhile to obtain more data at earlier times

Acknowledgements: ?
