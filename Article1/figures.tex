\begin{figure*}[htb!]\centering
    \includegraphics[draft=false,width=0.48\textwidth]{aia1600big_image.pdf}
    \includegraphics[draft=false,width=0.48\textwidth]{aia1700big_image.pdf}\\
    \includegraphics[draft=false,width=0.48\textwidth]{aia1600big_map.pdf}
    \includegraphics[draft=false,width=0.48\textwidth]{aia1700big_map.pdf}
    \caption{%
        Top row:
        Post-flare images, shown as a composite product over
        the time range shown, overlaid with contours showing the approximate
        location of the $B_{LOS}$ at $\pm$300 Gauss.
        Bottom rows: Spatial distribution of 3-minute power, obtained
        by applying a pixel-by-pixel Fourier transform over the images
        included in the top row.
        The left column shows results from AIA 1600\AA{}, and
        the right column shows results from AIA 1700\AA{}.
        White and black contours represent positive and negative polarity,
        respectively.
    \label{contours}}
\end{figure*}

\begin{figure*}[htb!]\centering
    \includegraphics[draft=false,width=0.9\textwidth]{aia1600before_20181204.pdf}
    \includegraphics[draft=false,width=0.9\textwidth]{aia1700before_20181204.pdf}
    \caption{%
        Spatial distribution of 3-minute power before the X-class flare,
        in log scale to bridge large contrasts.
        Locations whose time segment included saturated pixels were set to zero.
        \label{before}}
\end{figure*}

\begin{figure*}[htb!]\centering
    \includegraphics[draft=false,width=0.9\textwidth]{aia1600during_20181204.pdf}
    \includegraphics[draft=false,width=0.9\textwidth]{aia1700during_20181204.pdf}
    \caption{%
        Spatial distribution of 3-minute power during the X-class flare,
        in log scale to bridge large contrasts.
        Locations whose time segment included saturated pixels were set to zero.
        \label{during}}
\end{figure*}

\begin{figure*}[htb!]\centering
    \includegraphics[draft=false,width=0.9\textwidth]{aia1600after_20181204.pdf}
    \includegraphics[draft=false,width=0.9\textwidth]{aia1700after_20181204.pdf}
    \caption{%
        Spatial distribution of 3-minute power after the X-class flare,
        in log scale to bridge large contrasts.
        Locations whose time segment included saturated pixels were set to zero.
        \label{after}}
\end{figure*}




\begin{figure*}[htb!]\centering
    \includegraphics[draft=false,width=1.00\textwidth]{time-3minpower_flux.pdf}
    \includegraphics[draft=false,width=1.00\textwidth]{time-3minpower_maps.pdf}
    \caption{%
        Temporal evolution of the 3-minute power $P(t)$ in
        AIA 1600\AA{} (green curve) and AIA 1700\AA{} (purple curve).
        Top: $P(t)$ obtained by applying a Fourier transform to the
        integrated flux from AR 11158.
        Bottom: $P(t)$ per unsaturated pixel, obtained by summing over power maps.
        Each point in time
        %represents the sum (in x and y) of each power map and
        is plotted as a function of the center
        of the time segment over which the Fourier transform was applied to
        obtain the power map over which the point was summed.
        %The axis labeled $T = 64$ is scaled to show the
        %length of each time segment relative to the full time series.
        The vertical dashed lines mark the \textit{GOES} start, peak, and end
        times of the flare at 01:44, 01:56, and 02:06 UT, respectively.
        \label{power_vs_time}}
\end{figure*}

\begin{figure*}[htb!]\centering
    \includegraphics[draft=false,width=0.9\textwidth]{wa.pdf}
    \caption{
        Time-frequency power plots from AIA 1600\AA{} (top panel) and AIA
        1700\AA{} (bottom panel), obtained by applying a Fourier transform to
        integrated emission from NOAA AR 11158 in discrete time increments of
        64 frames ($\sim$25.6 minutes) each. The dashed horizontal line marks the
        central frequency $\nu_{c}$ at $\sim$5.6 mHz, corresponding to a period
        of 3 minutes. The dotted horizontal lines on either side of $\nu_{c}$
        mark the edges of the frequency bandpass $\Delta\nu$ = 1 mHz. The
        vertical lines mark the flare, start, peak, and end times as determined
        by \textit{GOES}. The power is scaled logarithmically and over the same
        range in both channels.
        \label{wa}}
\end{figure*}

